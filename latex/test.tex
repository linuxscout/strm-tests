
\section{Question}


\paragraph{Q1}






Convert the following numbers, \hfill\aRL{أنجز التحويلات الآتية}

$(193)_{ 8 } = (........)_{ 2}$



\paragraph{Q2}

Donner les intervalles qu'on peut représenter en nombre positifs, valeur absolue, complément à 1 et complément à 2  sur 52 bits

Give the intervals which can be represented in posiitve numbers, absolute value, 1's complement and 2's complement on 52 bits

\begin{arab}[utf]
حدد المجالات التي يمكن تمثيلها لأعداد الموجبة والتمثيل بالقيمة المطلقة والمتمم إلى 1 و 2 على   : 52 بت
\end{arab}



\paragraph{Q3}

Calculate  the following operations in base 8 :  أنجز العمليات الآتية في الأساس 8

\begin{verbatim}

  3530\,0315
- 1251\,2130
-------------------
=  ........

\end{verbatim}


\hrule width 1\linewidth
\pagebreak

\subsection{Correction}


\paragraph{Q1}






Convert the following numbers, \hfill\aRL{أنجز التحويلات الآتية}

$(193)_{ 8 } = (  1100\,0001)_{ 2}$



\textbf{Explanation of Base 10 to X}\\
\begin{tikzpicture}[x=1cm,y=1cm]
  % Draw title info
      \begin{scope}[shift={(0cm,-0cm)}]
  % Vertical bar
  \draw (0.75,0) -- (0.75,-2);
  % Horizontal bar
  \draw (0.75,-1) -- (1.5,-1);
  % Dividend
    \node[anchor=west] at (0,-0.5*1) { 193 };
  % Divisor
    \node[anchor=west] at (0.95,-0.5*1) { 2 };
  % Quotient
    \node[anchor=west] at (0.95,-1.5*1) { 96 };
  % Remainder
    \node[anchor=center] at (0.375,-1.5*1) { 1 };
\end{scope}

    \begin{scope}[shift={(0.75cm,-1cm)}]
  % Vertical bar
  \draw (0.75,0) -- (0.75,-2);
  % Horizontal bar
  \draw (0.75,-1) -- (1.5,-1);
  % Dividend
  % Divisor
    \node[anchor=west] at (0.95,-0.5*1) { 2 };
  % Quotient
    \node[anchor=west] at (0.95,-1.5*1) { 48 };
  % Remainder
    \node[anchor=center] at (0.375,-1.5*1) { 0 };
\end{scope}

    \begin{scope}[shift={(1.5cm,-2cm)}]
  % Vertical bar
  \draw (0.75,0) -- (0.75,-2);
  % Horizontal bar
  \draw (0.75,-1) -- (1.5,-1);
  % Dividend
  % Divisor
    \node[anchor=west] at (0.95,-0.5*1) { 2 };
  % Quotient
    \node[anchor=west] at (0.95,-1.5*1) { 24 };
  % Remainder
    \node[anchor=center] at (0.375,-1.5*1) { 0 };
\end{scope}

    \begin{scope}[shift={(2.25cm,-3cm)}]
  % Vertical bar
  \draw (0.75,0) -- (0.75,-2);
  % Horizontal bar
  \draw (0.75,-1) -- (1.5,-1);
  % Dividend
  % Divisor
    \node[anchor=west] at (0.95,-0.5*1) { 2 };
  % Quotient
    \node[anchor=west] at (0.95,-1.5*1) { 12 };
  % Remainder
    \node[anchor=center] at (0.375,-1.5*1) { 0 };
\end{scope}

    \begin{scope}[shift={(3.0cm,-4cm)}]
  % Vertical bar
  \draw (0.75,0) -- (0.75,-2);
  % Horizontal bar
  \draw (0.75,-1) -- (1.5,-1);
  % Dividend
  % Divisor
    \node[anchor=west] at (0.95,-0.5*1) { 2 };
  % Quotient
    \node[anchor=west] at (0.95,-1.5*1) { 6 };
  % Remainder
    \node[anchor=center] at (0.375,-1.5*1) { 0 };
\end{scope}

    \begin{scope}[shift={(3.75cm,-5cm)}]
  % Vertical bar
  \draw (0.75,0) -- (0.75,-2);
  % Horizontal bar
  \draw (0.75,-1) -- (1.5,-1);
  % Dividend
  % Divisor
    \node[anchor=west] at (0.95,-0.5*1) { 2 };
  % Quotient
    \node[anchor=west] at (0.95,-1.5*1) { 3 };
  % Remainder
    \node[anchor=center] at (0.375,-1.5*1) { 0 };
\end{scope}

    \begin{scope}[shift={(4.5cm,-6cm)}]
  % Vertical bar
  \draw (0.75,0) -- (0.75,-2);
  % Horizontal bar
  \draw (0.75,-1) -- (1.5,-1);
  % Dividend
  % Divisor
    \node[anchor=west] at (0.95,-0.5*1) { 2 };
  % Quotient
    \node[anchor=west] at (0.95,-1.5*1) { 1 };
  % Remainder
    \node[anchor=center] at (0.375,-1.5*1) { 1 };
\end{scope}

    \begin{scope}[shift={(5.25cm,-7cm)}]
  % Vertical bar
  \draw (0.75,0) -- (0.75,-2);
  % Horizontal bar
  \draw (0.75,-1) -- (1.5,-1);
  % Dividend
  % Divisor
    \node[anchor=west] at (0.95,-0.5*1) { 2 };
  % Quotient
    \node[anchor=west] at (0.95,-1.5*1) { 0 };
  % Remainder
    \node[anchor=center] at (0.375,-1.5*1) { 1 };
\end{scope}

\end{tikzpicture}


Result (bottom→top remainders): 11000001




\paragraph{Q2}

Donner les intervalles qu'on peut représenter en nombre positifs, valeur absolue, complément à 1 et complément à 2  sur 52 bits

Give the intervals which can be represented in posiitve numbers, absolute value, 1's complement and 2's complement on 52 bits

\begin{arab}[utf]
حدد المجالات التي يمكن تمثيلها لأعداد الموجبة والتمثيل بالقيمة المطلقة والمتمم إلى 1 و 2 على   : 52 بت
\end{arab}

\begin{itemize}
\item \textbf{Positifs}: $[0; 2^{ 52-1 }] = [0; 4503599627370495]$
\item \textbf{Unsigned value} $[-(2^{ 51 }-1 );2^{ 51 }-1] = [-2251799813685247, 2251799813685247]$
\item \textbf{One's compelement} $[-(2^{ 51 }-1 );2^{ 51 }-1] = [-2251799813685247, 2251799813685247]$
\item \textbf{Two's compelement} $[-2^{ 51 } ;2^{ 51 }-1] = [-2251799813685248, 2251799813685247]$
\end{itemize}


\paragraph{Q3}

Calculate  the following operations in base 8 :  أنجز العمليات الآتية في الأساس 8

\begin{verbatim}

  3530\,0315
- 1251\,2130
-------------------
=  2256\,6165

\end{verbatim}
\pagebreak

\paragraph{Q1}

Faire les conversion suivantes: NOT IMPLEMENTED (MESURE questions) 

\paragraph{Q2}






Convert the following numbers, \hfill\aRL{أنجز التحويلات الآتية}

$(9\,4275\,8564)_{ 2 } = (........)_{ 16}$



\paragraph{Q3}

Calculate  the following operations in base 16 :  أنجز العمليات الآتية في الأساس 16

\begin{verbatim}

  bce6\,c463
- a15\,8c8d
-------------------
=  ........

\end{verbatim}


\hrule width 1\linewidth
\pagebreak

\subsection{Correction}


\paragraph{Q1}

  44410010
- 32441214
----------
  11413241


\paragraph{Q2}






Convert the following numbers, \hfill\aRL{أنجز التحويلات الآتية}

$(9\,4275\,8564)_{ 2 } = (  3831\,5aa4)_{ 16}$



\textbf{Explanation of Base 10 to X}\\
\begin{tikzpicture}[x=1cm,y=1cm]
  % Draw title info
      \begin{scope}[shift={(0cm,-0cm)}]
  % Vertical bar
  \draw (2.25,0) -- (2.25,-2);
  % Horizontal bar
  \draw (2.25,-1) -- (4.5,-1);
  % Dividend
    \node[anchor=west] at (0,-0.5*1) { 942758564 };
  % Divisor
    \node[anchor=west] at (2.45,-0.5*1) { 16 };
  % Quotient
    \node[anchor=west] at (2.45,-1.5*1) { 58922410 };
  % Remainder
    \node[anchor=center] at (1.125,-1.5*1) { 4 };
\end{scope}

    \begin{scope}[shift={(2.25cm,-1cm)}]
  % Vertical bar
  \draw (2.25,0) -- (2.25,-2);
  % Horizontal bar
  \draw (2.25,-1) -- (4.5,-1);
  % Dividend
  % Divisor
    \node[anchor=west] at (2.45,-0.5*1) { 16 };
  % Quotient
    \node[anchor=west] at (2.45,-1.5*1) { 3682650 };
  % Remainder
    \node[anchor=center] at (1.125,-1.5*1) { 10 };
\end{scope}

    \begin{scope}[shift={(4.5cm,-2cm)}]
  % Vertical bar
  \draw (2.25,0) -- (2.25,-2);
  % Horizontal bar
  \draw (2.25,-1) -- (4.5,-1);
  % Dividend
  % Divisor
    \node[anchor=west] at (2.45,-0.5*1) { 16 };
  % Quotient
    \node[anchor=west] at (2.45,-1.5*1) { 230165 };
  % Remainder
    \node[anchor=center] at (1.125,-1.5*1) { 10 };
\end{scope}

    \begin{scope}[shift={(6.75cm,-3cm)}]
  % Vertical bar
  \draw (2.25,0) -- (2.25,-2);
  % Horizontal bar
  \draw (2.25,-1) -- (4.5,-1);
  % Dividend
  % Divisor
    \node[anchor=west] at (2.45,-0.5*1) { 16 };
  % Quotient
    \node[anchor=west] at (2.45,-1.5*1) { 14385 };
  % Remainder
    \node[anchor=center] at (1.125,-1.5*1) { 5 };
\end{scope}

    \begin{scope}[shift={(9.0cm,-4cm)}]
  % Vertical bar
  \draw (2.25,0) -- (2.25,-2);
  % Horizontal bar
  \draw (2.25,-1) -- (4.5,-1);
  % Dividend
  % Divisor
    \node[anchor=west] at (2.45,-0.5*1) { 16 };
  % Quotient
    \node[anchor=west] at (2.45,-1.5*1) { 899 };
  % Remainder
    \node[anchor=center] at (1.125,-1.5*1) { 1 };
\end{scope}

    \begin{scope}[shift={(11.25cm,-5cm)}]
  % Vertical bar
  \draw (2.25,0) -- (2.25,-2);
  % Horizontal bar
  \draw (2.25,-1) -- (4.5,-1);
  % Dividend
  % Divisor
    \node[anchor=west] at (2.45,-0.5*1) { 16 };
  % Quotient
    \node[anchor=west] at (2.45,-1.5*1) { 56 };
  % Remainder
    \node[anchor=center] at (1.125,-1.5*1) { 3 };
\end{scope}

    \begin{scope}[shift={(13.5cm,-6cm)}]
  % Vertical bar
  \draw (2.25,0) -- (2.25,-2);
  % Horizontal bar
  \draw (2.25,-1) -- (4.5,-1);
  % Dividend
  % Divisor
    \node[anchor=west] at (2.45,-0.5*1) { 16 };
  % Quotient
    \node[anchor=west] at (2.45,-1.5*1) { 3 };
  % Remainder
    \node[anchor=center] at (1.125,-1.5*1) { 8 };
\end{scope}

    \begin{scope}[shift={(15.75cm,-7cm)}]
  % Vertical bar
  \draw (2.25,0) -- (2.25,-2);
  % Horizontal bar
  \draw (2.25,-1) -- (4.5,-1);
  % Dividend
  % Divisor
    \node[anchor=west] at (2.45,-0.5*1) { 16 };
  % Quotient
    \node[anchor=west] at (2.45,-1.5*1) { 0 };
  % Remainder
    \node[anchor=center] at (1.125,-1.5*1) { 3 };
\end{scope}

\end{tikzpicture}


Result (bottom→top remainders): 38315aa4




\paragraph{Q3}

Calculate  the following operations in base 16 :  أنجز العمليات الآتية في الأساس 16

\begin{verbatim}

  bce6\,c463
- a15\,8c8d
-------------------
=  b2d1\,37d6

\end{verbatim}
\pagebreak

\paragraph{Q1}






Convert the following numbers, \hfill\aRL{أنجز التحويلات الآتية}

$(24)_{ 8 } = (........)_{ 2}$



\paragraph{Q2}

Unknown command: mesures

\paragraph{Q3}

Calculate  the following operations in base 8 :  أنجز العمليات الآتية في الأساس 8

\begin{verbatim}

  5341\,6306
- 2254\,5644
-------------------
=  ........

\end{verbatim}


\hrule width 1\linewidth
\pagebreak

\subsection{Correction}


\paragraph{Q1}






Convert the following numbers, \hfill\aRL{أنجز التحويلات الآتية}

$(24)_{ 8 } = (  1\,1000)_{ 2}$



\textbf{Explanation of Base 10 to X}\\
\begin{tikzpicture}[x=1cm,y=1cm]
  % Draw title info
      \begin{scope}[shift={(0cm,-0cm)}]
  % Vertical bar
  \draw (0.5,0) -- (0.5,-2);
  % Horizontal bar
  \draw (0.5,-1) -- (1.0,-1);
  % Dividend
    \node[anchor=west] at (0,-0.5*1) { 24 };
  % Divisor
    \node[anchor=west] at (0.7,-0.5*1) { 2 };
  % Quotient
    \node[anchor=west] at (0.7,-1.5*1) { 12 };
  % Remainder
    \node[anchor=center] at (0.25,-1.5*1) { 0 };
\end{scope}

    \begin{scope}[shift={(0.5cm,-1cm)}]
  % Vertical bar
  \draw (0.5,0) -- (0.5,-2);
  % Horizontal bar
  \draw (0.5,-1) -- (1.0,-1);
  % Dividend
  % Divisor
    \node[anchor=west] at (0.7,-0.5*1) { 2 };
  % Quotient
    \node[anchor=west] at (0.7,-1.5*1) { 6 };
  % Remainder
    \node[anchor=center] at (0.25,-1.5*1) { 0 };
\end{scope}

    \begin{scope}[shift={(1.0cm,-2cm)}]
  % Vertical bar
  \draw (0.5,0) -- (0.5,-2);
  % Horizontal bar
  \draw (0.5,-1) -- (1.0,-1);
  % Dividend
  % Divisor
    \node[anchor=west] at (0.7,-0.5*1) { 2 };
  % Quotient
    \node[anchor=west] at (0.7,-1.5*1) { 3 };
  % Remainder
    \node[anchor=center] at (0.25,-1.5*1) { 0 };
\end{scope}

    \begin{scope}[shift={(1.5cm,-3cm)}]
  % Vertical bar
  \draw (0.5,0) -- (0.5,-2);
  % Horizontal bar
  \draw (0.5,-1) -- (1.0,-1);
  % Dividend
  % Divisor
    \node[anchor=west] at (0.7,-0.5*1) { 2 };
  % Quotient
    \node[anchor=west] at (0.7,-1.5*1) { 1 };
  % Remainder
    \node[anchor=center] at (0.25,-1.5*1) { 1 };
\end{scope}

    \begin{scope}[shift={(2.0cm,-4cm)}]
  % Vertical bar
  \draw (0.5,0) -- (0.5,-2);
  % Horizontal bar
  \draw (0.5,-1) -- (1.0,-1);
  % Dividend
  % Divisor
    \node[anchor=west] at (0.7,-0.5*1) { 2 };
  % Quotient
    \node[anchor=west] at (0.7,-1.5*1) { 0 };
  % Remainder
    \node[anchor=center] at (0.25,-1.5*1) { 1 };
\end{scope}

\end{tikzpicture}


Result (bottom→top remainders): 11000




\paragraph{Q2}

Answer

\paragraph{Q3}

Calculate  the following operations in base 8 :  أنجز العمليات الآتية في الأساس 8

\begin{verbatim}

  5341\,6306
- 2254\,5644
-------------------
=  3065\,0442

\end{verbatim}
\pagebreak

\section{Question}

\paragraph{Q1}

Representer sous la norme IEEE-754 32 bits le nombre suivant
-46.769
\paragraph{Q2}

Simplifier les fonctions suivantes

\begin{karnaugh-map}[4][4][1][cd][ab]
          \minterms{0, 5, 7, 8, 10, 12, 13, 14}
          \maxterms{1, 2, 3, 4, 6, 9, 11, 15}
        %\autoterms[0]
        % simplification
        
          %\implicant{5}{15}
          %\implicantedge{8}{8}{10}{10}
          %\implicantedge{8}{8}{10}{10}[8,10]
        \end{karnaugh-map}\begin{karnaugh-map}[4][4][1][cd][ab]
          \minterms{0, 1, 3, 6, 9, 10, 12, 13, 14, 15}
          \maxterms{2, 4, 5, 7, 8, 11}
        %\autoterms[0]
        % simplification
        
          %\implicant{5}{15}
          %\implicantedge{8}{8}{10}{10}
          %\implicantedge{8}{8}{10}{10}[8,10]
        \end{karnaugh-map}\begin{karnaugh-map}[4][4][1][cd][ab]
          \minterms{1, 3, 4, 5, 7, 8, 11, 12, 13}
          \maxterms{0, 2, 6, 9, 10, 14, 15}
        %\autoterms[0]
        % simplification
        
          %\implicant{5}{15}
          %\implicantedge{8}{8}{10}{10}
          %\implicantedge{8}{8}{10}{10}[8,10]
        \end{karnaugh-map}
\hrule width 1\linewidth
\paragraph{Q1}

Representer sous la norme IEEE-754 32 bits le nombre suivant
-46.769
\paragraph{Q2}

Simplifier les fonctions suivantes

\begin{karnaugh-map}[4][4][1][cd][ab]
          \minterms{0, 5, 7, 8, 10, 12, 13, 14}
          \maxterms{1, 2, 3, 4, 6, 9, 11, 15}
        %\autoterms[0]
        % simplification
        
          %\implicant{5}{15}
          %\implicantedge{8}{8}{10}{10}
          %\implicantedge{8}{8}{10}{10}[8,10]
        \end{karnaugh-map}\begin{karnaugh-map}[4][4][1][cd][ab]
          \minterms{0, 1, 3, 6, 9, 10, 12, 13, 14, 15}
          \maxterms{2, 4, 5, 7, 8, 11}
        %\autoterms[0]
        % simplification
        
          %\implicant{5}{15}
          %\implicantedge{8}{8}{10}{10}
          %\implicantedge{8}{8}{10}{10}[8,10]
        \end{karnaugh-map}\begin{karnaugh-map}[4][4][1][cd][ab]
          \minterms{1, 3, 4, 5, 7, 8, 11, 12, 13}
          \maxterms{0, 2, 6, 9, 10, 14, 15}
        %\autoterms[0]
        % simplification
        
          %\implicant{5}{15}
          %\implicantedge{8}{8}{10}{10}
          %\implicantedge{8}{8}{10}{10}[8,10]
        \end{karnaugh-map}
\hrule width 1\linewidth
\paragraph{Q1}

Representer sous la norme IEEE-754 32 bits le nombre suivant
-46.769
\paragraph{Q2}

Simplifier les fonctions suivantes

\begin{karnaugh-map}[4][4][1][cd][ab]
          \minterms{0, 5, 7, 8, 10, 12, 13, 14}
          \maxterms{1, 2, 3, 4, 6, 9, 11, 15}
        %\autoterms[0]
        % simplification
        
          %\implicant{5}{15}
          %\implicantedge{8}{8}{10}{10}
          %\implicantedge{8}{8}{10}{10}[8,10]
        \end{karnaugh-map}\begin{karnaugh-map}[4][4][1][cd][ab]
          \minterms{0, 1, 3, 6, 9, 10, 12, 13, 14, 15}
          \maxterms{2, 4, 5, 7, 8, 11}
        %\autoterms[0]
        % simplification
        
          %\implicant{5}{15}
          %\implicantedge{8}{8}{10}{10}
          %\implicantedge{8}{8}{10}{10}[8,10]
        \end{karnaugh-map}\begin{karnaugh-map}[4][4][1][cd][ab]
          \minterms{1, 3, 4, 5, 7, 8, 11, 12, 13}
          \maxterms{0, 2, 6, 9, 10, 14, 15}
        %\autoterms[0]
        % simplification
        
          %\implicant{5}{15}
          %\implicantedge{8}{8}{10}{10}
          %\implicantedge{8}{8}{10}{10}[8,10]
        \end{karnaugh-map}
\hrule width 1\linewidth
\paragraph{Q1}

Representer sous la norme IEEE-754 32 bits le nombre suivant
-46.769
\paragraph{Q2}

Simplifier les fonctions suivantes

\begin{karnaugh-map}[4][4][1][cd][ab]
          \minterms{0, 5, 7, 8, 10, 12, 13, 14}
          \maxterms{1, 2, 3, 4, 6, 9, 11, 15}
        %\autoterms[0]
        % simplification
        
          %\implicant{5}{15}
          %\implicantedge{8}{8}{10}{10}
          %\implicantedge{8}{8}{10}{10}[8,10]
        \end{karnaugh-map}\begin{karnaugh-map}[4][4][1][cd][ab]
          \minterms{0, 1, 3, 6, 9, 10, 12, 13, 14, 15}
          \maxterms{2, 4, 5, 7, 8, 11}
        %\autoterms[0]
        % simplification
        
          %\implicant{5}{15}
          %\implicantedge{8}{8}{10}{10}
          %\implicantedge{8}{8}{10}{10}[8,10]
        \end{karnaugh-map}\begin{karnaugh-map}[4][4][1][cd][ab]
          \minterms{1, 3, 4, 5, 7, 8, 11, 12, 13}
          \maxterms{0, 2, 6, 9, 10, 14, 15}
        %\autoterms[0]
        % simplification
        
          %\implicant{5}{15}
          %\implicantedge{8}{8}{10}{10}
          %\implicantedge{8}{8}{10}{10}[8,10]
        \end{karnaugh-map}
\hrule width 1\linewidth
\paragraph{Q1}

Representer sous la norme IEEE-754 32 bits le nombre suivant
-46.769
\paragraph{Q2}

Simplifier les fonctions suivantes

\begin{karnaugh-map}[4][4][1][cd][ab]
          \minterms{0, 5, 7, 8, 10, 12, 13, 14}
          \maxterms{1, 2, 3, 4, 6, 9, 11, 15}
        %\autoterms[0]
        % simplification
        
          %\implicant{5}{15}
          %\implicantedge{8}{8}{10}{10}
          %\implicantedge{8}{8}{10}{10}[8,10]
        \end{karnaugh-map}\begin{karnaugh-map}[4][4][1][cd][ab]
          \minterms{0, 1, 3, 6, 9, 10, 12, 13, 14, 15}
          \maxterms{2, 4, 5, 7, 8, 11}
        %\autoterms[0]
        % simplification
        
          %\implicant{5}{15}
          %\implicantedge{8}{8}{10}{10}
          %\implicantedge{8}{8}{10}{10}[8,10]
        \end{karnaugh-map}\begin{karnaugh-map}[4][4][1][cd][ab]
          \minterms{1, 3, 4, 5, 7, 8, 11, 12, 13}
          \maxterms{0, 2, 6, 9, 10, 14, 15}
        %\autoterms[0]
        % simplification
        
          %\implicant{5}{15}
          %\implicantedge{8}{8}{10}{10}
          %\implicantedge{8}{8}{10}{10}[8,10]
        \end{karnaugh-map}
\hrule width 1\linewidth\pagebreak
\subsection{Correction}

\paragraph{Q1}

Representer sous la norme IEEE-754 32 bits le nombre suivant : -46.769

\begin{verbatim}46.7690 = 101110.111101100110011001100110011001
 Forme normalise 1.01110111101100110011001 x2\^5
Signe + => 0
 exposant 5 +127 = 132 =(10000100)
  pseudo mantisse  01110111101100110011001 
 La representation 
 11000010001110111101100110011001
 Hex C23BD999

\end{verbatim}
\paragraph{Q2}

Simplifier les fonctions suivantes
table 1

\begin{karnaugh-map}[4][4][1][cd][ab]
          \minterms{0, 5, 7, 8, 10, 12, 13, 14}
          \maxterms{1, 2, 3, 4, 6, 9, 11, 15}
        %\autoterms[0]
        % simplification
        \implicantedge{12}{8}{14}{10}
\implicant{5}{7}
\implicant{5}{13}
\implicantedge{0}{0}{8}{8}
          %\implicant{5}{15}
          %\implicantedge{8}{8}{10}{10}
          %\implicantedge{8}{8}{10}{10}[8,10]
        \end{karnaugh-map}Simplified Sum of products : $ a.\bar d + \bar a.b.d + b.\bar c.d + \bar b.\bar c.\bar d $

table 2

\begin{karnaugh-map}[4][4][1][cd][ab]
          \minterms{0, 1, 3, 6, 9, 10, 12, 13, 14, 15}
          \maxterms{2, 4, 5, 7, 8, 11}
        %\autoterms[0]
        % simplification
        \implicant{12}{14}
\implicant{14}{10}
\implicant{6}{14}
\implicant{1}{3}
\implicantedge{1}{1}{9}{9}
\implicant{0}{1}
          %\implicant{5}{15}
          %\implicantedge{8}{8}{10}{10}
          %\implicantedge{8}{8}{10}{10}[8,10]
        \end{karnaugh-map}Simplified Sum of products : $ a.b + a.c.\bar d + b.c.\bar d + \bar a.\bar b.d + \bar b.\bar c.d + \bar a.\bar b.\bar c $

table 3

\begin{karnaugh-map}[4][4][1][cd][ab]
          \minterms{1, 3, 4, 5, 7, 8, 11, 12, 13}
          \maxterms{0, 2, 6, 9, 10, 14, 15}
        %\autoterms[0]
        % simplification
        \implicant{4}{13}
\implicant{1}{7}
\implicantedge{3}{3}{11}{11}
\implicant{12}{8}
          %\implicant{5}{15}
          %\implicantedge{8}{8}{10}{10}
          %\implicantedge{8}{8}{10}{10}[8,10]
        \end{karnaugh-map}Simplified Sum of products : $ b.\bar c + \bar a.d + \bar b.c.d + a.\bar c.\bar d $

\pagebreak
\section{Question}

\paragraph{Q1}

Representer sous la norme IEEE-754 32 bits le nombre suivant
153.143
\paragraph{Q2}

Soit la fonction donnée par sa forme canonique, Tracer la table de karnaugh et simplifier.

F0(a, b, c, d) = $\bar a.b.c.\bar d + \bar a.b.c.d + a.\bar b.\bar c.d + a.\bar b.c.d$

F0(a, b, c, d) = $\varSigma([6, 7, 9, 11])$

F1(a, b, c, d) = $\bar a.\bar b.\bar c.\bar d + \bar a.\bar b.\bar c.d + \bar a.\bar b.c.d + \bar a.b.c.\bar d + a.\bar b.\bar c.\bar d + a.\bar b.c.\bar d + a.b.\bar c.d + a.b.c.\bar d$

F1(a, b, c, d) = $\varSigma([0, 1, 3, 6, 8, 10, 13, 14])$


\hrule width 1\linewidth
\paragraph{Q1}

Representer sous la norme IEEE-754 32 bits le nombre suivant
153.143
\paragraph{Q2}

Soit la fonction donnée par sa forme canonique, Tracer la table de karnaugh et simplifier.

F0(a, b, c, d) = $\bar a.b.c.\bar d + \bar a.b.c.d + a.\bar b.\bar c.d + a.\bar b.c.d$

F0(a, b, c, d) = $\varSigma([6, 7, 9, 11])$

F1(a, b, c, d) = $\bar a.\bar b.\bar c.\bar d + \bar a.\bar b.\bar c.d + \bar a.\bar b.c.d + \bar a.b.c.\bar d + a.\bar b.\bar c.\bar d + a.\bar b.c.\bar d + a.b.\bar c.d + a.b.c.\bar d$

F1(a, b, c, d) = $\varSigma([0, 1, 3, 6, 8, 10, 13, 14])$


\hrule width 1\linewidth
\paragraph{Q1}

Representer sous la norme IEEE-754 32 bits le nombre suivant
153.143
\paragraph{Q2}

Soit la fonction donnée par sa forme canonique, Tracer la table de karnaugh et simplifier.

F0(a, b, c, d) = $\bar a.b.c.\bar d + \bar a.b.c.d + a.\bar b.\bar c.d + a.\bar b.c.d$

F0(a, b, c, d) = $\varSigma([6, 7, 9, 11])$

F1(a, b, c, d) = $\bar a.\bar b.\bar c.\bar d + \bar a.\bar b.\bar c.d + \bar a.\bar b.c.d + \bar a.b.c.\bar d + a.\bar b.\bar c.\bar d + a.\bar b.c.\bar d + a.b.\bar c.d + a.b.c.\bar d$

F1(a, b, c, d) = $\varSigma([0, 1, 3, 6, 8, 10, 13, 14])$


\hrule width 1\linewidth
\paragraph{Q1}

Representer sous la norme IEEE-754 32 bits le nombre suivant
153.143
\paragraph{Q2}

Soit la fonction donnée par sa forme canonique, Tracer la table de karnaugh et simplifier.

F0(a, b, c, d) = $\bar a.b.c.\bar d + \bar a.b.c.d + a.\bar b.\bar c.d + a.\bar b.c.d$

F0(a, b, c, d) = $\varSigma([6, 7, 9, 11])$

F1(a, b, c, d) = $\bar a.\bar b.\bar c.\bar d + \bar a.\bar b.\bar c.d + \bar a.\bar b.c.d + \bar a.b.c.\bar d + a.\bar b.\bar c.\bar d + a.\bar b.c.\bar d + a.b.\bar c.d + a.b.c.\bar d$

F1(a, b, c, d) = $\varSigma([0, 1, 3, 6, 8, 10, 13, 14])$


\hrule width 1\linewidth
\paragraph{Q1}

Representer sous la norme IEEE-754 32 bits le nombre suivant
153.143
\paragraph{Q2}

Soit la fonction donnée par sa forme canonique, Tracer la table de karnaugh et simplifier.

F0(a, b, c, d) = $\bar a.b.c.\bar d + \bar a.b.c.d + a.\bar b.\bar c.d + a.\bar b.c.d$

F0(a, b, c, d) = $\varSigma([6, 7, 9, 11])$

F1(a, b, c, d) = $\bar a.\bar b.\bar c.\bar d + \bar a.\bar b.\bar c.d + \bar a.\bar b.c.d + \bar a.b.c.\bar d + a.\bar b.\bar c.\bar d + a.\bar b.c.\bar d + a.b.\bar c.d + a.b.c.\bar d$

F1(a, b, c, d) = $\varSigma([0, 1, 3, 6, 8, 10, 13, 14])$


\hrule width 1\linewidth\pagebreak
\subsection{Correction}

\paragraph{Q1}

Representer sous la norme IEEE-754 32 bits le nombre suivant : 153.143

\begin{verbatim}153.1430 = 10011001.001001001001101110100101111000
 Forme normalise 1.00110010010010010011011 x2\^7
Signe + => 0
 exposant 7 +127 = 134 =(10000110)
  pseudo mantisse  00110010010010010011011 
 La representation 
 01000011000110010010010010011011
 Hex 4319249B

\end{verbatim}
\paragraph{Q2}

Simplifier les fonctions suivantes

F0(a, b, c, d) = $\bar a.b.c.\bar d + \bar a.b.c.d + a.\bar b.\bar c.d + a.\bar b.c.d$

F0(a, b, c, d) = $\varSigma([6, 7, 9, 11])$

\begin{karnaugh-map}[4][4][1][cd][ab]
          \minterms{6, 7, 9, 11}
          \maxterms{0, 1, 2, 3, 4, 5, 8, 10, 12, 13, 14, 15}
        %\autoterms[0]
        % simplification
        \implicant{9}{11}
\implicant{7}{6}
          %\implicant{5}{15}
          %\implicantedge{8}{8}{10}{10}
          %\implicantedge{8}{8}{10}{10}[8,10]
        \end{karnaugh-map}Simplified Sum of products : $ a.\bar b.d + \bar a.b.c $

F1(a, b, c, d) = $\bar a.\bar b.\bar c.\bar d + \bar a.\bar b.\bar c.d + \bar a.\bar b.c.d + \bar a.b.c.\bar d + a.\bar b.\bar c.\bar d + a.\bar b.c.\bar d + a.b.\bar c.d + a.b.c.\bar d$

F1(a, b, c, d) = $\varSigma([0, 1, 3, 6, 8, 10, 13, 14])$

\begin{karnaugh-map}[4][4][1][cd][ab]
          \minterms{0, 1, 3, 6, 8, 10, 13, 14}
          \maxterms{2, 4, 5, 7, 9, 11, 12, 15}
        %\autoterms[0]
        % simplification
        \implicant{6}{14}
\implicantedge{8}{8}{10}{10}
\implicant{1}{3}
\implicant{13}{13}
\implicantedge{0}{0}{8}{8}
          %\implicant{5}{15}
          %\implicantedge{8}{8}{10}{10}
          %\implicantedge{8}{8}{10}{10}[8,10]
        \end{karnaugh-map}Simplified Sum of products : $ b.c.\bar d + a.\bar b.\bar d + \bar a.\bar b.d + a.b.\bar c.d + \bar b.\bar c.\bar d $

\pagebreak
\section{Question}

\paragraph{Q1}

Representer sous la norme IEEE-754 32 bits le nombre suivant
223.405
\paragraph{Q2}

Etudier la fonction suivante

f(a,b,c,d)= $ c.\bar d + a.\bar b.\bar d + \bar a.b.\bar d + \bar a.\bar b.d  +  a.\bar b.c + a.\bar b.d + \bar a.b.c.d + a.b.\bar c.\bar d + \bar a.\bar b.\bar c.\bar d $

\hrule width 1\linewidth
\paragraph{Q1}

Representer sous la norme IEEE-754 32 bits le nombre suivant
223.405
\paragraph{Q2}

Etudier la fonction suivante

f(a,b,c,d)= $ c.\bar d + a.\bar b.\bar d + \bar a.b.\bar d + \bar a.\bar b.d  +  a.\bar b.c + a.\bar b.d + \bar a.b.c.d + a.b.\bar c.\bar d + \bar a.\bar b.\bar c.\bar d $

\hrule width 1\linewidth
\paragraph{Q1}

Representer sous la norme IEEE-754 32 bits le nombre suivant
223.405
\paragraph{Q2}

Etudier la fonction suivante

f(a,b,c,d)= $ c.\bar d + a.\bar b.\bar d + \bar a.b.\bar d + \bar a.\bar b.d  +  a.\bar b.c + a.\bar b.d + \bar a.b.c.d + a.b.\bar c.\bar d + \bar a.\bar b.\bar c.\bar d $

\hrule width 1\linewidth
\paragraph{Q1}

Representer sous la norme IEEE-754 32 bits le nombre suivant
223.405
\paragraph{Q2}

Etudier la fonction suivante

f(a,b,c,d)= $ c.\bar d + a.\bar b.\bar d + \bar a.b.\bar d + \bar a.\bar b.d  +  a.\bar b.c + a.\bar b.d + \bar a.b.c.d + a.b.\bar c.\bar d + \bar a.\bar b.\bar c.\bar d $

\hrule width 1\linewidth
\paragraph{Q1}

Representer sous la norme IEEE-754 32 bits le nombre suivant
223.405
\paragraph{Q2}

Etudier la fonction suivante

f(a,b,c,d)= $ c.\bar d + a.\bar b.\bar d + \bar a.b.\bar d + \bar a.\bar b.d  +  a.\bar b.c + a.\bar b.d + \bar a.b.c.d + a.b.\bar c.\bar d + \bar a.\bar b.\bar c.\bar d $

\hrule width 1\linewidth\pagebreak
\subsection{Correction}

\paragraph{Q1}

Representer sous la norme IEEE-754 32 bits le nombre suivant : 223.405

\begin{verbatim}223.4050 = 11011111.011001111010111011001100110011
 Forme normalise 1.10111110110011110101110 x2\^7
Signe + => 0
 exposant 7 +127 = 134 =(10000110)
  pseudo mantisse  10111110110011110101110 
 La representation 
 01000011010111110110011110101110
 Hex 435F67AE

\end{verbatim}
\paragraph{Q2}

f(a,b,c,d)=$ c.\bar d + a.\bar b.\bar d + \bar a.b.\bar d + \bar a.\bar b.d  +  a.\bar b.c + a.\bar b.d + \bar a.b.c.d + a.b.\bar c.\bar d + \bar a.\bar b.\bar c.\bar d $
f(a,b,c,D)=$ \sum c.\bar d + a.\bar b.\bar d + \bar a.b.\bar d + \bar a.\bar b.d  +  a.\bar b.c + a.\bar b.d + \bar a.b.c.d + a.b.\bar c.\bar d + \bar a.\bar b.\bar c.\bar d $ 

%%\begin{table}
        \begin{tabular}{|c|c|c|c|c||c|}
    \toprule
         & A & B & C & D & F\\ \midrule0 & 0 & 0 & 0 & 0 & 1\\1 & 0 & 0 & 0 & 1 & 1\\2 & 0 & 0 & 1 & 0 & 1\\3 & 0 & 0 & 1 & 1 & 1\\\midrule4 & 0 & 1 & 0 & 0 & 1\\5 & 0 & 1 & 0 & 1 & 0\\6 & 0 & 1 & 1 & 0 & 1\\7 & 0 & 1 & 1 & 1 & 1\\\midrule8 & 1 & 0 & 0 & 0 & 1\\9 & 1 & 0 & 0 & 1 & 1\\10 & 1 & 0 & 1 & 0 & 1\\11 & 1 & 0 & 1 & 1 & 1\\\midrule12 & 1 & 1 & 0 & 0 & 1\\13 & 1 & 1 & 0 & 1 & 0\\14 & 1 & 1 & 1 & 0 & 1\\15 & 1 & 1 & 1 & 1 & 0\\\bottomrule
        \end{tabular}
        %%\end{table}
        
Sum of products 
 f(a,b,c,d) = $\bar a.\bar b.\bar c.\bar d + \bar a.\bar b.\bar c.d + \bar a.\bar b.c.\bar d + \bar a.\bar b.c.d + \bar a.b.\bar c.\bar d + \bar a.b.c.\bar d + \bar a.b.c.d + a.\bar b.\bar c.\bar d + a.\bar b.\bar c.d + a.\bar b.c.\bar d + a.\bar b.c.d + a.b.\bar c.\bar d + a.b.c.\bar d$

Product of sums 
 f(a,b,c,d) = $(a+\bar b+c+\bar d) . (\bar a+\bar b+c+\bar d) . (\bar a+\bar b+\bar c+\bar d)$

Karnough map
\begin{karnaugh-map}[4][4][1][cd][ab]
          \minterms{0, 1, 2, 3, 4, 6, 7, 8, 9, 10, 11, 12, 14}
          \maxterms{5, 13, 15}
        %\autoterms[0]
        % simplification
        \implicantedge{0}{2}{8}{10}
\implicantedge{0}{8}{2}{10}
\implicant{3}{6}
          %\implicant{5}{15}
          %\implicantedge{8}{8}{10}{10}
          %\implicantedge{8}{8}{10}{10}[8,10]
        \end{karnaugh-map}

Simplified Sum of products: $ \bar b + \bar d + \bar a.c $

Simplified Product of sums: $(\bar b+c+\bar d).(\bar a+\bar b+\bar d)$
\paragraph{Logigramme} de la fonction\\
        %%\missingfigure[figwidth=6cm]{Logigramme}

 \begin{tikzpicture}

\node (x) at (0, 3*1.5) {$A$};
            \node (y) at (0.5, 3*1.5) {$B$};
            \node (z) at (1, 3*1.5) {$C$};
            \node (w) at (1.5, 3*1.5) {$D$};
            \node[not gate US, draw, rotate=270] at ($(x) + (0.25, -0.4)$) (notx) {};
            \draw (x) -- (notx.input); 
            \node[not gate US, draw, rotate=270] at ($(y) + (0.25, -0.4)$) (noty) {};
            \draw (y) -- (noty.input); 
            \node[not gate US, draw, rotate=270] at ($(z) + (0.25, -0.4)$) (notz) {};
            \draw (z) -- (notz.input);
            \node[not gate US, draw, rotate=270] at ($(w) + (0.25, -0.4)$) (notw) {};
            \draw (w) -- (notw.input);
                
           
            \node[and gate US, draw, rotate=0, logic gate inputs=nnnn] at (2.5, 0*1.5) (xandy0) {};
            \draw (xandy0.output) -- node[above]{\scriptsize $ \bar B $} ($(xandy0) + (1.8, 0)$);
            
            %Y'

            \draw [line width=0.25mm,   red] (noty.output) -- ([xshift=0cm]noty.output) |- (xandy0.input 2);
                  
           
            \node[and gate US, draw, rotate=0, logic gate inputs=nnnn] at (2.5, 1*1.5) (xandy1) {};
            \draw (xandy1.output) -- node[above]{\scriptsize $ \bar D $} ($(xandy1) + (1.8, 0)$);
            
            %%W

            \draw [line width=0.25mm,   red] (notw.output) -- ([xshift=0cm]notw.output) |- (xandy1.input 4);
                  
           
            \node[and gate US, draw, rotate=0, logic gate inputs=nnnn] at (2.5, 2*1.5) (xandy2) {};
            \draw (xandy2.output) -- node[above]{\scriptsize $ \bar A.C $} ($(xandy2) + (1.8, 0)$);
            
            %% X'

            \draw  [line width=0.25mm,   red] (notx.output) -- ([xshift=0cm]notx.output) |- (xandy2.input 1);
            
            %%Z
            \draw (z) -| ($(z) + (0, 0)$) |- (xandy2.input 3);
        \node[or gate US, draw, rotate=0, logic gate inputs=nnnn] at (5.5, 3*0.6) (xory) {};


                    \draw (xory.output) -- node[above]{\scriptsize$F$} ($(xory) + (1, 0)$);

\draw (xandy0.output) -- ([xshift=1.40cm]xandy0.output) |- (xory.input 3);

\draw (xandy1.output) -- ([xshift=1.35cm]xandy1.output) |- (xory.input 2);

\draw (xandy2.output) -- ([xshift=1.30cm]xandy2.output) |- (xory.input 1);

 \end{tikzpicture}

\pagebreak
\section{Question}

\paragraph{Q1}

Representer en complément à 1 et à 2 le nombre  : -31

\paragraph{Q2}

Representer en complément à 1 et à 2 le nombre  : -77

\paragraph{Q3}

Simplifier les fonctions suivantes

\begin{karnaugh-map}[4][4][1][cd][ab]
          \minterms{1, 4, 5, 7, 8, 10, 13}
          \maxterms{0, 2, 3, 6, 9, 11, 12, 14, 15}
        %\autoterms[0]
        % simplification
        
          %\implicant{5}{15}
          %\implicantedge{8}{8}{10}{10}
          %\implicantedge{8}{8}{10}{10}[8,10]
        \end{karnaugh-map}\begin{karnaugh-map}[4][4][1][cd][ab]
          \minterms{1, 2, 3, 4, 5, 6, 12, 15}
          \maxterms{0, 7, 8, 9, 10, 11, 13, 14}
        %\autoterms[0]
        % simplification
        
          %\implicant{5}{15}
          %\implicantedge{8}{8}{10}{10}
          %\implicantedge{8}{8}{10}{10}[8,10]
        \end{karnaugh-map}\begin{karnaugh-map}[4][4][1][cd][ab]
          \minterms{1, 2, 9, 11, 15}
          \maxterms{0, 3, 4, 5, 6, 7, 8, 10, 12, 13, 14}
        %\autoterms[0]
        % simplification
        
          %\implicant{5}{15}
          %\implicantedge{8}{8}{10}{10}
          %\implicantedge{8}{8}{10}{10}[8,10]
        \end{karnaugh-map}
\hrule width 1\linewidth
\paragraph{Q1}

Representer en complément à 1 et à 2 le nombre  : -31

\paragraph{Q2}

Representer en complément à 1 et à 2 le nombre  : -77

\paragraph{Q3}

Simplifier les fonctions suivantes

\begin{karnaugh-map}[4][4][1][cd][ab]
          \minterms{1, 4, 5, 7, 8, 10, 13}
          \maxterms{0, 2, 3, 6, 9, 11, 12, 14, 15}
        %\autoterms[0]
        % simplification
        
          %\implicant{5}{15}
          %\implicantedge{8}{8}{10}{10}
          %\implicantedge{8}{8}{10}{10}[8,10]
        \end{karnaugh-map}\begin{karnaugh-map}[4][4][1][cd][ab]
          \minterms{1, 2, 3, 4, 5, 6, 12, 15}
          \maxterms{0, 7, 8, 9, 10, 11, 13, 14}
        %\autoterms[0]
        % simplification
        
          %\implicant{5}{15}
          %\implicantedge{8}{8}{10}{10}
          %\implicantedge{8}{8}{10}{10}[8,10]
        \end{karnaugh-map}\begin{karnaugh-map}[4][4][1][cd][ab]
          \minterms{1, 2, 9, 11, 15}
          \maxterms{0, 3, 4, 5, 6, 7, 8, 10, 12, 13, 14}
        %\autoterms[0]
        % simplification
        
          %\implicant{5}{15}
          %\implicantedge{8}{8}{10}{10}
          %\implicantedge{8}{8}{10}{10}[8,10]
        \end{karnaugh-map}
\hrule width 1\linewidth
\paragraph{Q1}

Representer en complément à 1 et à 2 le nombre  : -31

\paragraph{Q2}

Representer en complément à 1 et à 2 le nombre  : -77

\paragraph{Q3}

Simplifier les fonctions suivantes

\begin{karnaugh-map}[4][4][1][cd][ab]
          \minterms{1, 4, 5, 7, 8, 10, 13}
          \maxterms{0, 2, 3, 6, 9, 11, 12, 14, 15}
        %\autoterms[0]
        % simplification
        
          %\implicant{5}{15}
          %\implicantedge{8}{8}{10}{10}
          %\implicantedge{8}{8}{10}{10}[8,10]
        \end{karnaugh-map}\begin{karnaugh-map}[4][4][1][cd][ab]
          \minterms{1, 2, 3, 4, 5, 6, 12, 15}
          \maxterms{0, 7, 8, 9, 10, 11, 13, 14}
        %\autoterms[0]
        % simplification
        
          %\implicant{5}{15}
          %\implicantedge{8}{8}{10}{10}
          %\implicantedge{8}{8}{10}{10}[8,10]
        \end{karnaugh-map}\begin{karnaugh-map}[4][4][1][cd][ab]
          \minterms{1, 2, 9, 11, 15}
          \maxterms{0, 3, 4, 5, 6, 7, 8, 10, 12, 13, 14}
        %\autoterms[0]
        % simplification
        
          %\implicant{5}{15}
          %\implicantedge{8}{8}{10}{10}
          %\implicantedge{8}{8}{10}{10}[8,10]
        \end{karnaugh-map}
\hrule width 1\linewidth
\paragraph{Q1}

Representer en complément à 1 et à 2 le nombre  : -31

\paragraph{Q2}

Representer en complément à 1 et à 2 le nombre  : -77

\paragraph{Q3}

Simplifier les fonctions suivantes

\begin{karnaugh-map}[4][4][1][cd][ab]
          \minterms{1, 4, 5, 7, 8, 10, 13}
          \maxterms{0, 2, 3, 6, 9, 11, 12, 14, 15}
        %\autoterms[0]
        % simplification
        
          %\implicant{5}{15}
          %\implicantedge{8}{8}{10}{10}
          %\implicantedge{8}{8}{10}{10}[8,10]
        \end{karnaugh-map}\begin{karnaugh-map}[4][4][1][cd][ab]
          \minterms{1, 2, 3, 4, 5, 6, 12, 15}
          \maxterms{0, 7, 8, 9, 10, 11, 13, 14}
        %\autoterms[0]
        % simplification
        
          %\implicant{5}{15}
          %\implicantedge{8}{8}{10}{10}
          %\implicantedge{8}{8}{10}{10}[8,10]
        \end{karnaugh-map}\begin{karnaugh-map}[4][4][1][cd][ab]
          \minterms{1, 2, 9, 11, 15}
          \maxterms{0, 3, 4, 5, 6, 7, 8, 10, 12, 13, 14}
        %\autoterms[0]
        % simplification
        
          %\implicant{5}{15}
          %\implicantedge{8}{8}{10}{10}
          %\implicantedge{8}{8}{10}{10}[8,10]
        \end{karnaugh-map}
\hrule width 1\linewidth
\paragraph{Q1}

Representer en complément à 1 et à 2 le nombre  : -31

\paragraph{Q2}

Representer en complément à 1 et à 2 le nombre  : -77

\paragraph{Q3}

Simplifier les fonctions suivantes

\begin{karnaugh-map}[4][4][1][cd][ab]
          \minterms{1, 4, 5, 7, 8, 10, 13}
          \maxterms{0, 2, 3, 6, 9, 11, 12, 14, 15}
        %\autoterms[0]
        % simplification
        
          %\implicant{5}{15}
          %\implicantedge{8}{8}{10}{10}
          %\implicantedge{8}{8}{10}{10}[8,10]
        \end{karnaugh-map}\begin{karnaugh-map}[4][4][1][cd][ab]
          \minterms{1, 2, 3, 4, 5, 6, 12, 15}
          \maxterms{0, 7, 8, 9, 10, 11, 13, 14}
        %\autoterms[0]
        % simplification
        
          %\implicant{5}{15}
          %\implicantedge{8}{8}{10}{10}
          %\implicantedge{8}{8}{10}{10}[8,10]
        \end{karnaugh-map}\begin{karnaugh-map}[4][4][1][cd][ab]
          \minterms{1, 2, 9, 11, 15}
          \maxterms{0, 3, 4, 5, 6, 7, 8, 10, 12, 13, 14}
        %\autoterms[0]
        % simplification
        
          %\implicant{5}{15}
          %\implicantedge{8}{8}{10}{10}
          %\implicantedge{8}{8}{10}{10}[8,10]
        \end{karnaugh-map}
\hrule width 1\linewidth\pagebreak
\subsection{Correction}

\paragraph{Q1}

Representer en complément à 1 et à 2 le nombre  : -31

$-31$

$   -11111_{2}$

$ 11100000_{cp1}$

$ 11100001_{cp2}$
\paragraph{Q2}

Representer en complément à 1 et à 2 le nombre  : -77

$-77$

$ -1001101_{2}$

$ 10110010_{cp1}$

$ 10110011_{cp2}$
\paragraph{Q3}

Simplifier les fonctions suivantes
table 1

\begin{karnaugh-map}[4][4][1][cd][ab]
          \minterms{1, 4, 5, 7, 8, 10, 13}
          \maxterms{0, 2, 3, 6, 9, 11, 12, 14, 15}
        %\autoterms[0]
        % simplification
        \implicant{5}{7}
\implicant{5}{13}
\implicantedge{8}{8}{10}{10}
\implicant{4}{5}
\implicant{1}{5}
          %\implicant{5}{15}
          %\implicantedge{8}{8}{10}{10}
          %\implicantedge{8}{8}{10}{10}[8,10]
        \end{karnaugh-map}Simplified Sum of products : $ \bar a.b.d + b.\bar c.d + a.\bar b.\bar d + \bar a.b.\bar c + \bar a.\bar c.d $

table 2

\begin{karnaugh-map}[4][4][1][cd][ab]
          \minterms{1, 2, 3, 4, 5, 6, 12, 15}
          \maxterms{0, 7, 8, 9, 10, 11, 13, 14}
        %\autoterms[0]
        % simplification
        \implicant{15}{15}
\implicant{4}{12}
\implicant{2}{6}
\implicant{1}{3}
\implicant{1}{5}
          %\implicant{5}{15}
          %\implicantedge{8}{8}{10}{10}
          %\implicantedge{8}{8}{10}{10}[8,10]
        \end{karnaugh-map}Simplified Sum of products : $ a.b.c.d + b.\bar c.\bar d + \bar a.c.\bar d + \bar a.\bar b.d + \bar a.\bar c.d $

table 3

\begin{karnaugh-map}[4][4][1][cd][ab]
          \minterms{1, 2, 9, 11, 15}
          \maxterms{0, 3, 4, 5, 6, 7, 8, 10, 12, 13, 14}
        %\autoterms[0]
        % simplification
        \implicant{15}{11}
\implicantedge{1}{1}{9}{9}
\implicant{2}{2}
          %\implicant{5}{15}
          %\implicantedge{8}{8}{10}{10}
          %\implicantedge{8}{8}{10}{10}[8,10]
        \end{karnaugh-map}Simplified Sum of products : $ a.c.d + \bar b.\bar c.d + \bar a.\bar b.c.\bar d $

\pagebreak
\section{Question}

\paragraph{Q1}

Etudier la fonction suivante

f(a,b,c,d)= $ a.b.c.d + a.b.\bar c.\bar d + \bar a.\bar b.c.\bar d  +  a.d + a.\bar c + \bar a.c.\bar d $

\paragraph{Q2}

Simplifier l'expression suivante par les proprietés algébreiques 

S = $ a.b.c.d + a.\bar b.\bar c  +  a.\bar d + b.\bar d + a.b.c + \bar b.\bar c  +  a.\bar c.d + \bar b.\bar c.\bar d  +  \bar a.b.c.d + a.b.\bar c.\bar d + a.\bar b.c.\bar d $

\hrule width 1\linewidth
\paragraph{Q1}

Etudier la fonction suivante

f(a,b,c,d)= $ a.b.c.d + a.b.\bar c.\bar d + \bar a.\bar b.c.\bar d  +  a.d + a.\bar c + \bar a.c.\bar d $

\paragraph{Q2}

Simplifier l'expression suivante par les proprietés algébreiques 

S = $ a.b.c.d + a.\bar b.\bar c  +  a.\bar d + b.\bar d + a.b.c + \bar b.\bar c  +  a.\bar c.d + \bar b.\bar c.\bar d  +  \bar a.b.c.d + a.b.\bar c.\bar d + a.\bar b.c.\bar d $

\hrule width 1\linewidth
\paragraph{Q1}

Etudier la fonction suivante

f(a,b,c,d)= $ a.b.c.d + a.b.\bar c.\bar d + \bar a.\bar b.c.\bar d  +  a.d + a.\bar c + \bar a.c.\bar d $

\paragraph{Q2}

Simplifier l'expression suivante par les proprietés algébreiques 

S = $ a.b.c.d + a.\bar b.\bar c  +  a.\bar d + b.\bar d + a.b.c + \bar b.\bar c  +  a.\bar c.d + \bar b.\bar c.\bar d  +  \bar a.b.c.d + a.b.\bar c.\bar d + a.\bar b.c.\bar d $

\hrule width 1\linewidth
\paragraph{Q1}

Etudier la fonction suivante

f(a,b,c,d)= $ a.b.c.d + a.b.\bar c.\bar d + \bar a.\bar b.c.\bar d  +  a.d + a.\bar c + \bar a.c.\bar d $

\paragraph{Q2}

Simplifier l'expression suivante par les proprietés algébreiques 

S = $ a.b.c.d + a.\bar b.\bar c  +  a.\bar d + b.\bar d + a.b.c + \bar b.\bar c  +  a.\bar c.d + \bar b.\bar c.\bar d  +  \bar a.b.c.d + a.b.\bar c.\bar d + a.\bar b.c.\bar d $

\hrule width 1\linewidth
\paragraph{Q1}

Etudier la fonction suivante

f(a,b,c,d)= $ a.b.c.d + a.b.\bar c.\bar d + \bar a.\bar b.c.\bar d  +  a.d + a.\bar c + \bar a.c.\bar d $

\paragraph{Q2}

Simplifier l'expression suivante par les proprietés algébreiques 

S = $ a.b.c.d + a.\bar b.\bar c  +  a.\bar d + b.\bar d + a.b.c + \bar b.\bar c  +  a.\bar c.d + \bar b.\bar c.\bar d  +  \bar a.b.c.d + a.b.\bar c.\bar d + a.\bar b.c.\bar d $

\hrule width 1\linewidth\pagebreak
\subsection{Correction}

\paragraph{Q1}

f(a,b,c,d)=$ a.b.c.d + a.b.\bar c.\bar d + \bar a.\bar b.c.\bar d  +  a.d + a.\bar c + \bar a.c.\bar d $
f(a,b,c,D)=$ \sum a.b.c.d + a.b.\bar c.\bar d + \bar a.\bar b.c.\bar d  +  a.d + a.\bar c + \bar a.c.\bar d $ 

%%\begin{table}
        \begin{tabular}{|c|c|c|c|c||c|}
    \toprule
         & A & B & C & D & F\\ \midrule0 & 0 & 0 & 0 & 0 & 0\\1 & 0 & 0 & 0 & 1 & 0\\2 & 0 & 0 & 1 & 0 & 1\\3 & 0 & 0 & 1 & 1 & 0\\\midrule4 & 0 & 1 & 0 & 0 & 0\\5 & 0 & 1 & 0 & 1 & 0\\6 & 0 & 1 & 1 & 0 & 1\\7 & 0 & 1 & 1 & 1 & 0\\\midrule8 & 1 & 0 & 0 & 0 & 1\\9 & 1 & 0 & 0 & 1 & 1\\10 & 1 & 0 & 1 & 0 & 0\\11 & 1 & 0 & 1 & 1 & 1\\\midrule12 & 1 & 1 & 0 & 0 & 1\\13 & 1 & 1 & 0 & 1 & 1\\14 & 1 & 1 & 1 & 0 & 0\\15 & 1 & 1 & 1 & 1 & 1\\\bottomrule
        \end{tabular}
        %%\end{table}
        
Sum of products 
 f(a,b,c,d) = $\bar a.\bar b.c.\bar d + \bar a.b.c.\bar d + a.\bar b.\bar c.\bar d + a.\bar b.\bar c.d + a.\bar b.c.d + a.b.\bar c.\bar d + a.b.\bar c.d + a.b.c.d$

Product of sums 
 f(a,b,c,d) = $(a+b+c+d) . (a+b+c+\bar d) . (a+b+\bar c+\bar d) . (a+\bar b+c+d) . (a+\bar b+c+\bar d) . (a+\bar b+\bar c+\bar d) . (\bar a+b+\bar c+d) . (\bar a+\bar b+\bar c+d)$

Karnough map
\begin{karnaugh-map}[4][4][1][cd][ab]
          \minterms{2, 6, 8, 9, 11, 12, 13, 15}
          \maxterms{0, 1, 3, 4, 5, 7, 10, 14}
        %\autoterms[0]
        % simplification
        \implicant{13}{11}
\implicant{12}{9}
\implicant{2}{6}
          %\implicant{5}{15}
          %\implicantedge{8}{8}{10}{10}
          %\implicantedge{8}{8}{10}{10}[8,10]
        \end{karnaugh-map}

Simplified Sum of products: $ a.d + a.\bar c + \bar a.c.\bar d $

Simplified Product of sums: $(a+c).(a+\bar d).(\bar a+\bar c+d)$
\paragraph{Logigramme} de la fonction\\
        %%\missingfigure[figwidth=6cm]{Logigramme}

 \begin{tikzpicture}

\node (x) at (0, 3*1.5) {$A$};
            \node (y) at (0.5, 3*1.5) {$B$};
            \node (z) at (1, 3*1.5) {$C$};
            \node (w) at (1.5, 3*1.5) {$D$};
            \node[not gate US, draw, rotate=270] at ($(x) + (0.25, -0.4)$) (notx) {};
            \draw (x) -- (notx.input); 
            \node[not gate US, draw, rotate=270] at ($(y) + (0.25, -0.4)$) (noty) {};
            \draw (y) -- (noty.input); 
            \node[not gate US, draw, rotate=270] at ($(z) + (0.25, -0.4)$) (notz) {};
            \draw (z) -- (notz.input);
            \node[not gate US, draw, rotate=270] at ($(w) + (0.25, -0.4)$) (notw) {};
            \draw (w) -- (notw.input);
                
           
            \node[and gate US, draw, rotate=0, logic gate inputs=nnnn] at (2.5, 0*1.5) (xandy0) {};
            \draw (xandy0.output) -- node[above]{\scriptsize $ A.D $} ($(xandy0) + (1.8, 0)$);
            %% X
            \draw (x) -| ($(x) + (0, 0)$) |- (xandy0.input 1);
                %%W
            \draw (w) -| ($(w) + (0, 0)$) |- (xandy0.input 4);
                
           
            \node[and gate US, draw, rotate=0, logic gate inputs=nnnn] at (2.5, 1*1.5) (xandy1) {};
            \draw (xandy1.output) -- node[above]{\scriptsize $ A.\bar C $} ($(xandy1) + (1.8, 0)$);
            %% X
            \draw (x) -| ($(x) + (0, 0)$) |- (xandy1.input 1);
            
            %%Z'

            \draw [line width=0.25mm,   red] (notz.output) -- ([xshift=0cm]notz.output) |- (xandy1.input 3);
                  
           
            \node[and gate US, draw, rotate=0, logic gate inputs=nnnn] at (2.5, 2*1.5) (xandy2) {};
            \draw (xandy2.output) -- node[above]{\scriptsize $ \bar A.C.\bar D $} ($(xandy2) + (1.8, 0)$);
            
            %% X'

            \draw  [line width=0.25mm,   red] (notx.output) -- ([xshift=0cm]notx.output) |- (xandy2.input 1);
            
            %%Z
            \draw (z) -| ($(z) + (0, 0)$) |- (xandy2.input 3);
        
            %%W

            \draw [line width=0.25mm,   red] (notw.output) -- ([xshift=0cm]notw.output) |- (xandy2.input 4);
          \node[or gate US, draw, rotate=0, logic gate inputs=nnnn] at (5.5, 3*0.6) (xory) {};


                    \draw (xory.output) -- node[above]{\scriptsize$F$} ($(xory) + (1, 0)$);

\draw (xandy0.output) -- ([xshift=1.40cm]xandy0.output) |- (xory.input 3);

\draw (xandy1.output) -- ([xshift=1.35cm]xandy1.output) |- (xory.input 2);

\draw (xandy2.output) -- ([xshift=1.30cm]xandy2.output) |- (xory.input 1);

 \end{tikzpicture}


\paragraph{Q2}

Simplifier l'expression suivante

S = $ a.b.c.d + a.\bar b.\bar c  +  a.\bar d + b.\bar d + a.b.c + \bar b.\bar c  +  a.\bar c.d + \bar b.\bar c.\bar d  +  \bar a.b.c.d + a.b.\bar c.\bar d + a.\bar b.c.\bar d $

 = $ b.c + a.\bar c + a.\bar d + \bar b.\bar c + \bar c.\bar d $


Karnough map
\begin{karnaugh-map}[4][4][1][cd][ab]
          \minterms{0, 1, 4, 6, 7, 8, 9, 10, 12, 13, 14, 15}
          \maxterms{2, 3, 5, 11}
        %\autoterms[0]
        % simplification
        \implicant{7}{14}
\implicant{12}{9}
\implicantedge{12}{8}{14}{10}
\implicantedge{0}{1}{8}{9}
\implicant{0}{8}
          %\implicant{5}{15}
          %\implicantedge{8}{8}{10}{10}
          %\implicantedge{8}{8}{10}{10}[8,10]
        \end{karnaugh-map}

\pagebreak
\section{Question}

\paragraph{Q1}

Etudier la fonction suivante

f(a,b,c,d)= $ \bar a.\bar b.d + \bar b.\bar c.d + a.\bar b.c.\bar d + \bar a.b.c.\bar d  +  b.c.\bar d + \bar a.b.\bar c.d + a.\bar b.\bar c.\bar d $

\paragraph{Q2}

Simplifier l'expression suivante par les proprietés algébreiques 

S = $ b.c + b.d + \bar a.b + a.\bar c.d + a.\bar b.\bar c + \bar a.\bar c.\bar d + \bar b.\bar c.\bar d  +  b.\bar c + a.b.\bar d  +  a.b.d + \bar a.c.d + \bar a.\bar b.c  +  \bar a.c + b.c.d + a.\bar b.\bar c $

\hrule width 1\linewidth
\paragraph{Q1}

Etudier la fonction suivante

f(a,b,c,d)= $ \bar a.\bar b.d + \bar b.\bar c.d + a.\bar b.c.\bar d + \bar a.b.c.\bar d  +  b.c.\bar d + \bar a.b.\bar c.d + a.\bar b.\bar c.\bar d $

\paragraph{Q2}

Simplifier l'expression suivante par les proprietés algébreiques 

S = $ b.c + b.d + \bar a.b + a.\bar c.d + a.\bar b.\bar c + \bar a.\bar c.\bar d + \bar b.\bar c.\bar d  +  b.\bar c + a.b.\bar d  +  a.b.d + \bar a.c.d + \bar a.\bar b.c  +  \bar a.c + b.c.d + a.\bar b.\bar c $

\hrule width 1\linewidth
\paragraph{Q1}

Etudier la fonction suivante

f(a,b,c,d)= $ \bar a.\bar b.d + \bar b.\bar c.d + a.\bar b.c.\bar d + \bar a.b.c.\bar d  +  b.c.\bar d + \bar a.b.\bar c.d + a.\bar b.\bar c.\bar d $

\paragraph{Q2}

Simplifier l'expression suivante par les proprietés algébreiques 

S = $ b.c + b.d + \bar a.b + a.\bar c.d + a.\bar b.\bar c + \bar a.\bar c.\bar d + \bar b.\bar c.\bar d  +  b.\bar c + a.b.\bar d  +  a.b.d + \bar a.c.d + \bar a.\bar b.c  +  \bar a.c + b.c.d + a.\bar b.\bar c $

\hrule width 1\linewidth
\paragraph{Q1}

Etudier la fonction suivante

f(a,b,c,d)= $ \bar a.\bar b.d + \bar b.\bar c.d + a.\bar b.c.\bar d + \bar a.b.c.\bar d  +  b.c.\bar d + \bar a.b.\bar c.d + a.\bar b.\bar c.\bar d $

\paragraph{Q2}

Simplifier l'expression suivante par les proprietés algébreiques 

S = $ b.c + b.d + \bar a.b + a.\bar c.d + a.\bar b.\bar c + \bar a.\bar c.\bar d + \bar b.\bar c.\bar d  +  b.\bar c + a.b.\bar d  +  a.b.d + \bar a.c.d + \bar a.\bar b.c  +  \bar a.c + b.c.d + a.\bar b.\bar c $

\hrule width 1\linewidth
\paragraph{Q1}

Etudier la fonction suivante

f(a,b,c,d)= $ \bar a.\bar b.d + \bar b.\bar c.d + a.\bar b.c.\bar d + \bar a.b.c.\bar d  +  b.c.\bar d + \bar a.b.\bar c.d + a.\bar b.\bar c.\bar d $

\paragraph{Q2}

Simplifier l'expression suivante par les proprietés algébreiques 

S = $ b.c + b.d + \bar a.b + a.\bar c.d + a.\bar b.\bar c + \bar a.\bar c.\bar d + \bar b.\bar c.\bar d  +  b.\bar c + a.b.\bar d  +  a.b.d + \bar a.c.d + \bar a.\bar b.c  +  \bar a.c + b.c.d + a.\bar b.\bar c $

\hrule width 1\linewidth\pagebreak
\subsection{Correction}

\paragraph{Q1}

f(a,b,c,d)=$ \bar a.\bar b.d + \bar b.\bar c.d + a.\bar b.c.\bar d + \bar a.b.c.\bar d  +  b.c.\bar d + \bar a.b.\bar c.d + a.\bar b.\bar c.\bar d $
f(a,b,c,D)=$ \sum \bar a.\bar b.d + \bar b.\bar c.d + a.\bar b.c.\bar d + \bar a.b.c.\bar d  +  b.c.\bar d + \bar a.b.\bar c.d + a.\bar b.\bar c.\bar d $ 

%%\begin{table}
        \begin{tabular}{|c|c|c|c|c||c|}
    \toprule
         & A & B & C & D & F\\ \midrule0 & 0 & 0 & 0 & 0 & 0\\1 & 0 & 0 & 0 & 1 & 1\\2 & 0 & 0 & 1 & 0 & 0\\3 & 0 & 0 & 1 & 1 & 1\\\midrule4 & 0 & 1 & 0 & 0 & 0\\5 & 0 & 1 & 0 & 1 & 1\\6 & 0 & 1 & 1 & 0 & 1\\7 & 0 & 1 & 1 & 1 & 0\\\midrule8 & 1 & 0 & 0 & 0 & 1\\9 & 1 & 0 & 0 & 1 & 1\\10 & 1 & 0 & 1 & 0 & 1\\11 & 1 & 0 & 1 & 1 & 0\\\midrule12 & 1 & 1 & 0 & 0 & 0\\13 & 1 & 1 & 0 & 1 & 0\\14 & 1 & 1 & 1 & 0 & 1\\15 & 1 & 1 & 1 & 1 & 0\\\bottomrule
        \end{tabular}
        %%\end{table}
        
Sum of products 
 f(a,b,c,d) = $\bar a.\bar b.\bar c.d + \bar a.\bar b.c.d + \bar a.b.\bar c.d + \bar a.b.c.\bar d + a.\bar b.\bar c.\bar d + a.\bar b.\bar c.d + a.\bar b.c.\bar d + a.b.c.\bar d$

Product of sums 
 f(a,b,c,d) = $(a+b+c+d) . (a+b+\bar c+d) . (a+\bar b+c+d) . (a+\bar b+\bar c+\bar d) . (\bar a+b+\bar c+\bar d) . (\bar a+\bar b+c+d) . (\bar a+\bar b+c+\bar d) . (\bar a+\bar b+\bar c+\bar d)$

Karnough map
\begin{karnaugh-map}[4][4][1][cd][ab]
          \minterms{1, 3, 5, 6, 8, 9, 10, 14}
          \maxterms{0, 2, 4, 7, 11, 12, 13, 15}
        %\autoterms[0]
        % simplification
        \implicant{6}{14}
\implicant{8}{9}
\implicantedge{8}{8}{10}{10}
\implicant{1}{3}
\implicant{1}{5}
          %\implicant{5}{15}
          %\implicantedge{8}{8}{10}{10}
          %\implicantedge{8}{8}{10}{10}[8,10]
        \end{karnaugh-map}

Simplified Sum of products: $ b.c.\bar d + a.\bar b.\bar c + a.\bar b.\bar d + \bar a.\bar b.d + \bar a.\bar c.d $

Simplified Product of sums: $(a+b+d).(\bar b+c+d).(\bar a+\bar b+c).(\bar a+\bar c+\bar d).(\bar b+\bar c+\bar d)$
\paragraph{Logigramme} de la fonction\\
        %%\missingfigure[figwidth=6cm]{Logigramme}

 \begin{tikzpicture}

\node (x) at (0, 5*1.5) {$A$};
            \node (y) at (0.5, 5*1.5) {$B$};
            \node (z) at (1, 5*1.5) {$C$};
            \node (w) at (1.5, 5*1.5) {$D$};
            \node[not gate US, draw, rotate=270] at ($(x) + (0.25, -0.4)$) (notx) {};
            \draw (x) -- (notx.input); 
            \node[not gate US, draw, rotate=270] at ($(y) + (0.25, -0.4)$) (noty) {};
            \draw (y) -- (noty.input); 
            \node[not gate US, draw, rotate=270] at ($(z) + (0.25, -0.4)$) (notz) {};
            \draw (z) -- (notz.input);
            \node[not gate US, draw, rotate=270] at ($(w) + (0.25, -0.4)$) (notw) {};
            \draw (w) -- (notw.input);
                
           
            \node[and gate US, draw, rotate=0, logic gate inputs=nnnn] at (2.5, 0*1.5) (xandy0) {};
            \draw (xandy0.output) -- node[above]{\scriptsize $ B.C.\bar D $} ($(xandy0) + (1.8, 0)$);
              
            %% Y
            \draw (y) -| ($(y) + (0, 0)$) |- (xandy0.input 2);
        
            %%Z
            \draw (z) -| ($(z) + (0, 0)$) |- (xandy0.input 3);
        
            %%W

            \draw [line width=0.25mm,   red] (notw.output) -- ([xshift=0cm]notw.output) |- (xandy0.input 4);
                  
           
            \node[and gate US, draw, rotate=0, logic gate inputs=nnnn] at (2.5, 1*1.5) (xandy1) {};
            \draw (xandy1.output) -- node[above]{\scriptsize $ A.\bar B.\bar C $} ($(xandy1) + (1.8, 0)$);
            %% X
            \draw (x) -| ($(x) + (0, 0)$) |- (xandy1.input 1);
            
            %Y'

            \draw [line width=0.25mm,   red] (noty.output) -- ([xshift=0cm]noty.output) |- (xandy1.input 2);
          
            %%Z'

            \draw [line width=0.25mm,   red] (notz.output) -- ([xshift=0cm]notz.output) |- (xandy1.input 3);
                  
           
            \node[and gate US, draw, rotate=0, logic gate inputs=nnnn] at (2.5, 2*1.5) (xandy2) {};
            \draw (xandy2.output) -- node[above]{\scriptsize $ A.\bar B.\bar D $} ($(xandy2) + (1.8, 0)$);
            %% X
            \draw (x) -| ($(x) + (0, 0)$) |- (xandy2.input 1);
            
            %Y'

            \draw [line width=0.25mm,   red] (noty.output) -- ([xshift=0cm]noty.output) |- (xandy2.input 2);
          
            %%W

            \draw [line width=0.25mm,   red] (notw.output) -- ([xshift=0cm]notw.output) |- (xandy2.input 4);
                  
           
            \node[and gate US, draw, rotate=0, logic gate inputs=nnnn] at (2.5, 3*1.5) (xandy3) {};
            \draw (xandy3.output) -- node[above]{\scriptsize $ \bar A.\bar B.D $} ($(xandy3) + (1.8, 0)$);
            
            %% X'

            \draw  [line width=0.25mm,   red] (notx.output) -- ([xshift=0cm]notx.output) |- (xandy3.input 1);
            
            %Y'

            \draw [line width=0.25mm,   red] (noty.output) -- ([xshift=0cm]noty.output) |- (xandy3.input 2);
              %%W
            \draw (w) -| ($(w) + (0, 0)$) |- (xandy3.input 4);
                
           
            \node[and gate US, draw, rotate=0, logic gate inputs=nnnn] at (2.5, 4*1.5) (xandy4) {};
            \draw (xandy4.output) -- node[above]{\scriptsize $ \bar A.\bar C.D $} ($(xandy4) + (1.8, 0)$);
            
            %% X'

            \draw  [line width=0.25mm,   red] (notx.output) -- ([xshift=0cm]notx.output) |- (xandy4.input 1);
            
            %%Z'

            \draw [line width=0.25mm,   red] (notz.output) -- ([xshift=0cm]notz.output) |- (xandy4.input 3);
              %%W
            \draw (w) -| ($(w) + (0, 0)$) |- (xandy4.input 4);
        \node[or gate US, draw, rotate=0, logic gate inputs=nnnnnn] at (5.5, 5*0.6) (xory) {};


                    \draw (xory.output) -- node[above]{\scriptsize$F$} ($(xory) + (1, 0)$);

\draw (xandy0.output) -- ([xshift=1.40cm]xandy0.output) |- (xory.input 5);

\draw (xandy1.output) -- ([xshift=1.35cm]xandy1.output) |- (xory.input 4);

\draw (xandy2.output) -- ([xshift=1.30cm]xandy2.output) |- (xory.input 3);

\draw (xandy3.output) -- ([xshift=1.25cm]xandy3.output) |- (xory.input 2);

\draw (xandy4.output) -- ([xshift=1.30cm]xandy4.output) |- (xory.input 1);

 \end{tikzpicture}


\paragraph{Q2}

Simplifier l'expression suivante

S = $ b.c + b.d + \bar a.b + a.\bar c.d + a.\bar b.\bar c + \bar a.\bar c.\bar d + \bar b.\bar c.\bar d  +  b.\bar c + a.b.\bar d  +  a.b.d + \bar a.c.d + \bar a.\bar b.c  +  \bar a.c + b.c.d + a.\bar b.\bar c $

 = $ b + a.\bar c + \bar a.c + \bar a.\bar d $


Karnough map
\begin{karnaugh-map}[4][4][1][cd][ab]
          \minterms{0, 2, 3, 4, 5, 6, 7, 8, 9, 12, 13, 14, 15}
          \maxterms{1, 10, 11}
        %\autoterms[0]
        % simplification
        \implicant{4}{14}
\implicant{12}{9}
\implicant{3}{6}
\implicantedge{0}{4}{2}{6}
          %\implicant{5}{15}
          %\implicantedge{8}{8}{10}{10}
          %\implicantedge{8}{8}{10}{10}[8,10]
        \end{karnaugh-map}

\pagebreak
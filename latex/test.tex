
\section{Question}

\paragraph{Q1}

Etudier la fonction suivante

f(a,b,c,d)= $a.\bar b.d+\bar a.b.c.d+\bar a.\bar b.\bar c.\bar d + a.\bar d+b.\bar c+b.\bar d+\bar c.\bar d$

\paragraph{Q2}

Simplifier l'expression suivante par les proprietés algébreiques 

S = $b.\bar c+\bar c.d+\bar a.b.d+\bar a.\bar b.\bar d + a.d+a.b.c+b.c.d+\bar a.\bar b.\bar c + \bar a.\bar b+\bar b.\bar c+\bar b.\bar d+a.b.c.d+\bar a.\bar c.d + a.\bar c.\bar d+\bar a.b.c.d$

\hrule width 1\linewidth
\paragraph{Q1}

Etudier la fonction suivante

f(a,b,c,d)= $a.\bar b.d+\bar a.b.c.d+\bar a.\bar b.\bar c.\bar d + a.\bar d+b.\bar c+b.\bar d+\bar c.\bar d$

\paragraph{Q2}

Simplifier l'expression suivante par les proprietés algébreiques 

S = $b.\bar c+\bar c.d+\bar a.b.d+\bar a.\bar b.\bar d + a.d+a.b.c+b.c.d+\bar a.\bar b.\bar c + \bar a.\bar b+\bar b.\bar c+\bar b.\bar d+a.b.c.d+\bar a.\bar c.d + a.\bar c.\bar d+\bar a.b.c.d$

\hrule width 1\linewidth
\paragraph{Q1}

Etudier la fonction suivante

f(a,b,c,d)= $a.\bar b.d+\bar a.b.c.d+\bar a.\bar b.\bar c.\bar d + a.\bar d+b.\bar c+b.\bar d+\bar c.\bar d$

\paragraph{Q2}

Simplifier l'expression suivante par les proprietés algébreiques 

S = $b.\bar c+\bar c.d+\bar a.b.d+\bar a.\bar b.\bar d + a.d+a.b.c+b.c.d+\bar a.\bar b.\bar c + \bar a.\bar b+\bar b.\bar c+\bar b.\bar d+a.b.c.d+\bar a.\bar c.d + a.\bar c.\bar d+\bar a.b.c.d$

\hrule width 1\linewidth
\paragraph{Q1}

Etudier la fonction suivante

f(a,b,c,d)= $a.\bar b.d+\bar a.b.c.d+\bar a.\bar b.\bar c.\bar d + a.\bar d+b.\bar c+b.\bar d+\bar c.\bar d$

\paragraph{Q2}

Simplifier l'expression suivante par les proprietés algébreiques 

S = $b.\bar c+\bar c.d+\bar a.b.d+\bar a.\bar b.\bar d + a.d+a.b.c+b.c.d+\bar a.\bar b.\bar c + \bar a.\bar b+\bar b.\bar c+\bar b.\bar d+a.b.c.d+\bar a.\bar c.d + a.\bar c.\bar d+\bar a.b.c.d$

\hrule width 1\linewidth
\paragraph{Q1}

Etudier la fonction suivante

f(a,b,c,d)= $a.\bar b.d+\bar a.b.c.d+\bar a.\bar b.\bar c.\bar d + a.\bar d+b.\bar c+b.\bar d+\bar c.\bar d$

\paragraph{Q2}

Simplifier l'expression suivante par les proprietés algébreiques 

S = $b.\bar c+\bar c.d+\bar a.b.d+\bar a.\bar b.\bar d + a.d+a.b.c+b.c.d+\bar a.\bar b.\bar c + \bar a.\bar b+\bar b.\bar c+\bar b.\bar d+a.b.c.d+\bar a.\bar c.d + a.\bar c.\bar d+\bar a.b.c.d$

\hrule width 1\linewidth\pagebreak
\subsection{Correction}

\paragraph{Q1}

f(a,b,c,d)=$a.\bar b.d+\bar a.b.c.d+\bar a.\bar b.\bar c.\bar d + a.\bar d+b.\bar c+b.\bar d+\bar c.\bar d$
f(a,b,c,D)=$ \sum a.\bar b.d+\bar a.b.c.d+\bar a.\bar b.\bar c.\bar d + a.\bar d+b.\bar c+b.\bar d+\bar c.\bar d $ 

%%\begin{table}
        \begin{tabular}{|c|c|c|c|c||c|}
    \toprule
         & A & B & C & D & F\\ \midrule0 & 0 & 0 & 0 & 0 & 1\\1 & 0 & 0 & 0 & 1 & 0\\2 & 0 & 0 & 1 & 0 & 0\\3 & 0 & 0 & 1 & 1 & 0\\\midrule4 & 0 & 1 & 0 & 0 & 1\\5 & 0 & 1 & 0 & 1 & 1\\6 & 0 & 1 & 1 & 0 & 1\\7 & 0 & 1 & 1 & 1 & 1\\\midrule8 & 1 & 0 & 0 & 0 & 1\\9 & 1 & 0 & 0 & 1 & 1\\10 & 1 & 0 & 1 & 0 & 1\\11 & 1 & 0 & 1 & 1 & 1\\\midrule12 & 1 & 1 & 0 & 0 & 1\\13 & 1 & 1 & 0 & 1 & 1\\14 & 1 & 1 & 1 & 0 & 1\\15 & 1 & 1 & 1 & 1 & 0\\\bottomrule
        \end{tabular}
        %%\end{table}
        
Sum of products 
 f(a,b,c,d) = $\bar a.\bar b.\bar c.\bar d + \bar a.b.\bar c.\bar d + \bar a.\bar b.\bar c.d + \bar a.\bar b.c.\bar d + \bar a.\bar b.\bar c.d + a.\bar b.\bar c.\bar d + \bar a.\bar b.\bar c.d + \bar a.\bar b.c.\bar d + \bar a.\bar b.\bar c.d + \bar a.b.\bar c.\bar d + \bar a.\bar b.\bar c.d + \bar a.\bar b.c.\bar d$

Product of sums 
 f(a,b,c,d) = $\bar a.\bar b.\bar c.\bar d.\bar a.\bar b.\bar c.d.\bar a.\bar b.\bar c.\bar d.\bar a.\bar b.\bar c.\bar d$

Karnough map\todo{fix map}
\begin{karnaugh-map}[4][4][1][cd][ab]
          \minterms{0, 4, 5, 6, 7, 8, 9, 10, 11, 12, 13, 14}
          \maxterms{1, 2, 3, 15}
        %\autoterms[0]
          %\implicant{5}{15}
          %\implicantedge{8}{8}{10}{10}
          %\implicantedge{8}{8}{10}{10}[8,10]
        \end{karnaugh-map}

Simplified Sum of products: $a.\bar b+\bar a.b+b.\bar c+b.\bar d+\bar c.\bar d$

Simplified Product of sums: $(a+b+b+\bar b+\bar c+\bar c).(a+\bar d)+\bar d).(\bar a$
\paragraph{Logigramme} de la fonction\\
        %%\missingfigure[figwidth=6cm]{Logigramme}

 \begin{tikzpicture}

\node (x) at (0, 5*1.5) {$A$};
            \node (y) at (0.5, 5*1.5) {$B$};
            \node (z) at (1, 5*1.5) {$C$};
            \node (w) at (1.5, 5*1.5) {$D$};
            \node[not gate US, draw, rotate=270] at ($(x) + (0.25, -0.3)$) (notx) {};
            \draw (x) -- (notx.input); 
            \node[not gate US, draw, rotate=270] at ($(y) + (0.25, -0.3)$) (noty) {};
            \draw (y) -- (noty.input); 
            \node[not gate US, draw, rotate=270] at ($(z) + (0.25, -0.3)$) (notz) {};
            \draw (z) -- (notz.input);
            \node[not gate US, draw, rotate=270] at ($(w) + (0.25, -0.3)$) (notw) {};
            \draw (w) -- (notw.input);
                
           
            \node[and gate US, draw, rotate=0, logic gate inputs=nnnn] at (2.5, 0*1.5) (xandy0) {};
            \draw (xandy0.output) -- node[above]{\scriptsize $A.\bar B$} ($(xandy0) + (1.8, 0)$);
            %% X
            \draw (x) -| ($(x) + (0, 0)$) |- (xandy0.input 1);
            
            %Y'

            \draw [line width=0.25mm,   red] (noty.output) -- ([xshift=0cm]noty.output) |- (xandy0.input 2);
                  
           
            \node[and gate US, draw, rotate=0, logic gate inputs=nnnn] at (2.5, 1*1.5) (xandy1) {};
            \draw (xandy1.output) -- node[above]{\scriptsize $\bar A.B$} ($(xandy1) + (1.8, 0)$);
            
            %% X'

            \draw  [line width=0.25mm,   red] (notx.output) -- ([xshift=0cm]notx.output) |- (xandy1.input 1);
              
            %% Y
            \draw (y) -| ($(y) + (0, 0)$) |- (xandy1.input 2);
                
           
            \node[and gate US, draw, rotate=0, logic gate inputs=nnnn] at (2.5, 2*1.5) (xandy2) {};
            \draw (xandy2.output) -- node[above]{\scriptsize $B.\bar C$} ($(xandy2) + (1.8, 0)$);
              
            %% Y
            \draw (y) -| ($(y) + (0, 0)$) |- (xandy2.input 2);
        
            %%Z'

            \draw [line width=0.25mm,   red] (notz.output) -- ([xshift=0cm]notz.output) |- (xandy2.input 3);
                  
           
            \node[and gate US, draw, rotate=0, logic gate inputs=nnnn] at (2.5, 3*1.5) (xandy3) {};
            \draw (xandy3.output) -- node[above]{\scriptsize $B.\bar D$} ($(xandy3) + (1.8, 0)$);
              
            %% Y
            \draw (y) -| ($(y) + (0, 0)$) |- (xandy3.input 2);
        
            %%W

            \draw [line width=0.25mm,   red] (notw.output) -- ([xshift=0cm]notw.output) |- (xandy3.input 4);
                  
           
            \node[and gate US, draw, rotate=0, logic gate inputs=nnnn] at (2.5, 4*1.5) (xandy4) {};
            \draw (xandy4.output) -- node[above]{\scriptsize $\bar C.\bar D$} ($(xandy4) + (1.8, 0)$);
            
            %%Z'

            \draw [line width=0.25mm,   red] (notz.output) -- ([xshift=0cm]notz.output) |- (xandy4.input 3);
          
            %%W

            \draw [line width=0.25mm,   red] (notw.output) -- ([xshift=0cm]notw.output) |- (xandy4.input 4);
          \node[or gate US, draw, rotate=0, logic gate inputs=nnnnnn] at (5.5, 5*0.5) (xory) {};


                    \draw (xory.output) -- node[above]{\scriptsize$F$} ($(xory) + (1, 0)$);

\draw (xandy0.output) -- ([xshift=1.40cm]xandy0.output) |- (xory.input 5);

\draw (xandy1.output) -- ([xshift=1.35cm]xandy1.output) |- (xory.input 4);

\draw (xandy2.output) -- ([xshift=1.30cm]xandy2.output) |- (xory.input 3);

\draw (xandy3.output) -- ([xshift=1.25cm]xandy3.output) |- (xory.input 2);

\draw (xandy4.output) -- ([xshift=1.30cm]xandy4.output) |- (xory.input 1);

 \end{tikzpicture}


\paragraph{Q2}

Simplifier l'expression suivante

S = $b.\bar c+\bar c.d+\bar a.b.d+\bar a.\bar b.\bar d + a.d+a.b.c+b.c.d+\bar a.\bar b.\bar c + \bar a.\bar b+\bar b.\bar c+\bar b.\bar d+a.b.c.d+\bar a.\bar c.d + a.\bar c.\bar d+\bar a.b.c.d$

 = $a+d+\bar b+\bar c$


Karnough map\todo{fix map}
\begin{karnaugh-map}[4][4][1][cd][ab]
          \minterms{0, 1, 2, 3, 4, 5, 7, 8, 9, 10, 11, 12, 13, 14, 15}
          \maxterms{6}
        %\autoterms[0]
          %\implicant{5}{15}
          %\implicantedge{8}{8}{10}{10}
          %\implicantedge{8}{8}{10}{10}[8,10]
        \end{karnaugh-map}

\pagebreak
\section{Question}

\paragraph{Q1}

Etudier la fonction suivante

f(a,b,c,d)= $a.b.c+a.\bar c.d+\bar a.c.d+\bar b.\bar c.\bar d + b.d+\bar a.\bar b.\bar d+\bar a.\bar c.\bar d$

\paragraph{Q2}

Simplifier l'expression suivante par les proprietés algébreiques 

S = $a.c.\bar d+\bar a.b.c+a.b.\bar c.d+\bar a.\bar b.\bar c+\bar a.\bar c.\bar d + a.b+a.\bar d+b.c.\bar d+\bar b.c.d + a.b+b.c+a.c.d+\bar a.\bar d+\bar c.\bar d + b.c.d+a.b.\bar c+b.\bar c.\bar d+a.\bar b.c.\bar d$

\hrule width 1\linewidth
\paragraph{Q1}

Etudier la fonction suivante

f(a,b,c,d)= $a.b.c+a.\bar c.d+\bar a.c.d+\bar b.\bar c.\bar d + b.d+\bar a.\bar b.\bar d+\bar a.\bar c.\bar d$

\paragraph{Q2}

Simplifier l'expression suivante par les proprietés algébreiques 

S = $a.c.\bar d+\bar a.b.c+a.b.\bar c.d+\bar a.\bar b.\bar c+\bar a.\bar c.\bar d + a.b+a.\bar d+b.c.\bar d+\bar b.c.d + a.b+b.c+a.c.d+\bar a.\bar d+\bar c.\bar d + b.c.d+a.b.\bar c+b.\bar c.\bar d+a.\bar b.c.\bar d$

\hrule width 1\linewidth
\paragraph{Q1}

Etudier la fonction suivante

f(a,b,c,d)= $a.b.c+a.\bar c.d+\bar a.c.d+\bar b.\bar c.\bar d + b.d+\bar a.\bar b.\bar d+\bar a.\bar c.\bar d$

\paragraph{Q2}

Simplifier l'expression suivante par les proprietés algébreiques 

S = $a.c.\bar d+\bar a.b.c+a.b.\bar c.d+\bar a.\bar b.\bar c+\bar a.\bar c.\bar d + a.b+a.\bar d+b.c.\bar d+\bar b.c.d + a.b+b.c+a.c.d+\bar a.\bar d+\bar c.\bar d + b.c.d+a.b.\bar c+b.\bar c.\bar d+a.\bar b.c.\bar d$

\hrule width 1\linewidth
\paragraph{Q1}

Etudier la fonction suivante

f(a,b,c,d)= $a.b.c+a.\bar c.d+\bar a.c.d+\bar b.\bar c.\bar d + b.d+\bar a.\bar b.\bar d+\bar a.\bar c.\bar d$

\paragraph{Q2}

Simplifier l'expression suivante par les proprietés algébreiques 

S = $a.c.\bar d+\bar a.b.c+a.b.\bar c.d+\bar a.\bar b.\bar c+\bar a.\bar c.\bar d + a.b+a.\bar d+b.c.\bar d+\bar b.c.d + a.b+b.c+a.c.d+\bar a.\bar d+\bar c.\bar d + b.c.d+a.b.\bar c+b.\bar c.\bar d+a.\bar b.c.\bar d$

\hrule width 1\linewidth
\paragraph{Q1}

Etudier la fonction suivante

f(a,b,c,d)= $a.b.c+a.\bar c.d+\bar a.c.d+\bar b.\bar c.\bar d + b.d+\bar a.\bar b.\bar d+\bar a.\bar c.\bar d$

\paragraph{Q2}

Simplifier l'expression suivante par les proprietés algébreiques 

S = $a.c.\bar d+\bar a.b.c+a.b.\bar c.d+\bar a.\bar b.\bar c+\bar a.\bar c.\bar d + a.b+a.\bar d+b.c.\bar d+\bar b.c.d + a.b+b.c+a.c.d+\bar a.\bar d+\bar c.\bar d + b.c.d+a.b.\bar c+b.\bar c.\bar d+a.\bar b.c.\bar d$

\hrule width 1\linewidth\pagebreak
\subsection{Correction}

\paragraph{Q1}

f(a,b,c,d)=$a.b.c+a.\bar c.d+\bar a.c.d+\bar b.\bar c.\bar d + b.d+\bar a.\bar b.\bar d+\bar a.\bar c.\bar d$
f(a,b,c,D)=$ \sum a.b.c+a.\bar c.d+\bar a.c.d+\bar b.\bar c.\bar d + b.d+\bar a.\bar b.\bar d+\bar a.\bar c.\bar d $ 

%%\begin{table}
        \begin{tabular}{|c|c|c|c|c||c|}
    \toprule
         & A & B & C & D & F\\ \midrule0 & 0 & 0 & 0 & 0 & 1\\1 & 0 & 0 & 0 & 1 & 0\\2 & 0 & 0 & 1 & 0 & 1\\3 & 0 & 0 & 1 & 1 & 1\\\midrule4 & 0 & 1 & 0 & 0 & 1\\5 & 0 & 1 & 0 & 1 & 1\\6 & 0 & 1 & 1 & 0 & 0\\7 & 0 & 1 & 1 & 1 & 1\\\midrule8 & 1 & 0 & 0 & 0 & 1\\9 & 1 & 0 & 0 & 1 & 1\\10 & 1 & 0 & 1 & 0 & 0\\11 & 1 & 0 & 1 & 1 & 0\\\midrule12 & 1 & 1 & 0 & 0 & 0\\13 & 1 & 1 & 0 & 1 & 1\\14 & 1 & 1 & 1 & 0 & 1\\15 & 1 & 1 & 1 & 1 & 1\\\bottomrule
        \end{tabular}
        %%\end{table}
        
Sum of products 
 f(a,b,c,d) = $\bar a.\bar b.\bar c.\bar d + \bar a.\bar b.c.\bar d + \bar a.\bar b.\bar c.d + \bar a.b.\bar c.\bar d + \bar a.\bar b.\bar c.d + \bar a.\bar b.\bar c.d + a.\bar b.\bar c.\bar d + \bar a.\bar b.\bar c.d + \bar a.\bar b.\bar c.d + \bar a.\bar b.c.\bar d + \bar a.\bar b.\bar c.d$

Product of sums 
 f(a,b,c,d) = $\bar a.\bar b.\bar c.\bar d.\bar a.\bar b.\bar c.d.\bar a.\bar b.\bar c.d.\bar a.\bar b.\bar c.\bar d.\bar a.\bar b.c.d$

Karnough map\todo{fix map}
\begin{karnaugh-map}[4][4][1][cd][ab]
          \minterms{0, 2, 3, 4, 5, 7, 8, 9, 13, 14, 15}
          \maxterms{1, 6, 10, 11, 12}
        %\autoterms[0]
          %\implicant{5}{15}
          %\implicantedge{8}{8}{10}{10}
          %\implicantedge{8}{8}{10}{10}[8,10]
        \end{karnaugh-map}

Simplified Sum of products: $b.d+a.b.c+a.\bar c.d+\bar a.c.d+a.\bar b.\bar c+\bar a.b.\bar c+\bar a.\bar b.c+\bar a.\bar b.\bar d+\bar a.\bar c.\bar d+\bar b.\bar c.\bar d$

Simplified Product of sums: $(b+\bar a+\bar a+b+\bar b+\bar b)+c+\bar c).(a+\bar c).(c+d+d+\bar d).(a$
\paragraph{Logigramme} de la fonction\\
        %%\missingfigure[figwidth=6cm]{Logigramme}

 \begin{tikzpicture}

\node (x) at (0, 10*1.5) {$A$};
            \node (y) at (0.5, 10*1.5) {$B$};
            \node (z) at (1, 10*1.5) {$C$};
            \node (w) at (1.5, 10*1.5) {$D$};
            \node[not gate US, draw, rotate=270] at ($(x) + (0.25, -0.3)$) (notx) {};
            \draw (x) -- (notx.input); 
            \node[not gate US, draw, rotate=270] at ($(y) + (0.25, -0.3)$) (noty) {};
            \draw (y) -- (noty.input); 
            \node[not gate US, draw, rotate=270] at ($(z) + (0.25, -0.3)$) (notz) {};
            \draw (z) -- (notz.input);
            \node[not gate US, draw, rotate=270] at ($(w) + (0.25, -0.3)$) (notw) {};
            \draw (w) -- (notw.input);
                
           
            \node[and gate US, draw, rotate=0, logic gate inputs=nnnn] at (2.5, 0*1.5) (xandy0) {};
            \draw (xandy0.output) -- node[above]{\scriptsize $B.D$} ($(xandy0) + (1.8, 0)$);
              
            %% Y
            \draw (y) -| ($(y) + (0, 0)$) |- (xandy0.input 2);
            %%W
            \draw (w) -| ($(w) + (0, 0)$) |- (xandy0.input 4);
                
           
            \node[and gate US, draw, rotate=0, logic gate inputs=nnnn] at (2.5, 1*1.5) (xandy1) {};
            \draw (xandy1.output) -- node[above]{\scriptsize $A.B.C$} ($(xandy1) + (1.8, 0)$);
            %% X
            \draw (x) -| ($(x) + (0, 0)$) |- (xandy1.input 1);
              
            %% Y
            \draw (y) -| ($(y) + (0, 0)$) |- (xandy1.input 2);
        
            %%Z
            \draw (z) -| ($(z) + (0, 0)$) |- (xandy1.input 3);
                
           
            \node[and gate US, draw, rotate=0, logic gate inputs=nnnn] at (2.5, 2*1.5) (xandy2) {};
            \draw (xandy2.output) -- node[above]{\scriptsize $A.\bar C.D$} ($(xandy2) + (1.8, 0)$);
            %% X
            \draw (x) -| ($(x) + (0, 0)$) |- (xandy2.input 1);
            
            %%Z'

            \draw [line width=0.25mm,   red] (notz.output) -- ([xshift=0cm]notz.output) |- (xandy2.input 3);
              %%W
            \draw (w) -| ($(w) + (0, 0)$) |- (xandy2.input 4);
                
           
            \node[and gate US, draw, rotate=0, logic gate inputs=nnnn] at (2.5, 3*1.5) (xandy3) {};
            \draw (xandy3.output) -- node[above]{\scriptsize $\bar A.C.D$} ($(xandy3) + (1.8, 0)$);
            
            %% X'

            \draw  [line width=0.25mm,   red] (notx.output) -- ([xshift=0cm]notx.output) |- (xandy3.input 1);
            
            %%Z
            \draw (z) -| ($(z) + (0, 0)$) |- (xandy3.input 3);
            %%W
            \draw (w) -| ($(w) + (0, 0)$) |- (xandy3.input 4);
                
           
            \node[and gate US, draw, rotate=0, logic gate inputs=nnnn] at (2.5, 4*1.5) (xandy4) {};
            \draw (xandy4.output) -- node[above]{\scriptsize $A.\bar B.\bar C$} ($(xandy4) + (1.8, 0)$);
            %% X
            \draw (x) -| ($(x) + (0, 0)$) |- (xandy4.input 1);
            
            %Y'

            \draw [line width=0.25mm,   red] (noty.output) -- ([xshift=0cm]noty.output) |- (xandy4.input 2);
          
            %%Z'

            \draw [line width=0.25mm,   red] (notz.output) -- ([xshift=0cm]notz.output) |- (xandy4.input 3);
                  
           
            \node[and gate US, draw, rotate=0, logic gate inputs=nnnn] at (2.5, 5*1.5) (xandy5) {};
            \draw (xandy5.output) -- node[above]{\scriptsize $\bar A.B.\bar C$} ($(xandy5) + (1.8, 0)$);
            
            %% X'

            \draw  [line width=0.25mm,   red] (notx.output) -- ([xshift=0cm]notx.output) |- (xandy5.input 1);
              
            %% Y
            \draw (y) -| ($(y) + (0, 0)$) |- (xandy5.input 2);
        
            %%Z'

            \draw [line width=0.25mm,   red] (notz.output) -- ([xshift=0cm]notz.output) |- (xandy5.input 3);
                  
           
            \node[and gate US, draw, rotate=0, logic gate inputs=nnnn] at (2.5, 6*1.5) (xandy6) {};
            \draw (xandy6.output) -- node[above]{\scriptsize $\bar A.\bar B.C$} ($(xandy6) + (1.8, 0)$);
            
            %% X'

            \draw  [line width=0.25mm,   red] (notx.output) -- ([xshift=0cm]notx.output) |- (xandy6.input 1);
            
            %Y'

            \draw [line width=0.25mm,   red] (noty.output) -- ([xshift=0cm]noty.output) |- (xandy6.input 2);
          
            %%Z
            \draw (z) -| ($(z) + (0, 0)$) |- (xandy6.input 3);
                
           
            \node[and gate US, draw, rotate=0, logic gate inputs=nnnn] at (2.5, 7*1.5) (xandy7) {};
            \draw (xandy7.output) -- node[above]{\scriptsize $\bar A.\bar B.\bar D$} ($(xandy7) + (1.8, 0)$);
            
            %% X'

            \draw  [line width=0.25mm,   red] (notx.output) -- ([xshift=0cm]notx.output) |- (xandy7.input 1);
            
            %Y'

            \draw [line width=0.25mm,   red] (noty.output) -- ([xshift=0cm]noty.output) |- (xandy7.input 2);
          
            %%W

            \draw [line width=0.25mm,   red] (notw.output) -- ([xshift=0cm]notw.output) |- (xandy7.input 4);
                  
           
            \node[and gate US, draw, rotate=0, logic gate inputs=nnnn] at (2.5, 8*1.5) (xandy8) {};
            \draw (xandy8.output) -- node[above]{\scriptsize $\bar A.\bar C.\bar D$} ($(xandy8) + (1.8, 0)$);
            
            %% X'

            \draw  [line width=0.25mm,   red] (notx.output) -- ([xshift=0cm]notx.output) |- (xandy8.input 1);
            
            %%Z'

            \draw [line width=0.25mm,   red] (notz.output) -- ([xshift=0cm]notz.output) |- (xandy8.input 3);
          
            %%W

            \draw [line width=0.25mm,   red] (notw.output) -- ([xshift=0cm]notw.output) |- (xandy8.input 4);
                  
           
            \node[and gate US, draw, rotate=0, logic gate inputs=nnnn] at (2.5, 9*1.5) (xandy9) {};
            \draw (xandy9.output) -- node[above]{\scriptsize $\bar B.\bar C.\bar D$} ($(xandy9) + (1.8, 0)$);
            
            %Y'

            \draw [line width=0.25mm,   red] (noty.output) -- ([xshift=0cm]noty.output) |- (xandy9.input 2);
          
            %%Z'

            \draw [line width=0.25mm,   red] (notz.output) -- ([xshift=0cm]notz.output) |- (xandy9.input 3);
          
            %%W

            \draw [line width=0.25mm,   red] (notw.output) -- ([xshift=0cm]notw.output) |- (xandy9.input 4);
          \node[or gate US, draw, rotate=0, logic gate inputs=nnnnnnnnnnn] at (5.5, 10*0.5) (xory) {};


                    \draw (xory.output) -- node[above]{\scriptsize$F$} ($(xory) + (1, 0)$);

\draw (xandy0.output) -- ([xshift=1.40cm]xandy0.output) |- (xory.input 10);

\draw (xandy1.output) -- ([xshift=1.35cm]xandy1.output) |- (xory.input 9);

\draw (xandy2.output) -- ([xshift=1.30cm]xandy2.output) |- (xory.input 8);

\draw (xandy3.output) -- ([xshift=1.25cm]xandy3.output) |- (xory.input 7);

\draw (xandy4.output) -- ([xshift=1.20cm]xandy4.output) |- (xory.input 6);

\draw (xandy5.output) -- ([xshift=1.15cm]xandy5.output) |- (xory.input 5);

\draw (xandy6.output) -- ([xshift=1.20cm]xandy6.output) |- (xory.input 4);

\draw (xandy7.output) -- ([xshift=1.25cm]xandy7.output) |- (xory.input 3);

\draw (xandy8.output) -- ([xshift=1.30cm]xandy8.output) |- (xory.input 2);

\draw (xandy9.output) -- ([xshift=1.35cm]xandy9.output) |- (xory.input 1);

 \end{tikzpicture}


\paragraph{Q2}

Simplifier l'expression suivante

S = $a.c.\bar d+\bar a.b.c+a.b.\bar c.d+\bar a.\bar b.\bar c+\bar a.\bar c.\bar d + a.b+a.\bar d+b.c.\bar d+\bar b.c.d + a.b+b.c+a.c.d+\bar a.\bar d+\bar c.\bar d + b.c.d+a.b.\bar c+b.\bar c.\bar d+a.\bar b.c.\bar d$

 = $c+\bar d+a.b+\bar a.\bar b$


Karnough map\todo{fix map}
\begin{karnaugh-map}[4][4][1][cd][ab]
          \minterms{0, 1, 2, 3, 4, 6, 7, 8, 10, 11, 12, 13, 14, 15}
          \maxterms{5, 9}
        %\autoterms[0]
          %\implicant{5}{15}
          %\implicantedge{8}{8}{10}{10}
          %\implicantedge{8}{8}{10}{10}[8,10]
        \end{karnaugh-map}

\pagebreak
\section{Question}

\paragraph{Q1}

Etudier la fonction suivante

f(a,b,c,d)= $c.d+\bar a.b+\bar b.c+\bar b.d + a.b.d+a.b.\bar c+\bar a.\bar b.c.d+\bar a.\bar b.\bar c.\bar d$

\paragraph{Q2}

Simplifier les fonctions suivantes

\begin{karnaugh-map}[4][4][1][cd][ab]
          \minterms{0, 1, 3, 7, 8, 9, 10, 13, 14, 15}
          \maxterms{2, 4, 5, 6, 11, 12}
        %\autoterms[0]
          %\implicant{5}{15}
          %\implicantedge{8}{8}{10}{10}
          %\implicantedge{8}{8}{10}{10}[8,10]
        \end{karnaugh-map}\begin{karnaugh-map}[4][4][1][cd][ab]
          \minterms{0, 1, 2, 4, 6, 8, 12, 13, 14}
          \maxterms{3, 5, 7, 9, 10, 11, 15}
        %\autoterms[0]
          %\implicant{5}{15}
          %\implicantedge{8}{8}{10}{10}
          %\implicantedge{8}{8}{10}{10}[8,10]
        \end{karnaugh-map}\begin{karnaugh-map}[4][4][1][cd][ab]
          \minterms{0, 4, 7, 12}
          \maxterms{1, 2, 3, 5, 6, 8, 9, 10, 11, 13, 14, 15}
        %\autoterms[0]
          %\implicant{5}{15}
          %\implicantedge{8}{8}{10}{10}
          %\implicantedge{8}{8}{10}{10}[8,10]
        \end{karnaugh-map}
\hrule width 1\linewidth
\paragraph{Q1}

Etudier la fonction suivante

f(a,b,c,d)= $c.d+\bar a.b+\bar b.c+\bar b.d + a.b.d+a.b.\bar c+\bar a.\bar b.c.d+\bar a.\bar b.\bar c.\bar d$

\paragraph{Q2}

Simplifier les fonctions suivantes

\begin{karnaugh-map}[4][4][1][cd][ab]
          \minterms{0, 1, 3, 7, 8, 9, 10, 13, 14, 15}
          \maxterms{2, 4, 5, 6, 11, 12}
        %\autoterms[0]
          %\implicant{5}{15}
          %\implicantedge{8}{8}{10}{10}
          %\implicantedge{8}{8}{10}{10}[8,10]
        \end{karnaugh-map}\begin{karnaugh-map}[4][4][1][cd][ab]
          \minterms{0, 1, 2, 4, 6, 8, 12, 13, 14}
          \maxterms{3, 5, 7, 9, 10, 11, 15}
        %\autoterms[0]
          %\implicant{5}{15}
          %\implicantedge{8}{8}{10}{10}
          %\implicantedge{8}{8}{10}{10}[8,10]
        \end{karnaugh-map}\begin{karnaugh-map}[4][4][1][cd][ab]
          \minterms{0, 4, 7, 12}
          \maxterms{1, 2, 3, 5, 6, 8, 9, 10, 11, 13, 14, 15}
        %\autoterms[0]
          %\implicant{5}{15}
          %\implicantedge{8}{8}{10}{10}
          %\implicantedge{8}{8}{10}{10}[8,10]
        \end{karnaugh-map}
\hrule width 1\linewidth
\paragraph{Q1}

Etudier la fonction suivante

f(a,b,c,d)= $c.d+\bar a.b+\bar b.c+\bar b.d + a.b.d+a.b.\bar c+\bar a.\bar b.c.d+\bar a.\bar b.\bar c.\bar d$

\paragraph{Q2}

Simplifier les fonctions suivantes

\begin{karnaugh-map}[4][4][1][cd][ab]
          \minterms{0, 1, 3, 7, 8, 9, 10, 13, 14, 15}
          \maxterms{2, 4, 5, 6, 11, 12}
        %\autoterms[0]
          %\implicant{5}{15}
          %\implicantedge{8}{8}{10}{10}
          %\implicantedge{8}{8}{10}{10}[8,10]
        \end{karnaugh-map}\begin{karnaugh-map}[4][4][1][cd][ab]
          \minterms{0, 1, 2, 4, 6, 8, 12, 13, 14}
          \maxterms{3, 5, 7, 9, 10, 11, 15}
        %\autoterms[0]
          %\implicant{5}{15}
          %\implicantedge{8}{8}{10}{10}
          %\implicantedge{8}{8}{10}{10}[8,10]
        \end{karnaugh-map}\begin{karnaugh-map}[4][4][1][cd][ab]
          \minterms{0, 4, 7, 12}
          \maxterms{1, 2, 3, 5, 6, 8, 9, 10, 11, 13, 14, 15}
        %\autoterms[0]
          %\implicant{5}{15}
          %\implicantedge{8}{8}{10}{10}
          %\implicantedge{8}{8}{10}{10}[8,10]
        \end{karnaugh-map}
\hrule width 1\linewidth
\paragraph{Q1}

Etudier la fonction suivante

f(a,b,c,d)= $c.d+\bar a.b+\bar b.c+\bar b.d + a.b.d+a.b.\bar c+\bar a.\bar b.c.d+\bar a.\bar b.\bar c.\bar d$

\paragraph{Q2}

Simplifier les fonctions suivantes

\begin{karnaugh-map}[4][4][1][cd][ab]
          \minterms{0, 1, 3, 7, 8, 9, 10, 13, 14, 15}
          \maxterms{2, 4, 5, 6, 11, 12}
        %\autoterms[0]
          %\implicant{5}{15}
          %\implicantedge{8}{8}{10}{10}
          %\implicantedge{8}{8}{10}{10}[8,10]
        \end{karnaugh-map}\begin{karnaugh-map}[4][4][1][cd][ab]
          \minterms{0, 1, 2, 4, 6, 8, 12, 13, 14}
          \maxterms{3, 5, 7, 9, 10, 11, 15}
        %\autoterms[0]
          %\implicant{5}{15}
          %\implicantedge{8}{8}{10}{10}
          %\implicantedge{8}{8}{10}{10}[8,10]
        \end{karnaugh-map}\begin{karnaugh-map}[4][4][1][cd][ab]
          \minterms{0, 4, 7, 12}
          \maxterms{1, 2, 3, 5, 6, 8, 9, 10, 11, 13, 14, 15}
        %\autoterms[0]
          %\implicant{5}{15}
          %\implicantedge{8}{8}{10}{10}
          %\implicantedge{8}{8}{10}{10}[8,10]
        \end{karnaugh-map}
\hrule width 1\linewidth
\paragraph{Q1}

Etudier la fonction suivante

f(a,b,c,d)= $c.d+\bar a.b+\bar b.c+\bar b.d + a.b.d+a.b.\bar c+\bar a.\bar b.c.d+\bar a.\bar b.\bar c.\bar d$

\paragraph{Q2}

Simplifier les fonctions suivantes

\begin{karnaugh-map}[4][4][1][cd][ab]
          \minterms{0, 1, 3, 7, 8, 9, 10, 13, 14, 15}
          \maxterms{2, 4, 5, 6, 11, 12}
        %\autoterms[0]
          %\implicant{5}{15}
          %\implicantedge{8}{8}{10}{10}
          %\implicantedge{8}{8}{10}{10}[8,10]
        \end{karnaugh-map}\begin{karnaugh-map}[4][4][1][cd][ab]
          \minterms{0, 1, 2, 4, 6, 8, 12, 13, 14}
          \maxterms{3, 5, 7, 9, 10, 11, 15}
        %\autoterms[0]
          %\implicant{5}{15}
          %\implicantedge{8}{8}{10}{10}
          %\implicantedge{8}{8}{10}{10}[8,10]
        \end{karnaugh-map}\begin{karnaugh-map}[4][4][1][cd][ab]
          \minterms{0, 4, 7, 12}
          \maxterms{1, 2, 3, 5, 6, 8, 9, 10, 11, 13, 14, 15}
        %\autoterms[0]
          %\implicant{5}{15}
          %\implicantedge{8}{8}{10}{10}
          %\implicantedge{8}{8}{10}{10}[8,10]
        \end{karnaugh-map}
\hrule width 1\linewidth\pagebreak
\subsection{Correction}

\paragraph{Q1}

f(a,b,c,d)=$c.d+\bar a.b+\bar b.c+\bar b.d + a.b.d+a.b.\bar c+\bar a.\bar b.c.d+\bar a.\bar b.\bar c.\bar d$
f(a,b,c,D)=$ \sum c.d+\bar a.b+\bar b.c+\bar b.d + a.b.d+a.b.\bar c+\bar a.\bar b.c.d+\bar a.\bar b.\bar c.\bar d $ 

%%\begin{table}
        \begin{tabular}{|c|c|c|c|c||c|}
    \toprule
         & A & B & C & D & F\\ \midrule0 & 0 & 0 & 0 & 0 & 1\\1 & 0 & 0 & 0 & 1 & 1\\2 & 0 & 0 & 1 & 0 & 1\\3 & 0 & 0 & 1 & 1 & 1\\\midrule4 & 0 & 1 & 0 & 0 & 1\\5 & 0 & 1 & 0 & 1 & 1\\6 & 0 & 1 & 1 & 0 & 1\\7 & 0 & 1 & 1 & 1 & 1\\\midrule8 & 1 & 0 & 0 & 0 & 0\\9 & 1 & 0 & 0 & 1 & 1\\10 & 1 & 0 & 1 & 0 & 1\\11 & 1 & 0 & 1 & 1 & 1\\\midrule12 & 1 & 1 & 0 & 0 & 1\\13 & 1 & 1 & 0 & 1 & 1\\14 & 1 & 1 & 1 & 0 & 0\\15 & 1 & 1 & 1 & 1 & 1\\\bottomrule
        \end{tabular}
        %%\end{table}
        
Sum of products 
 f(a,b,c,d) = $\bar a.\bar b.\bar c.\bar d + \bar a.\bar b.\bar c.d + \bar a.\bar b.c.\bar d + \bar a.\bar b.\bar c.d + \bar a.b.\bar c.\bar d + \bar a.\bar b.\bar c.d + \bar a.\bar b.c.\bar d + \bar a.\bar b.\bar c.d + \bar a.\bar b.\bar c.d + \bar a.\bar b.c.\bar d + \bar a.\bar b.\bar c.d + \bar a.b.\bar c.\bar d + \bar a.\bar b.\bar c.d + \bar a.\bar b.\bar c.d$

Product of sums 
 f(a,b,c,d) = $\bar a.b.c.d.\bar a.\bar b.\bar c.d$

Karnough map\todo{fix map}
\begin{karnaugh-map}[4][4][1][cd][ab]
          \minterms{0, 1, 2, 3, 4, 5, 6, 7, 9, 10, 11, 12, 13, 15}
          \maxterms{8, 14}
        %\autoterms[0]
          %\implicant{5}{15}
          %\implicantedge{8}{8}{10}{10}
          %\implicantedge{8}{8}{10}{10}[8,10]
        \end{karnaugh-map}

Simplified Sum of products: $d+\bar a+b.\bar c+\bar b.c$

Simplified Product of sums: $(b+\bar a+\bar a).(d+\bar b+c+\bar c)+d$
\paragraph{Logigramme} de la fonction\\
        %%\missingfigure[figwidth=6cm]{Logigramme}

 \begin{tikzpicture}

\node (x) at (0, 4*1.5) {$A$};
            \node (y) at (0.5, 4*1.5) {$B$};
            \node (z) at (1, 4*1.5) {$C$};
            \node (w) at (1.5, 4*1.5) {$D$};
            \node[not gate US, draw, rotate=270] at ($(x) + (0.25, -0.3)$) (notx) {};
            \draw (x) -- (notx.input); 
            \node[not gate US, draw, rotate=270] at ($(y) + (0.25, -0.3)$) (noty) {};
            \draw (y) -- (noty.input); 
            \node[not gate US, draw, rotate=270] at ($(z) + (0.25, -0.3)$) (notz) {};
            \draw (z) -- (notz.input);
            \node[not gate US, draw, rotate=270] at ($(w) + (0.25, -0.3)$) (notw) {};
            \draw (w) -- (notw.input);
                
           
            \node[and gate US, draw, rotate=0, logic gate inputs=nnnn] at (2.5, 0*1.5) (xandy0) {};
            \draw (xandy0.output) -- node[above]{\scriptsize $D$} ($(xandy0) + (1.8, 0)$);
                %%W
            \draw (w) -| ($(w) + (0, 0)$) |- (xandy0.input 4);
                
           
            \node[and gate US, draw, rotate=0, logic gate inputs=nnnn] at (2.5, 1*1.5) (xandy1) {};
            \draw (xandy1.output) -- node[above]{\scriptsize $\bar A$} ($(xandy1) + (1.8, 0)$);
            
            %% X'

            \draw  [line width=0.25mm,   red] (notx.output) -- ([xshift=0cm]notx.output) |- (xandy1.input 1);
                    
           
            \node[and gate US, draw, rotate=0, logic gate inputs=nnnn] at (2.5, 2*1.5) (xandy2) {};
            \draw (xandy2.output) -- node[above]{\scriptsize $B.\bar C$} ($(xandy2) + (1.8, 0)$);
              
            %% Y
            \draw (y) -| ($(y) + (0, 0)$) |- (xandy2.input 2);
        
            %%Z'

            \draw [line width=0.25mm,   red] (notz.output) -- ([xshift=0cm]notz.output) |- (xandy2.input 3);
                  
           
            \node[and gate US, draw, rotate=0, logic gate inputs=nnnn] at (2.5, 3*1.5) (xandy3) {};
            \draw (xandy3.output) -- node[above]{\scriptsize $\bar B.C$} ($(xandy3) + (1.8, 0)$);
            
            %Y'

            \draw [line width=0.25mm,   red] (noty.output) -- ([xshift=0cm]noty.output) |- (xandy3.input 2);
          
            %%Z
            \draw (z) -| ($(z) + (0, 0)$) |- (xandy3.input 3);
        \node[or gate US, draw, rotate=0, logic gate inputs=nnnnn] at (5.5, 4*0.5) (xory) {};


                    \draw (xory.output) -- node[above]{\scriptsize$F$} ($(xory) + (1, 0)$);

\draw (xandy0.output) -- ([xshift=1.40cm]xandy0.output) |- (xory.input 4);

\draw (xandy1.output) -- ([xshift=1.35cm]xandy1.output) |- (xory.input 3);

\draw (xandy2.output) -- ([xshift=1.30cm]xandy2.output) |- (xory.input 2);

\draw (xandy3.output) -- ([xshift=1.35cm]xandy3.output) |- (xory.input 1);

 \end{tikzpicture}


\paragraph{Q2}

Simplifier les fonctions suivantes
table 1

Simplified Sum of products : $a.b.d+\bar b.\bar c+a.c.\bar d+\bar a.c.d$

table 2

Simplified Sum of products : $b.\bar d+\bar a.\bar d+\bar c.\bar d+a.b.\bar c+\bar a.\bar b.\bar c$

table 3

Simplified Sum of products : $b.\bar c.\bar d+\bar a.b.c.d+\bar a.\bar c.\bar d$

\pagebreak
\section{Question}

\paragraph{Q1}

Etudier la fonction suivante

f(a,b,c,d)= $a.b+\bar a.\bar b.\bar c.d + a.c.\bar d+\bar a.c.d+a.b.\bar c.d+\bar a.\bar b.\bar d+\bar a.\bar c.\bar d$

\paragraph{Q2}

Simplifier les fonctions suivantes

\begin{karnaugh-map}[4][4][1][cd][ab]
          \minterms{0, 1, 3, 4, 5, 7, 8, 10, 12, 14, 15}
          \maxterms{2, 6, 9, 11, 13}
        %\autoterms[0]
          %\implicant{5}{15}
          %\implicantedge{8}{8}{10}{10}
          %\implicantedge{8}{8}{10}{10}[8,10]
        \end{karnaugh-map}\begin{karnaugh-map}[4][4][1][cd][ab]
          \minterms{0, 1, 4, 8, 9, 10}
          \maxterms{2, 3, 5, 6, 7, 11, 12, 13, 14, 15}
        %\autoterms[0]
          %\implicant{5}{15}
          %\implicantedge{8}{8}{10}{10}
          %\implicantedge{8}{8}{10}{10}[8,10]
        \end{karnaugh-map}\begin{karnaugh-map}[4][4][1][cd][ab]
          \minterms{0, 4, 5, 6, 9, 10, 12, 15}
          \maxterms{1, 2, 3, 7, 8, 11, 13, 14}
        %\autoterms[0]
          %\implicant{5}{15}
          %\implicantedge{8}{8}{10}{10}
          %\implicantedge{8}{8}{10}{10}[8,10]
        \end{karnaugh-map}
\hrule width 1\linewidth
\paragraph{Q1}

Etudier la fonction suivante

f(a,b,c,d)= $a.b+\bar a.\bar b.\bar c.d + a.c.\bar d+\bar a.c.d+a.b.\bar c.d+\bar a.\bar b.\bar d+\bar a.\bar c.\bar d$

\paragraph{Q2}

Simplifier les fonctions suivantes

\begin{karnaugh-map}[4][4][1][cd][ab]
          \minterms{0, 1, 3, 4, 5, 7, 8, 10, 12, 14, 15}
          \maxterms{2, 6, 9, 11, 13}
        %\autoterms[0]
          %\implicant{5}{15}
          %\implicantedge{8}{8}{10}{10}
          %\implicantedge{8}{8}{10}{10}[8,10]
        \end{karnaugh-map}\begin{karnaugh-map}[4][4][1][cd][ab]
          \minterms{0, 1, 4, 8, 9, 10}
          \maxterms{2, 3, 5, 6, 7, 11, 12, 13, 14, 15}
        %\autoterms[0]
          %\implicant{5}{15}
          %\implicantedge{8}{8}{10}{10}
          %\implicantedge{8}{8}{10}{10}[8,10]
        \end{karnaugh-map}\begin{karnaugh-map}[4][4][1][cd][ab]
          \minterms{0, 4, 5, 6, 9, 10, 12, 15}
          \maxterms{1, 2, 3, 7, 8, 11, 13, 14}
        %\autoterms[0]
          %\implicant{5}{15}
          %\implicantedge{8}{8}{10}{10}
          %\implicantedge{8}{8}{10}{10}[8,10]
        \end{karnaugh-map}
\hrule width 1\linewidth
\paragraph{Q1}

Etudier la fonction suivante

f(a,b,c,d)= $a.b+\bar a.\bar b.\bar c.d + a.c.\bar d+\bar a.c.d+a.b.\bar c.d+\bar a.\bar b.\bar d+\bar a.\bar c.\bar d$

\paragraph{Q2}

Simplifier les fonctions suivantes

\begin{karnaugh-map}[4][4][1][cd][ab]
          \minterms{0, 1, 3, 4, 5, 7, 8, 10, 12, 14, 15}
          \maxterms{2, 6, 9, 11, 13}
        %\autoterms[0]
          %\implicant{5}{15}
          %\implicantedge{8}{8}{10}{10}
          %\implicantedge{8}{8}{10}{10}[8,10]
        \end{karnaugh-map}\begin{karnaugh-map}[4][4][1][cd][ab]
          \minterms{0, 1, 4, 8, 9, 10}
          \maxterms{2, 3, 5, 6, 7, 11, 12, 13, 14, 15}
        %\autoterms[0]
          %\implicant{5}{15}
          %\implicantedge{8}{8}{10}{10}
          %\implicantedge{8}{8}{10}{10}[8,10]
        \end{karnaugh-map}\begin{karnaugh-map}[4][4][1][cd][ab]
          \minterms{0, 4, 5, 6, 9, 10, 12, 15}
          \maxterms{1, 2, 3, 7, 8, 11, 13, 14}
        %\autoterms[0]
          %\implicant{5}{15}
          %\implicantedge{8}{8}{10}{10}
          %\implicantedge{8}{8}{10}{10}[8,10]
        \end{karnaugh-map}
\hrule width 1\linewidth
\paragraph{Q1}

Etudier la fonction suivante

f(a,b,c,d)= $a.b+\bar a.\bar b.\bar c.d + a.c.\bar d+\bar a.c.d+a.b.\bar c.d+\bar a.\bar b.\bar d+\bar a.\bar c.\bar d$

\paragraph{Q2}

Simplifier les fonctions suivantes

\begin{karnaugh-map}[4][4][1][cd][ab]
          \minterms{0, 1, 3, 4, 5, 7, 8, 10, 12, 14, 15}
          \maxterms{2, 6, 9, 11, 13}
        %\autoterms[0]
          %\implicant{5}{15}
          %\implicantedge{8}{8}{10}{10}
          %\implicantedge{8}{8}{10}{10}[8,10]
        \end{karnaugh-map}\begin{karnaugh-map}[4][4][1][cd][ab]
          \minterms{0, 1, 4, 8, 9, 10}
          \maxterms{2, 3, 5, 6, 7, 11, 12, 13, 14, 15}
        %\autoterms[0]
          %\implicant{5}{15}
          %\implicantedge{8}{8}{10}{10}
          %\implicantedge{8}{8}{10}{10}[8,10]
        \end{karnaugh-map}\begin{karnaugh-map}[4][4][1][cd][ab]
          \minterms{0, 4, 5, 6, 9, 10, 12, 15}
          \maxterms{1, 2, 3, 7, 8, 11, 13, 14}
        %\autoterms[0]
          %\implicant{5}{15}
          %\implicantedge{8}{8}{10}{10}
          %\implicantedge{8}{8}{10}{10}[8,10]
        \end{karnaugh-map}
\hrule width 1\linewidth
\paragraph{Q1}

Etudier la fonction suivante

f(a,b,c,d)= $a.b+\bar a.\bar b.\bar c.d + a.c.\bar d+\bar a.c.d+a.b.\bar c.d+\bar a.\bar b.\bar d+\bar a.\bar c.\bar d$

\paragraph{Q2}

Simplifier les fonctions suivantes

\begin{karnaugh-map}[4][4][1][cd][ab]
          \minterms{0, 1, 3, 4, 5, 7, 8, 10, 12, 14, 15}
          \maxterms{2, 6, 9, 11, 13}
        %\autoterms[0]
          %\implicant{5}{15}
          %\implicantedge{8}{8}{10}{10}
          %\implicantedge{8}{8}{10}{10}[8,10]
        \end{karnaugh-map}\begin{karnaugh-map}[4][4][1][cd][ab]
          \minterms{0, 1, 4, 8, 9, 10}
          \maxterms{2, 3, 5, 6, 7, 11, 12, 13, 14, 15}
        %\autoterms[0]
          %\implicant{5}{15}
          %\implicantedge{8}{8}{10}{10}
          %\implicantedge{8}{8}{10}{10}[8,10]
        \end{karnaugh-map}\begin{karnaugh-map}[4][4][1][cd][ab]
          \minterms{0, 4, 5, 6, 9, 10, 12, 15}
          \maxterms{1, 2, 3, 7, 8, 11, 13, 14}
        %\autoterms[0]
          %\implicant{5}{15}
          %\implicantedge{8}{8}{10}{10}
          %\implicantedge{8}{8}{10}{10}[8,10]
        \end{karnaugh-map}
\hrule width 1\linewidth\pagebreak
\subsection{Correction}

\paragraph{Q1}

f(a,b,c,d)=$a.b+\bar a.\bar b.\bar c.d + a.c.\bar d+\bar a.c.d+a.b.\bar c.d+\bar a.\bar b.\bar d+\bar a.\bar c.\bar d$
f(a,b,c,D)=$ \sum a.b+\bar a.\bar b.\bar c.d + a.c.\bar d+\bar a.c.d+a.b.\bar c.d+\bar a.\bar b.\bar d+\bar a.\bar c.\bar d $ 

%%\begin{table}
        \begin{tabular}{|c|c|c|c|c||c|}
    \toprule
         & A & B & C & D & F\\ \midrule0 & 0 & 0 & 0 & 0 & 1\\1 & 0 & 0 & 0 & 1 & 1\\2 & 0 & 0 & 1 & 0 & 1\\3 & 0 & 0 & 1 & 1 & 1\\\midrule4 & 0 & 1 & 0 & 0 & 1\\5 & 0 & 1 & 0 & 1 & 0\\6 & 0 & 1 & 1 & 0 & 0\\7 & 0 & 1 & 1 & 1 & 1\\\midrule8 & 1 & 0 & 0 & 0 & 0\\9 & 1 & 0 & 0 & 1 & 0\\10 & 1 & 0 & 1 & 0 & 1\\11 & 1 & 0 & 1 & 1 & 0\\\midrule12 & 1 & 1 & 0 & 0 & 1\\13 & 1 & 1 & 0 & 1 & 1\\14 & 1 & 1 & 1 & 0 & 1\\15 & 1 & 1 & 1 & 1 & 1\\\bottomrule
        \end{tabular}
        %%\end{table}
        
Sum of products 
 f(a,b,c,d) = $\bar a.\bar b.\bar c.\bar d + \bar a.\bar b.\bar c.d + \bar a.\bar b.c.\bar d + \bar a.\bar b.\bar c.d + \bar a.b.\bar c.\bar d + \bar a.\bar b.\bar c.d + \bar a.\bar b.c.\bar d + \bar a.b.\bar c.\bar d + \bar a.\bar b.\bar c.d + \bar a.\bar b.c.\bar d + \bar a.\bar b.\bar c.d$

Product of sums 
 f(a,b,c,d) = $\bar a.\bar b.\bar c.\bar d.\bar a.\bar b.\bar c.d.\bar a.b.c.d.\bar a.\bar b.\bar c.\bar d.\bar a.\bar b.\bar c.\bar d$

Karnough map\todo{fix map}
\begin{karnaugh-map}[4][4][1][cd][ab]
          \minterms{0, 1, 2, 3, 4, 7, 10, 12, 13, 14, 15}
          \maxterms{5, 6, 8, 9, 11}
        %\autoterms[0]
          %\implicant{5}{15}
          %\implicantedge{8}{8}{10}{10}
          %\implicantedge{8}{8}{10}{10}[8,10]
        \end{karnaugh-map}

Simplified Sum of products: $a.b+\bar a.\bar b+\bar a.c.d+\bar b.c.\bar d+\bar a.\bar c.\bar d$

Simplified Product of sums: $(b+\bar a+\bar a).(b+\bar b+\bar b+c+c+\bar c)+d+\bar d).(a+\bar d).(a$
\paragraph{Logigramme} de la fonction\\
        %%\missingfigure[figwidth=6cm]{Logigramme}

 \begin{tikzpicture}

\node (x) at (0, 5*1.5) {$A$};
            \node (y) at (0.5, 5*1.5) {$B$};
            \node (z) at (1, 5*1.5) {$C$};
            \node (w) at (1.5, 5*1.5) {$D$};
            \node[not gate US, draw, rotate=270] at ($(x) + (0.25, -0.3)$) (notx) {};
            \draw (x) -- (notx.input); 
            \node[not gate US, draw, rotate=270] at ($(y) + (0.25, -0.3)$) (noty) {};
            \draw (y) -- (noty.input); 
            \node[not gate US, draw, rotate=270] at ($(z) + (0.25, -0.3)$) (notz) {};
            \draw (z) -- (notz.input);
            \node[not gate US, draw, rotate=270] at ($(w) + (0.25, -0.3)$) (notw) {};
            \draw (w) -- (notw.input);
                
           
            \node[and gate US, draw, rotate=0, logic gate inputs=nnnn] at (2.5, 0*1.5) (xandy0) {};
            \draw (xandy0.output) -- node[above]{\scriptsize $A.B$} ($(xandy0) + (1.8, 0)$);
            %% X
            \draw (x) -| ($(x) + (0, 0)$) |- (xandy0.input 1);
              
            %% Y
            \draw (y) -| ($(y) + (0, 0)$) |- (xandy0.input 2);
                
           
            \node[and gate US, draw, rotate=0, logic gate inputs=nnnn] at (2.5, 1*1.5) (xandy1) {};
            \draw (xandy1.output) -- node[above]{\scriptsize $\bar A.\bar B$} ($(xandy1) + (1.8, 0)$);
            
            %% X'

            \draw  [line width=0.25mm,   red] (notx.output) -- ([xshift=0cm]notx.output) |- (xandy1.input 1);
            
            %Y'

            \draw [line width=0.25mm,   red] (noty.output) -- ([xshift=0cm]noty.output) |- (xandy1.input 2);
                  
           
            \node[and gate US, draw, rotate=0, logic gate inputs=nnnn] at (2.5, 2*1.5) (xandy2) {};
            \draw (xandy2.output) -- node[above]{\scriptsize $\bar A.C.D$} ($(xandy2) + (1.8, 0)$);
            
            %% X'

            \draw  [line width=0.25mm,   red] (notx.output) -- ([xshift=0cm]notx.output) |- (xandy2.input 1);
            
            %%Z
            \draw (z) -| ($(z) + (0, 0)$) |- (xandy2.input 3);
            %%W
            \draw (w) -| ($(w) + (0, 0)$) |- (xandy2.input 4);
                
           
            \node[and gate US, draw, rotate=0, logic gate inputs=nnnn] at (2.5, 3*1.5) (xandy3) {};
            \draw (xandy3.output) -- node[above]{\scriptsize $\bar B.C.\bar D$} ($(xandy3) + (1.8, 0)$);
            
            %Y'

            \draw [line width=0.25mm,   red] (noty.output) -- ([xshift=0cm]noty.output) |- (xandy3.input 2);
          
            %%Z
            \draw (z) -| ($(z) + (0, 0)$) |- (xandy3.input 3);
        
            %%W

            \draw [line width=0.25mm,   red] (notw.output) -- ([xshift=0cm]notw.output) |- (xandy3.input 4);
                  
           
            \node[and gate US, draw, rotate=0, logic gate inputs=nnnn] at (2.5, 4*1.5) (xandy4) {};
            \draw (xandy4.output) -- node[above]{\scriptsize $\bar A.\bar C.\bar D$} ($(xandy4) + (1.8, 0)$);
            
            %% X'

            \draw  [line width=0.25mm,   red] (notx.output) -- ([xshift=0cm]notx.output) |- (xandy4.input 1);
            
            %%Z'

            \draw [line width=0.25mm,   red] (notz.output) -- ([xshift=0cm]notz.output) |- (xandy4.input 3);
          
            %%W

            \draw [line width=0.25mm,   red] (notw.output) -- ([xshift=0cm]notw.output) |- (xandy4.input 4);
          \node[or gate US, draw, rotate=0, logic gate inputs=nnnnnn] at (5.5, 5*0.5) (xory) {};


                    \draw (xory.output) -- node[above]{\scriptsize$F$} ($(xory) + (1, 0)$);

\draw (xandy0.output) -- ([xshift=1.40cm]xandy0.output) |- (xory.input 5);

\draw (xandy1.output) -- ([xshift=1.35cm]xandy1.output) |- (xory.input 4);

\draw (xandy2.output) -- ([xshift=1.30cm]xandy2.output) |- (xory.input 3);

\draw (xandy3.output) -- ([xshift=1.25cm]xandy3.output) |- (xory.input 2);

\draw (xandy4.output) -- ([xshift=1.30cm]xandy4.output) |- (xory.input 1);

 \end{tikzpicture}


\paragraph{Q2}

Simplifier les fonctions suivantes
table 1

Simplified Sum of products : $a.\bar d+\bar a.d+b.c.d+\bar a.\bar c$

table 2

Simplified Sum of products : $\bar b.\bar c+a.\bar b.\bar d+\bar a.\bar c.\bar d$

table 3

Simplified Sum of products : $a.b.c.d+\bar a.b.\bar c+\bar a.b.\bar d+b.\bar c.\bar d+\bar a.\bar c.\bar d+a.\bar b.c.\bar d+a.\bar b.\bar c.d$

\pagebreak
\section{Question}

\paragraph{Q1}

Etudier la fonction suivante

f(a,b,c,d)= $a.b.c+\bar b.\bar c+a.\bar c.d+\bar a.c.d + \bar a.c+a.b.d+b.c.d+a.b.\bar c+a.\bar c.d+\bar a.b.\bar d+b.\bar c.\bar d+\bar b.c.\bar d+\bar a.\bar b.d+\bar b.\bar c.d$

\paragraph{Q2}

Simplifier l'expression suivante par les proprietés algébreiques 

S = $a.b.c.d+\bar a.b.\bar c+\bar a.\bar b.\bar d+a.\bar b.\bar c.d + \bar a.b.d+\bar a.\bar b.\bar c+a.b.\bar c.\bar d+a.\bar b.c.\bar d + a.\bar c.d+\bar b.c.d+\bar a.\bar b.c+a.b.c.\bar d+\bar a.\bar c.\bar d + a.b.\bar c.\bar d+\bar a.b.c.\bar d+\bar a.\bar b.\bar c.d$

\hrule width 1\linewidth
\paragraph{Q1}

Etudier la fonction suivante

f(a,b,c,d)= $a.b.c+\bar b.\bar c+a.\bar c.d+\bar a.c.d + \bar a.c+a.b.d+b.c.d+a.b.\bar c+a.\bar c.d+\bar a.b.\bar d+b.\bar c.\bar d+\bar b.c.\bar d+\bar a.\bar b.d+\bar b.\bar c.d$

\paragraph{Q2}

Simplifier l'expression suivante par les proprietés algébreiques 

S = $a.b.c.d+\bar a.b.\bar c+\bar a.\bar b.\bar d+a.\bar b.\bar c.d + \bar a.b.d+\bar a.\bar b.\bar c+a.b.\bar c.\bar d+a.\bar b.c.\bar d + a.\bar c.d+\bar b.c.d+\bar a.\bar b.c+a.b.c.\bar d+\bar a.\bar c.\bar d + a.b.\bar c.\bar d+\bar a.b.c.\bar d+\bar a.\bar b.\bar c.d$

\hrule width 1\linewidth
\paragraph{Q1}

Etudier la fonction suivante

f(a,b,c,d)= $a.b.c+\bar b.\bar c+a.\bar c.d+\bar a.c.d + \bar a.c+a.b.d+b.c.d+a.b.\bar c+a.\bar c.d+\bar a.b.\bar d+b.\bar c.\bar d+\bar b.c.\bar d+\bar a.\bar b.d+\bar b.\bar c.d$

\paragraph{Q2}

Simplifier l'expression suivante par les proprietés algébreiques 

S = $a.b.c.d+\bar a.b.\bar c+\bar a.\bar b.\bar d+a.\bar b.\bar c.d + \bar a.b.d+\bar a.\bar b.\bar c+a.b.\bar c.\bar d+a.\bar b.c.\bar d + a.\bar c.d+\bar b.c.d+\bar a.\bar b.c+a.b.c.\bar d+\bar a.\bar c.\bar d + a.b.\bar c.\bar d+\bar a.b.c.\bar d+\bar a.\bar b.\bar c.d$

\hrule width 1\linewidth
\paragraph{Q1}

Etudier la fonction suivante

f(a,b,c,d)= $a.b.c+\bar b.\bar c+a.\bar c.d+\bar a.c.d + \bar a.c+a.b.d+b.c.d+a.b.\bar c+a.\bar c.d+\bar a.b.\bar d+b.\bar c.\bar d+\bar b.c.\bar d+\bar a.\bar b.d+\bar b.\bar c.d$

\paragraph{Q2}

Simplifier l'expression suivante par les proprietés algébreiques 

S = $a.b.c.d+\bar a.b.\bar c+\bar a.\bar b.\bar d+a.\bar b.\bar c.d + \bar a.b.d+\bar a.\bar b.\bar c+a.b.\bar c.\bar d+a.\bar b.c.\bar d + a.\bar c.d+\bar b.c.d+\bar a.\bar b.c+a.b.c.\bar d+\bar a.\bar c.\bar d + a.b.\bar c.\bar d+\bar a.b.c.\bar d+\bar a.\bar b.\bar c.d$

\hrule width 1\linewidth
\paragraph{Q1}

Etudier la fonction suivante

f(a,b,c,d)= $a.b.c+\bar b.\bar c+a.\bar c.d+\bar a.c.d + \bar a.c+a.b.d+b.c.d+a.b.\bar c+a.\bar c.d+\bar a.b.\bar d+b.\bar c.\bar d+\bar b.c.\bar d+\bar a.\bar b.d+\bar b.\bar c.d$

\paragraph{Q2}

Simplifier l'expression suivante par les proprietés algébreiques 

S = $a.b.c.d+\bar a.b.\bar c+\bar a.\bar b.\bar d+a.\bar b.\bar c.d + \bar a.b.d+\bar a.\bar b.\bar c+a.b.\bar c.\bar d+a.\bar b.c.\bar d + a.\bar c.d+\bar b.c.d+\bar a.\bar b.c+a.b.c.\bar d+\bar a.\bar c.\bar d + a.b.\bar c.\bar d+\bar a.b.c.\bar d+\bar a.\bar b.\bar c.d$

\hrule width 1\linewidth\pagebreak
\subsection{Correction}

\paragraph{Q1}

f(a,b,c,d)=$a.b.c+\bar b.\bar c+a.\bar c.d+\bar a.c.d + \bar a.c+a.b.d+b.c.d+a.b.\bar c+a.\bar c.d+\bar a.b.\bar d+b.\bar c.\bar d+\bar b.c.\bar d+\bar a.\bar b.d+\bar b.\bar c.d$
f(a,b,c,D)=$ \sum a.b.c+\bar b.\bar c+a.\bar c.d+\bar a.c.d + \bar a.c+a.b.d+b.c.d+a.b.\bar c+a.\bar c.d+\bar a.b.\bar d+b.\bar c.\bar d+\bar b.c.\bar d+\bar a.\bar b.d+\bar b.\bar c.d $ 

%%\begin{table}
        \begin{tabular}{|c|c|c|c|c||c|}
    \toprule
         & A & B & C & D & F\\ \midrule0 & 0 & 0 & 0 & 0 & 1\\1 & 0 & 0 & 0 & 1 & 1\\2 & 0 & 0 & 1 & 0 & 1\\3 & 0 & 0 & 1 & 1 & 1\\\midrule4 & 0 & 1 & 0 & 0 & 1\\5 & 0 & 1 & 0 & 1 & 0\\6 & 0 & 1 & 1 & 0 & 1\\7 & 0 & 1 & 1 & 1 & 1\\\midrule8 & 1 & 0 & 0 & 0 & 1\\9 & 1 & 0 & 0 & 1 & 1\\10 & 1 & 0 & 1 & 0 & 1\\11 & 1 & 0 & 1 & 1 & 0\\\midrule12 & 1 & 1 & 0 & 0 & 1\\13 & 1 & 1 & 0 & 1 & 1\\14 & 1 & 1 & 1 & 0 & 1\\15 & 1 & 1 & 1 & 1 & 1\\\bottomrule
        \end{tabular}
        %%\end{table}
        
Sum of products 
 f(a,b,c,d) = $\bar a.\bar b.\bar c.\bar d + \bar a.\bar b.\bar c.d + \bar a.\bar b.c.\bar d + \bar a.\bar b.\bar c.d + \bar a.b.\bar c.\bar d + \bar a.\bar b.c.\bar d + \bar a.\bar b.\bar c.d + a.\bar b.\bar c.\bar d + \bar a.\bar b.\bar c.d + \bar a.\bar b.c.\bar d + \bar a.b.\bar c.\bar d + \bar a.\bar b.\bar c.d + \bar a.\bar b.c.\bar d + \bar a.\bar b.\bar c.d$

Product of sums 
 f(a,b,c,d) = $\bar a.\bar b.\bar c.\bar d.\bar a.\bar b.\bar c.\bar d$

Karnough map\todo{fix map}
\begin{karnaugh-map}[4][4][1][cd][ab]
          \minterms{0, 1, 2, 3, 4, 6, 7, 8, 9, 10, 12, 13, 14, 15}
          \maxterms{5, 11}
        %\autoterms[0]
          %\implicant{5}{15}
          %\implicantedge{8}{8}{10}{10}
          %\implicantedge{8}{8}{10}{10}[8,10]
        \end{karnaugh-map}

Simplified Sum of products: $\bar d+a.b+b.c+a.\bar c+\bar a.c+\bar a.\bar b+\bar b.\bar c$

Simplified Product of sums: $(a+\bar a+\bar b+c+\bar c+\bar d)+\bar d).(b$
\paragraph{Logigramme} de la fonction\\
        %%\missingfigure[figwidth=6cm]{Logigramme}

 \begin{tikzpicture}

\node (x) at (0, 7*1.5) {$A$};
            \node (y) at (0.5, 7*1.5) {$B$};
            \node (z) at (1, 7*1.5) {$C$};
            \node (w) at (1.5, 7*1.5) {$D$};
            \node[not gate US, draw, rotate=270] at ($(x) + (0.25, -0.3)$) (notx) {};
            \draw (x) -- (notx.input); 
            \node[not gate US, draw, rotate=270] at ($(y) + (0.25, -0.3)$) (noty) {};
            \draw (y) -- (noty.input); 
            \node[not gate US, draw, rotate=270] at ($(z) + (0.25, -0.3)$) (notz) {};
            \draw (z) -- (notz.input);
            \node[not gate US, draw, rotate=270] at ($(w) + (0.25, -0.3)$) (notw) {};
            \draw (w) -- (notw.input);
                
           
            \node[and gate US, draw, rotate=0, logic gate inputs=nnnn] at (2.5, 0*1.5) (xandy0) {};
            \draw (xandy0.output) -- node[above]{\scriptsize $\bar D$} ($(xandy0) + (1.8, 0)$);
            
            %%W

            \draw [line width=0.25mm,   red] (notw.output) -- ([xshift=0cm]notw.output) |- (xandy0.input 4);
                  
           
            \node[and gate US, draw, rotate=0, logic gate inputs=nnnn] at (2.5, 1*1.5) (xandy1) {};
            \draw (xandy1.output) -- node[above]{\scriptsize $A.B$} ($(xandy1) + (1.8, 0)$);
            %% X
            \draw (x) -| ($(x) + (0, 0)$) |- (xandy1.input 1);
              
            %% Y
            \draw (y) -| ($(y) + (0, 0)$) |- (xandy1.input 2);
                
           
            \node[and gate US, draw, rotate=0, logic gate inputs=nnnn] at (2.5, 2*1.5) (xandy2) {};
            \draw (xandy2.output) -- node[above]{\scriptsize $B.C$} ($(xandy2) + (1.8, 0)$);
              
            %% Y
            \draw (y) -| ($(y) + (0, 0)$) |- (xandy2.input 2);
        
            %%Z
            \draw (z) -| ($(z) + (0, 0)$) |- (xandy2.input 3);
                
           
            \node[and gate US, draw, rotate=0, logic gate inputs=nnnn] at (2.5, 3*1.5) (xandy3) {};
            \draw (xandy3.output) -- node[above]{\scriptsize $A.\bar C$} ($(xandy3) + (1.8, 0)$);
            %% X
            \draw (x) -| ($(x) + (0, 0)$) |- (xandy3.input 1);
            
            %%Z'

            \draw [line width=0.25mm,   red] (notz.output) -- ([xshift=0cm]notz.output) |- (xandy3.input 3);
                  
           
            \node[and gate US, draw, rotate=0, logic gate inputs=nnnn] at (2.5, 4*1.5) (xandy4) {};
            \draw (xandy4.output) -- node[above]{\scriptsize $\bar A.C$} ($(xandy4) + (1.8, 0)$);
            
            %% X'

            \draw  [line width=0.25mm,   red] (notx.output) -- ([xshift=0cm]notx.output) |- (xandy4.input 1);
            
            %%Z
            \draw (z) -| ($(z) + (0, 0)$) |- (xandy4.input 3);
                
           
            \node[and gate US, draw, rotate=0, logic gate inputs=nnnn] at (2.5, 5*1.5) (xandy5) {};
            \draw (xandy5.output) -- node[above]{\scriptsize $\bar A.\bar B$} ($(xandy5) + (1.8, 0)$);
            
            %% X'

            \draw  [line width=0.25mm,   red] (notx.output) -- ([xshift=0cm]notx.output) |- (xandy5.input 1);
            
            %Y'

            \draw [line width=0.25mm,   red] (noty.output) -- ([xshift=0cm]noty.output) |- (xandy5.input 2);
                  
           
            \node[and gate US, draw, rotate=0, logic gate inputs=nnnn] at (2.5, 6*1.5) (xandy6) {};
            \draw (xandy6.output) -- node[above]{\scriptsize $\bar B.\bar C$} ($(xandy6) + (1.8, 0)$);
            
            %Y'

            \draw [line width=0.25mm,   red] (noty.output) -- ([xshift=0cm]noty.output) |- (xandy6.input 2);
          
            %%Z'

            \draw [line width=0.25mm,   red] (notz.output) -- ([xshift=0cm]notz.output) |- (xandy6.input 3);
          \node[or gate US, draw, rotate=0, logic gate inputs=nnnnnnnn] at (5.5, 7*0.5) (xory) {};


                    \draw (xory.output) -- node[above]{\scriptsize$F$} ($(xory) + (1, 0)$);

\draw (xandy0.output) -- ([xshift=1.40cm]xandy0.output) |- (xory.input 7);

\draw (xandy1.output) -- ([xshift=1.35cm]xandy1.output) |- (xory.input 6);

\draw (xandy2.output) -- ([xshift=1.30cm]xandy2.output) |- (xory.input 5);

\draw (xandy3.output) -- ([xshift=1.25cm]xandy3.output) |- (xory.input 4);

\draw (xandy4.output) -- ([xshift=1.20cm]xandy4.output) |- (xory.input 3);

\draw (xandy5.output) -- ([xshift=1.25cm]xandy5.output) |- (xory.input 2);

\draw (xandy6.output) -- ([xshift=1.30cm]xandy6.output) |- (xory.input 1);

 \end{tikzpicture}


\paragraph{Q2}

Simplifier l'expression suivante

S = $a.b.c.d+\bar a.b.\bar c+\bar a.\bar b.\bar d+a.\bar b.\bar c.d + \bar a.b.d+\bar a.\bar b.\bar c+a.b.\bar c.\bar d+a.\bar b.c.\bar d + a.\bar c.d+\bar b.c.d+\bar a.\bar b.c+a.b.c.\bar d+\bar a.\bar c.\bar d + a.b.\bar c.\bar d+\bar a.b.c.\bar d+\bar a.\bar b.\bar c.d$

 = $b+c+d+\bar a$


Karnough map\todo{fix map}
\begin{karnaugh-map}[4][4][1][cd][ab]
          \minterms{0, 1, 2, 3, 4, 5, 6, 7, 9, 10, 11, 12, 13, 14, 15}
          \maxterms{8}
        %\autoterms[0]
          %\implicant{5}{15}
          %\implicantedge{8}{8}{10}{10}
          %\implicantedge{8}{8}{10}{10}[8,10]
        \end{karnaugh-map}

\pagebreak
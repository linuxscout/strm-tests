
\section{Question}


\paragraph{Q1}

Representer sous la norme IEEE-754 32 bits le nombre suivant

\begin{arab}[utf]
مثل العدد الآتي حسب المعيار IEEE-754 32 bits
\end{arab}

151.67



\hrule width 1\linewidth
\pagebreak

\subsection{Correction}


\paragraph{Q1}

Representer sous la norme IEEE-754 32 bits le nombre suivant

\begin{arab}[utf]
مثل العدد الآتي حسب المعيار IEEE-754 32 bits
\end{arab}

151.67
$1001\,0111.1010\,1011\,1011\,0011\,0011\,0011\,0011\,00$

\begin{itemize}
  \item \textbf{Input:} $151.6700 =  1001\,0111.1010\,1011\,1011\,0011\,0011\,0011\,0011\,00 $
  \item \textbf{Normalized form:} $1.001\,0111\,1010\,1011\,1011\,0011 \times 2^7$
  \item \textbf{Sign bit:} + $\Rightarrow$ 0
  \item \textbf{Exponent:} $7 + 127 = 134 \Rightarrow 1000\,0110$
  \item \textbf{Pseudo-mantissa:} $001\,0111\,1010\,1011\,1011\,0011$
  \item \textbf{Final binary representation:}  $0100\,0011\,0001\,0111\,1010\,1011\,1011\,0011 $
  \item \textbf{Hexadecimal form:} $ 4317\,ABB3 $
  \end{itemize}

\pagebreak

\paragraph{Q1}

Donner les intervalles qu'on peut représenter en nombre positifs, valeur absolue, complément à 1 et complément à 2  sur 46 bits

Give the intervals which can be represented in posiitve numbers, absolute value, 1's complement and 2's complement on 46 bits

\begin{arab}[utf]
حدد المجالات التي يمكن تمثيلها لأعداد الموجبة والتمثيل بالقيمة المطلقة والمتمم إلى 1 و 2 على   : 46 بت
\end{arab}




\hrule width 1\linewidth
\pagebreak

\subsection{Correction}


\paragraph{Q1}

Donner les intervalles qu'on peut représenter en nombre positifs, valeur absolue, complément à 1 et complément à 2  sur 46 bits

Give the intervals which can be represented in posiitve numbers, absolute value, 1's complement and 2's complement on 46 bits

\begin{arab}[utf]
حدد المجالات التي يمكن تمثيلها لأعداد الموجبة والتمثيل بالقيمة المطلقة والمتمم إلى 1 و 2 على   : 46 بت
\end{arab}

\begin{itemize}
\item \textbf{Positifs}: $[0; 2^{ 46-1 }] = [0; 70368744177663]$
\item \textbf{Unsigned value} $[-(2^{ 45 }-1 );2^{ 45 }-1] = [-35184372088831, 35184372088831]$
\item \textbf{One's compelement} $[-(2^{ 45 }-1 );2^{ 45 }-1] = [-35184372088831, 35184372088831]$
\item \textbf{Two's compelement} $[-2^{ 45 } ;2^{ 45 }-1] = [-35184372088832, 35184372088831]$
\end{itemize}

\pagebreak

\paragraph{Q1}

Represent in 1's and 2's complement, the following number

Representer en complément à 1 et à 2 le nombre suivant :

\begin{arab}[utf]
مثل  العدد الآتي في المتمم إلى الواحد وإلى الاثنين  :
\end{arab}

$108$




\hrule width 1\linewidth
\pagebreak

\subsection{Correction}


\paragraph{Q1}

Represent in 1's and 2's complement, the following number

Representer en complément à 1 et à 2 le nombre suivant :

\begin{arab}[utf]
مثل  العدد الآتي في المتمم إلى الواحد وإلى الاثنين  :
\end{arab}

$108$

\begin{itemize}
\item $-108 = ( -110\,1100 )_{2}$
\item $( 1001\,0011 )_{c1}$
\item +1
\item $( 1001\,0100 )_{c2}$
\end{itemize}

\pagebreak

\paragraph{Q1}


Simplifier l'expression suivant.


Simpilfy the following expression
\begin{arab}[utf]
بسّط العبارة الآتية
\end{arab}

$S = \overline{a}.\overline{b}.d + a.b.\overline{c}.d + a.\overline{b}.\overline{c}.\overline{d} + \overline{a}.b.\overline{c}.\overline{d}  +  \overline{a}.\overline{b}.d + a.b.\overline{c}.d + a.\overline{b}.\overline{c}.\overline{d} + \overline{a}.b.\overline{c}.\overline{d} $






\hrule width 1\linewidth
\pagebreak

\subsection{Correction}


\paragraph{Q1}


Simplifier l'expression suivant.


Simpilfy the following expression
\begin{arab}[utf]
بسّط العبارة الآتية
\end{arab}

$S = \overline{a}.\overline{b}.d + a.b.\overline{c}.d + a.\overline{b}.\overline{c}.\overline{d} + \overline{a}.b.\overline{c}.\overline{d}  +  \overline{a}.\overline{b}.d + a.b.\overline{c}.d + a.\overline{b}.\overline{c}.\overline{d} + \overline{a}.b.\overline{c}.\overline{d} $




\textbf{Karnaugh Table \aRL{جدول كارنوف}}

\begin{karnaugh-map}[4][4][1][CD][AB]
  \minterms{ 1, 3, 4, 8, 13 }
  \maxterms{ 0, 2, 5, 6, 7, 9, 10, 11, 12, 14, 15 }

   \implicant{1}{3}
\implicant{13}{13}
\implicant{8}{8}
\implicant{4}{4}

 \end{karnaugh-map}

    Simplified Sum of products : $$\\
    Simplified product of sums : $$


\pagebreak

\paragraph{Q1}

Simplify the following Karnaugh table

\begin{arab}[utf]
بسّط الدوال الآتية باستعمال جدول كارنوف.
\end{arab}
\begin{karnaugh-map}[4][4][1][CD][AB]
  \minterms{ 0, 2, 6, 10, 11, 12 }
  \maxterms{ 1, 3, 4, 5, 7, 8, 9, 13, 14, 15 }
 \end{karnaugh-map}
\begin{karnaugh-map}[4][4][1][CD][AB]
  \minterms{ 3, 4, 6, 9, 10, 11, 12, 13, 14, 15 }
  \maxterms{ 0, 1, 2, 5, 7, 8 }
 \end{karnaugh-map}
\begin{karnaugh-map}[4][4][1][CD][AB]
  \minterms{ 2, 5, 6, 7, 8, 9, 10, 12, 13 }
  \maxterms{ 0, 1, 3, 4, 11, 14, 15 }
 \end{karnaugh-map}



\hrule width 1\linewidth
\pagebreak

\subsection{Correction}


\paragraph{Q1}

\begin{karnaugh-map}[4][4][1][CD][AB]
  \minterms{ 0, 2, 6, 10, 11, 12 }
  \maxterms{ 1, 3, 4, 5, 7, 8, 9, 13, 14, 15 }
       \implicant{10}{11}
\implicant{2}{6}
\implicantedge{0}{0}{2}{2}
\implicant{12}{12}
 \end{karnaugh-map}
    Simplified Sum of products : $ a.\overline{b}.c + \overline{a}.c.\overline{d} + \overline{a}.\overline{b}.\overline{d} + a.b.\overline{c}.\overline{d} $\\
    Simplified product of sums : $(a+\overline{d}).(c+\overline{d}).(a+\overline{b}+c).(\overline{a}+b+c).(\overline{a}+\overline{b}+\overline{c})$
\begin{karnaugh-map}[4][4][1][CD][AB]
  \minterms{ 3, 4, 6, 9, 10, 11, 12, 13, 14, 15 }
  \maxterms{ 0, 1, 2, 5, 7, 8 }
       \implicant{15}{10}
\implicant{13}{11}
\implicantedge{4}{12}{6}{14}
\implicantedge{3}{3}{11}{11}
 \end{karnaugh-map}
    Simplified Sum of products : $ a.c + a.d + b.\overline{d} + \overline{b}.c.d $\\
    Simplified product of sums : $(a+b+c).(a+b+d).(b+c+d).(a+\overline{b}+\overline{d})$
\begin{karnaugh-map}[4][4][1][CD][AB]
  \minterms{ 2, 5, 6, 7, 8, 9, 10, 12, 13 }
  \maxterms{ 0, 1, 3, 4, 11, 14, 15 }
       \implicant{12}{9}
\implicant{5}{7}
\implicant{2}{6}
\implicantedge{2}{2}{10}{10}
 \end{karnaugh-map}
    Simplified Sum of products : $ a.\overline{c} + \overline{a}.b.d + \overline{a}.c.\overline{d} + \overline{b}.c.\overline{d} $\\
    Simplified product of sums : $(a+c+d).(a+b+\overline{d}).(b+\overline{c}+\overline{d}).(\overline{a}+\overline{b}+\overline{c})$

\pagebreak

\paragraph{Q1}


Soit la fonction donnée par sa forme canonique, Tracer la table de karnaugh et simplifier.


Let the function be given by its canonical form, Draw the Karnaugh table and simplify.

\begin{arab}[utf]
لتكن الدالة المعطاة بشكلها القانوني، ارسم جدول كارنو وبسطها
\end{arab}
    $F1(A,B,C,D) = \overline{A}.\overline{B}.\overline{C}.D + \overline{A}.\overline{B}.C.\overline{D} + \overline{A}.B.\overline{C}.\overline{D} + \overline{A}.B.C.D + A.\overline{B}.\overline{C}.\overline{D} + A.\overline{B}.\overline{C}.D + A.\overline{B}.C.D + A.B.\overline{C}.\overline{D} + A.B.\overline{C}.D + A.B.C.D$
    $F2(A,B,C,D) = \overline{A}.\overline{B}.C.\overline{D} + \overline{A}.\overline{B}.C.D + \overline{A}.B.\overline{C}.\overline{D} + \overline{A}.B.\overline{C}.D + \overline{A}.B.C.\overline{D} + A.\overline{B}.\overline{C}.\overline{D} + A.\overline{B}.\overline{C}.D + A.\overline{B}.C.\overline{D} + A.B.\overline{C}.\overline{D} + A.B.\overline{C}.D + A.B.C.D$




\hrule width 1\linewidth
\pagebreak

\subsection{Correction}


\paragraph{Q1}


Soit la fonction donnée par sa forme canonique, Tracer la table de karnaugh et simplifier.


Let the function be given by its canonical form, Draw the Karnaugh table and simplify.

\begin{arab}[utf]
لتكن الدالة المعطاة بشكلها القانوني، ارسم جدول كارنو وبسطها
\end{arab}

    $F1(A,B,C,D) = \overline{A}.\overline{B}.\overline{C}.D + \overline{A}.\overline{B}.C.\overline{D} + \overline{A}.B.\overline{C}.\overline{D} + \overline{A}.B.C.D + A.\overline{B}.\overline{C}.\overline{D} + A.\overline{B}.\overline{C}.D + A.\overline{B}.C.D + A.B.\overline{C}.\overline{D} + A.B.\overline{C}.D + A.B.C.D$

\begin{karnaugh-map}[4][4][1][CD][AB]
  \minterms{ 1, 2, 4, 7, 8, 9, 11, 12, 13, 15 }
  \maxterms{ 0, 3, 5, 6, 10, 14 }

    \implicant{13}{11}
\implicant{12}{9}
\implicant{7}{15}
\implicant{4}{12}
\implicantedge{1}{1}{9}{9}
\implicant{2}{2}

 \end{karnaugh-map}

    Simplified Sum of products : $ a.d + a.\overline{c} + b.c.d + b.\overline{c}.\overline{d} + \overline{b}.\overline{c}.d + \overline{a}.\overline{b}.c.\overline{d} $\\
    Simplified product of sums : $(a+b+c+d).(\overline{a}+\overline{c}+d).(\overline{b}+\overline{c}+d).(a+b+\overline{c}+\overline{d}).(a+\overline{b}+c+\overline{d})$

    $F2(A,B,C,D) = \overline{A}.\overline{B}.C.\overline{D} + \overline{A}.\overline{B}.C.D + \overline{A}.B.\overline{C}.\overline{D} + \overline{A}.B.\overline{C}.D + \overline{A}.B.C.\overline{D} + A.\overline{B}.\overline{C}.\overline{D} + A.\overline{B}.\overline{C}.D + A.\overline{B}.C.\overline{D} + A.B.\overline{C}.\overline{D} + A.B.\overline{C}.D + A.B.C.D$

\begin{karnaugh-map}[4][4][1][CD][AB]
  \minterms{ 2, 3, 4, 5, 6, 8, 9, 10, 12, 13, 15 }
  \maxterms{ 0, 1, 7, 11, 14 }

    \implicant{12}{9}
\implicant{4}{13}
\implicant{13}{15}
\implicant{3}{2}
\implicant{2}{6}
\implicantedge{2}{2}{10}{10}

 \end{karnaugh-map}

    Simplified Sum of products : $ a.\overline{c} + b.\overline{c} + a.b.d + \overline{a}.\overline{b}.c + \overline{a}.c.\overline{d} + \overline{b}.c.\overline{d} $\\
    Simplified product of sums : $(a+b+c).(a+\overline{b}+\overline{c}+\overline{d}).(\overline{a}+b+\overline{c}+\overline{d}).(\overline{a}+\overline{b}+\overline{c}+d)$


\pagebreak

\paragraph{Q1}


Etudier la fonction suivante:


Study the following function:

\begin{arab}[utf]
ادرس الدالة الآتية:
\end{arab}
$F(A,B,C,D) = \overline{a}.\overline{b}.\overline{d} + \overline{a}.\overline{c}.\overline{d}  +  \overline{a}.\overline{b}.\overline{d} + \overline{a}.\overline{c}.\overline{d} $






\hrule width 1\linewidth
\pagebreak

\subsection{Correction}


\paragraph{Q1}


Etudier la fonction suivante:


Study the following function:

\begin{arab}[utf]
ادرس الدالة الآتية:
\end{arab}
$F(A,B,C,D) = \overline{a}.\overline{b}.\overline{d} + \overline{a}.\overline{c}.\overline{d}  +  \overline{a}.\overline{b}.\overline{d} + \overline{a}.\overline{c}.\overline{d} $



$F = [0, 2, 4]$

\textbf{Don't Care }

$F = []$



\textbf{Truth Table \aRL{جدول الحقيقة}}



\begin{tabular}{|c|c|c|c||c|}
\hline
A & B & C & D & F \\
        \hline
  0 & 0 & 0 & 0 & 1 \\
  0 & 0 & 0 & 1 & 0 \\
  0 & 0 & 1 & 0 & 1 \\
  0 & 0 & 1 & 1 & 0 \\
        \hline
  0 & 1 & 0 & 0 & 1 \\
  0 & 1 & 0 & 1 & 0 \\
  0 & 1 & 1 & 0 & 0 \\
  0 & 1 & 1 & 1 & 0 \\
        \hline
  1 & 0 & 0 & 0 & 0 \\
  1 & 0 & 0 & 1 & 0 \\
  1 & 0 & 1 & 0 & 0 \\
  1 & 0 & 1 & 1 & 0 \\
        \hline
  1 & 1 & 0 & 0 & 0 \\
  1 & 1 & 0 & 1 & 0 \\
  1 & 1 & 1 & 0 & 0 \\
  1 & 1 & 1 & 1 & 0 \\
\hline
\end{tabular}


\textbf{Canonical Forms \aRL{الأشكال القانونية}}
\begin{itemize}
\item $F(A,B,C,D) =  \overline{A}.\overline{B}.\overline{C}.\overline{D} + \overline{A}.\overline{B}.C.\overline{D} + \overline{A}.B.\overline{C}.\overline{D}$
\item $F(A,B,C,D) = (A+B+C+\overline{D}) . (A+B+\overline{C}+\overline{D}) . (A+\overline{B}+C+\overline{D}) . (A+\overline{B}+\overline{C}+D) . (A+\overline{B}+\overline{C}+\overline{D}) . (\overline{A}+B+C+D) . (\overline{A}+B+C+\overline{D}) . (\overline{A}+B+\overline{C}+D) . (\overline{A}+B+\overline{C}+\overline{D}) . (\overline{A}+\overline{B}+C+D) . (\overline{A}+\overline{B}+C+\overline{D}) . (\overline{A}+\overline{B}+\overline{C}+D) . (\overline{A}+\overline{B}+\overline{C}+\overline{D})$
 \item $F(A,B,C,D) =  \sum(0, 2, 4)$
 \item $F(A,B,C,D) =  \prod(1, 3, 5, 6, 7, 8, 9, 10, 11, 12, 13, 14, 15)$
\end{itemize}









\textbf{Karnaugh Table \aRL{جدول كارنوف}}

\begin{karnaugh-map}[4][4][1][CD][AB]
  \minterms{ 0, 2, 4 }
  \maxterms{ 1, 3, 5, 6, 7, 8, 9, 10, 11, 12, 13, 14, 15 }

       \implicantedge{0}{0}{2}{2}
\implicant{0}{4}

 \end{karnaugh-map}

    Simplified Sum of products : $ \overline{a}.\overline{b}.\overline{d} + \overline{a}.\overline{c}.\overline{d} $\\
    Simplified product of sums : $(\overline{a}).(\overline{d}).(\overline{b}+\overline{c})$\\

\textbf{Logic diagram \aRL{المخطط المنطقي}}




STRUCTURED LOGIGRAM-FILE-TEMPLATE



\begin{tikzpicture}

%%Paramaters
%% var position, can be modified
\def\varPos{ 1.36 }


\def\FunctionPos{6}
    \node (x1) at (0.0, \varPos) {$ A $};
    \node[not gate US, draw, rotate=270, scale=0.7] at ($(x1) + (0.25, -0.6)$) (notx1) {};
        \draw ($(x1)+(0,-1ex)$) -| (notx1.input);
    \node (x2) at (0.5, \varPos) {$ B $};
    \node[not gate US, draw, rotate=270, scale=0.7] at ($(x2) + (0.25, -0.6)$) (notx2) {};
        \draw ($(x2)+(0,-1ex)$) -| (notx2.input);
    \node (x3) at (1.0, \varPos) {$ C $};
    \node[not gate US, draw, rotate=270, scale=0.7] at ($(x3) + (0.25, -0.6)$) (notx3) {};
        \draw ($(x3)+(0,-1ex)$) -| (notx3.input);
    \node (x4) at (1.5, \varPos) {$ D $};
    \node[not gate US, draw, rotate=270, scale=0.7] at ($(x4) + (0.25, -0.6)$) (notx4) {};
        \draw ($(x4)+(0,-1ex)$) -| (notx4.input);



     %% ***Function F : Gate for term n° 1 [ {'default': "a'.c'.d'", 'formatted': "a'.c'.d'"} ]***
           \node[and gate US, draw, rotate=0, logic gate inputs=nnnn] at (2.5, 0.0) (xandyFg0) {};
           \draw (xandyFg0.output) -- node[above]{\scriptsize $ \overline{a}.\overline{c}.\overline{d} $} ($(xandyFg0) + (1.8, 0)$);
                \draw [line width=0.25mm,   red] (notx1.output)
                -- ([xshift=0cm]notx1.output) |- (xandyFg0.input 1);
                \draw [line width=0.25mm,   red] (notx3.output)
                -- ([xshift=0cm]notx3.output) |- (xandyFg0.input 2);
                \draw [line width=0.25mm,   red] (notx4.output)
                -- ([xshift=0cm]notx4.output) |- (xandyFg0.input 3);
     %% ***Function F : Gate for term n° 2 [ {'default': "a'.b'.d'", 'formatted': "a'.b'.d'"} ]***
           \node[and gate US, draw, rotate=0, logic gate inputs=nnnn] at (2.5, 1.2) (xandyFg1) {};
           \draw (xandyFg1.output) -- node[above]{\scriptsize $ \overline{a}.\overline{b}.\overline{d} $} ($(xandyFg1) + (1.8, 0)$);
                \draw [line width=0.25mm,   red] (notx1.output)
                -- ([xshift=0cm]notx1.output) |- (xandyFg1.input 1);
                \draw [line width=0.25mm,   red] (notx2.output)
                -- ([xshift=0cm]notx2.output) |- (xandyFg1.input 2);
                \draw [line width=0.25mm,   red] (notx4.output)
                -- ([xshift=0cm]notx4.output) |- (xandyFg1.input 3);


    %% y_pos : the position of OR gate according to their related gates

    %% Function F Large OR Gate

        \node[or gate US, draw, rotate=0, logic gate inputs=nnn] at (\FunctionPos, 0.6) (xoryF) {};
        \draw (xoryF.output) -- node[above]{\scriptsize $F$} ($(xoryF.east) + (+3ex, 0)$);



             \draw (xandyFg0.output) -- ++(1.6,0) |- (xoryF.input 2);



             \draw (xandyFg1.output) -- ++(1.6,0) |- (xoryF.input 1);



 \end{tikzpicture}



\pagebreak

\paragraph{Q1}











Convert the following numbers, \hfill\aRL{أنجز التحويلات الآتية}

$(4b)_{ 16 } = (........)_{ 2}$




\hrule width 1\linewidth
\pagebreak

\subsection{Correction}


\paragraph{Q1}











Convert the following numbers, \hfill\aRL{أنجز التحويلات الآتية}

$(4b)_{ 16 } = (  100\,1011)_{ 2}$


  \subsection*{Convert from base 16 to base 2}
  \[
\begin{tabular}{|c|*{ 2 }{c|}}
\hline
Base 16
 & 4
 & b
 \\
\hline
Base 2
 & 0100
 & 1011
 \\
\hline
\end{tabular}
\]




%<!-- Explanation of Base X to 10 -->


Result (bottom→top remainders): 1001011



\pagebreak

\paragraph{Q1}

Calculate  the following operations in base 8 :  أنجز العمليات الآتية في الأساس 8

\begin{verbatim}

  6620\,6645
- 5720\,5763
-------------------
=  ........

\end{verbatim}


\hrule width 1\linewidth
\pagebreak

\subsection{Correction}


\paragraph{Q1}

Calculate  the following operations in base 8 :  أنجز العمليات الآتية في الأساس 8

\begin{verbatim}

  6620\,6645
- 5720\,5763
-------------------
=  700\,0662

\end{verbatim}
\pagebreak

\paragraph{Q1}

% Question
\section*{Question}
A file was downloaded in 2.00 minutes with a speed of 51200.00 KB/s. What was the file size in GB?



\hrule width 1\linewidth
\pagebreak

\subsection{Correction}


\paragraph{Q1}

% Question
\section*{Question}
A file was downloaded in 2.00 minutes with a speed of 51200.00 KB/s. What was the file size in GB?
% Given parameters
\subsection*{Given}
\begin{itemize}
  \item Speed = 51200.00 \, KB/s
  \item Time = 0.08 \, minutes
\end{itemize}

% Solution steps
\subsection*{Solution}
\begin{tabular}{>{\bfseries}p{1cm} p{4cm} p{8cm}}
\toprule
Step & Operation & Expression \\
\midrule
1 & Convert time to seconds & 2.00 minutes = 120 seconds \\
2 & Convert speed to MB/s & 51200.00 KB/s = 50.00 MB/s \\
3 & Size = Speed × Time & 50.00 × 120 = 6000.00 MB \\
4 & Convert MB → GB & 6000.00 × 0.000976562 = 5.86 GB \\
\bottomrule
\end{tabular}

% Final Answer
\subsection*{Final Answer}
\textbf5.86 GB

\pagebreak

\paragraph{Q1}


Etudier la fonction suivante:


Study the following function:

\begin{arab}[utf]
ادرس الدالة الآتية:
\end{arab}
$F(A,B,C,D) =$






\hrule width 1\linewidth
\pagebreak

\subsection{Correction}


\paragraph{Q1}


Etudier la fonction suivante:


Study the following function:

\begin{arab}[utf]
ادرس الدالة الآتية:
\end{arab}
$F(A,B,C,D) =$



$Fx0 = [1, 3, 12, 14, 15]$

\textbf{Don't Care }

$Fx0 = []$



\textbf{Truth Table \aRL{جدول الحقيقة}}



\begin{tabular}{|c|c|c|c||c|}
\hline
X & Y & Z & W & F \\
        \hline
  0 & 0 & 0 & 0 & 0 \\
  0 & 0 & 0 & 1 & 1 \\
  0 & 0 & 1 & 0 & 0 \\
  0 & 0 & 1 & 1 & 1 \\
        \hline
  0 & 1 & 0 & 0 & 0 \\
  0 & 1 & 0 & 1 & 0 \\
  0 & 1 & 1 & 0 & 0 \\
  0 & 1 & 1 & 1 & 0 \\
        \hline
  1 & 0 & 0 & 0 & 0 \\
  1 & 0 & 0 & 1 & 0 \\
  1 & 0 & 1 & 0 & 0 \\
  1 & 0 & 1 & 1 & 0 \\
        \hline
  1 & 1 & 0 & 0 & 1 \\
  1 & 1 & 0 & 1 & 0 \\
  1 & 1 & 1 & 0 & 1 \\
  1 & 1 & 1 & 1 & 1 \\
\hline
\end{tabular}


\textbf{Canonical Forms \aRL{الأشكال القانونية}}
\begin{itemize}
\item $Fx0(A,B,C,D) =  \overline{X}.\overline{Y}.\overline{Z}.W + \overline{X}.\overline{Y}.Z.W + X.Y.\overline{Z}.\overline{W} + X.Y.Z.\overline{W} + X.Y.Z.W$
\item $Fx0(A,B,C,D) = (X+Y+Z+W) . (X+Y+\overline{Z}+W) . (X+\overline{Y}+Z+W) . (X+\overline{Y}+Z+\overline{W}) . (X+\overline{Y}+\overline{Z}+W) . (X+\overline{Y}+\overline{Z}+\overline{W}) . (\overline{X}+Y+Z+W) . (\overline{X}+Y+Z+\overline{W}) . (\overline{X}+Y+\overline{Z}+W) . (\overline{X}+Y+\overline{Z}+\overline{W}) . (\overline{X}+\overline{Y}+Z+\overline{W})$
 \item $Fx0(A,B,C,D) =  \sum(1, 3, 12, 14, 15)$
 \item $Fx0(A,B,C,D) =  \prod(0, 2, 4, 5, 6, 7, 8, 9, 10, 11, 13)$
\end{itemize}









\textbf{Karnaugh Table \aRL{جدول كارنوف}}

\begin{karnaugh-map}[4][4][1][ZW][XY]
  \minterms{ 1, 3, 12, 14, 15 }
  \maxterms{ 0, 2, 4, 5, 6, 7, 8, 9, 10, 11, 13 }

       \implicant{15}{14}
\implicantedge{12}{12}{14}{14}
\implicant{1}{3}

 \end{karnaugh-map}

    Simplified Sum of products : $ x.y.z + \overline{w}.x.y + w.\overline{x}.\overline{y} $\\
    Simplified product of sums : $(w+x).(x+\overline{y}).(\overline{x}+y).(\overline{w}+\overline{y}+z)$\\

\textbf{Logic diagram \aRL{المخطط المنطقي}}




STRUCTURED LOGIGRAM-FILE-TEMPLATE



\begin{tikzpicture}

%%Paramaters
%% var position, can be modified
\def\varPos{ 2.72 }


\def\FunctionPos{6}
    \node (x1) at (0.0, \varPos) {$ X $};
    \node[not gate US, draw, rotate=270, scale=0.7] at ($(x1) + (0.25, -0.6)$) (notx1) {};
        \draw ($(x1)+(0,-1ex)$) -| (notx1.input);
    \node (x2) at (0.5, \varPos) {$ Y $};
    \node[not gate US, draw, rotate=270, scale=0.7] at ($(x2) + (0.25, -0.6)$) (notx2) {};
        \draw ($(x2)+(0,-1ex)$) -| (notx2.input);
    \node (x3) at (1.0, \varPos) {$ Z $};
    \node[not gate US, draw, rotate=270, scale=0.7] at ($(x3) + (0.25, -0.6)$) (notx3) {};
        \draw ($(x3)+(0,-1ex)$) -| (notx3.input);
    \node (x4) at (1.5, \varPos) {$ W $};
    \node[not gate US, draw, rotate=270, scale=0.7] at ($(x4) + (0.25, -0.6)$) (notx4) {};
        \draw ($(x4)+(0,-1ex)$) -| (notx4.input);



     %% ***Function Fx0 : Gate for term n° 1 [ {'default': "w.x'.y'", 'formatted': "w.x'.y'"} ]***
           \node[and gate US, draw, rotate=0, logic gate inputs=nnnn] at (2.5, 0.0) (xandyFx0g0) {};
           \draw (xandyFx0g0.output) -- node[above]{\scriptsize $ w.\overline{x}.\overline{y} $} ($(xandyFx0g0) + (1.8, 0)$);
                \draw [line width=0.25mm,   red] (notx1.output)
                -- ([xshift=0cm]notx1.output) |- (xandyFx0g0.input 1);
                \draw [line width=0.25mm,   red] (notx2.output)
                -- ([xshift=0cm]notx2.output) |- (xandyFx0g0.input 2);
                \draw ($(x4) + (0, -1ex)$)|- (xandyFx0g0.input 3);
     %% ***Function Fx0 : Gate for term n° 2 [ {'default': "w'.x.y", 'formatted': "w'.x.y"} ]***
           \node[and gate US, draw, rotate=0, logic gate inputs=nnnn] at (2.5, 1.2) (xandyFx0g1) {};
           \draw (xandyFx0g1.output) -- node[above]{\scriptsize $ \overline{w}.x.y $} ($(xandyFx0g1) + (1.8, 0)$);
                \draw ($(x1) + (0, -1ex)$)|- (xandyFx0g1.input 1);
                \draw ($(x2) + (0, -1ex)$)|- (xandyFx0g1.input 2);
                \draw [line width=0.25mm,   red] (notx4.output)
                -- ([xshift=0cm]notx4.output) |- (xandyFx0g1.input 3);
     %% ***Function Fx0 : Gate for term n° 3 [ {'default': 'x.y.z', 'formatted': 'x.y.z'} ]***
           \node[and gate US, draw, rotate=0, logic gate inputs=nnnn] at (2.5, 2.4) (xandyFx0g2) {};
           \draw (xandyFx0g2.output) -- node[above]{\scriptsize $ x.y.z $} ($(xandyFx0g2) + (1.8, 0)$);
                \draw ($(x1) + (0, -1ex)$)|- (xandyFx0g2.input 1);
                \draw ($(x2) + (0, -1ex)$)|- (xandyFx0g2.input 2);
                \draw ($(x3) + (0, -1ex)$)|- (xandyFx0g2.input 3);


    %% y_pos : the position of OR gate according to their related gates

    %% Function Fx0 Large OR Gate

        \node[or gate US, draw, rotate=0, logic gate inputs=nnnn] at (\FunctionPos, 1.2) (xoryFx0) {};
        \draw (xoryFx0.output) -- node[above]{\scriptsize $Fx0$} ($(xoryFx0.east) + (+3ex, 0)$);



             \draw (xandyFx0g0.output) -- ++(1.6,0) |- (xoryFx0.input 3);



             \draw (xandyFx0g1.output) -- ++(1.6,0) |- (xoryFx0.input 2);



             \draw (xandyFx0g2.output) -- ++(1.6500000000000001,0) |- (xoryFx0.input 1);



 \end{tikzpicture}



\pagebreak

\paragraph{Q1}

Etudier le circuit suivant:


Study the following circuit:

\begin{arab}[utf]
ادرس الدارة الآتية:
\end{arab}


$Fx0 = [1, 3, 12, 14, 15]$


$F0 = [3, 5, 6, 7, 9, 10, 11, 12, 13, 14]$


$F1 = [1, 2, 4, 7, 8, 11, 13, 14]$


$F2 = [0, 1, 2, 3, 4, 5, 6, 7, 11, 12]$


$F3 = [0, 1, 2, 3, 4, 5, 6, 7]$


\textbf{Don't Care }
$Fx0 = []$
$F0 = []$
$F1 = []$
$F2 = []$
$F3 = []$


$F(A,B,C,D) =$







\hrule width 1\linewidth
\pagebreak

\subsection{Correction}


\paragraph{Q1}

Etudier le circuit suivant:


Study the following circuit:

\begin{arab}[utf]
ادرس الدارة الآتية:
\end{arab}


$Fx0 = [1, 3, 12, 14, 15]$


$F0 = [3, 5, 6, 7, 9, 10, 11, 12, 13, 14]$


$F1 = [1, 2, 4, 7, 8, 11, 13, 14]$


$F2 = [0, 1, 2, 3, 4, 5, 6, 7, 11, 12]$


$F3 = [0, 1, 2, 3, 4, 5, 6, 7]$


\textbf{Don't Care }
$Fx0 = []$
$F0 = []$
$F1 = []$
$F2 = []$
$F3 = []$


$F(A,B,C,D) =$





\textbf{Inputs and Outputs \aRL{المداخل والمخارج}}

\begin{itemize}
\item Inputs

    \begin{itemize}
        \item $X = 0 /1 $
        \item $Y = 0 /1 $
        \item $Z = 0 /1 $
        \item $W = 0 /1 $
    \end{itemize}
\item Outputs
    \begin{itemize}
        \item $Fx0 = 0 /1 $
        \item $F0 = 0 /1 $
        \item $F1 = 0 /1 $
        \item $F2 = 0 /1 $
        \item $F3 = 0 /1 $
    \end{itemize}
\end{itemize}

\textbf{Truth Table \aRL{جدول الحقيقة}}

\begin{tabular}{|c|c|c|c||c|c|c|c|c|}
\hline
X & Y & Z & W
  & Fx0
  & F0
  & F1
  & F2
  & F3
 \\
\hline
  0 & 0 & 0 & 0
    & 0
    & 0
    & 0
    & 1
    & 1
 \\
  0 & 0 & 0 & 1
    & 1
    & 0
    & 1
    & 1
    & 1
 \\
  0 & 0 & 1 & 0
    & 0
    & 0
    & 1
    & 1
    & 1
 \\
  0 & 0 & 1 & 1
    & 1
    & 1
    & 0
    & 1
    & 1
 \\
\hline  0 & 1 & 0 & 0
    & 0
    & 0
    & 1
    & 1
    & 1
 \\
  0 & 1 & 0 & 1
    & 0
    & 1
    & 0
    & 1
    & 1
 \\
  0 & 1 & 1 & 0
    & 0
    & 1
    & 0
    & 1
    & 1
 \\
  0 & 1 & 1 & 1
    & 0
    & 1
    & 1
    & 1
    & 1
 \\
\hline  1 & 0 & 0 & 0
    & 0
    & 0
    & 1
    & 0
    & 0
 \\
  1 & 0 & 0 & 1
    & 0
    & 1
    & 0
    & 0
    & 0
 \\
  1 & 0 & 1 & 0
    & 0
    & 1
    & 0
    & 0
    & 0
 \\
  1 & 0 & 1 & 1
    & 0
    & 1
    & 1
    & 1
    & 0
 \\
\hline  1 & 1 & 0 & 0
    & 1
    & 1
    & 0
    & 1
    & 0
 \\
  1 & 1 & 0 & 1
    & 0
    & 1
    & 1
    & 0
    & 0
 \\
  1 & 1 & 1 & 0
    & 1
    & 1
    & 1
    & 0
    & 0
 \\
  1 & 1 & 1 & 1
    & 1
    & 0
    & 0
    & 0
    & 0
 \\
\hline\end{tabular}



\textbf{Canonical Forms \aRL{الأشكال القانونية}}
\begin{itemize}
\item \textbf{First Canonical Forms \aRL{ الأشكال القانونية الأولى}}
    \begin{itemize}
        \item $Fx0(X, Y, Z, W) =  \overline{X}.\overline{Y}.\overline{Z}.W + \overline{X}.\overline{Y}.Z.W + X.Y.\overline{Z}.\overline{W} + X.Y.Z.\overline{W} + X.Y.Z.W$
        \item $F0(X, Y, Z, W) =  \overline{X}.\overline{Y}.Z.W + \overline{X}.Y.\overline{Z}.W + \overline{X}.Y.Z.\overline{W} + \overline{X}.Y.Z.W + X.\overline{Y}.\overline{Z}.W + X.\overline{Y}.Z.\overline{W} + X.\overline{Y}.Z.W + X.Y.\overline{Z}.\overline{W} + X.Y.\overline{Z}.W + X.Y.Z.\overline{W}$
        \item $F1(X, Y, Z, W) =  \overline{X}.\overline{Y}.\overline{Z}.W + \overline{X}.\overline{Y}.Z.\overline{W} + \overline{X}.Y.\overline{Z}.\overline{W} + \overline{X}.Y.Z.W + X.\overline{Y}.\overline{Z}.\overline{W} + X.\overline{Y}.Z.W + X.Y.\overline{Z}.W + X.Y.Z.\overline{W}$
        \item $F2(X, Y, Z, W) =  \overline{X}.\overline{Y}.\overline{Z}.\overline{W} + \overline{X}.\overline{Y}.\overline{Z}.W + \overline{X}.\overline{Y}.Z.\overline{W} + \overline{X}.\overline{Y}.Z.W + \overline{X}.Y.\overline{Z}.\overline{W} + \overline{X}.Y.\overline{Z}.W + \overline{X}.Y.Z.\overline{W} + \overline{X}.Y.Z.W + X.\overline{Y}.Z.W + X.Y.\overline{Z}.\overline{W}$
        \item $F3(X, Y, Z, W) =  \overline{X}.\overline{Y}.\overline{Z}.\overline{W} + \overline{X}.\overline{Y}.\overline{Z}.W + \overline{X}.\overline{Y}.Z.\overline{W} + \overline{X}.\overline{Y}.Z.W + \overline{X}.Y.\overline{Z}.\overline{W} + \overline{X}.Y.\overline{Z}.W + \overline{X}.Y.Z.\overline{W} + \overline{X}.Y.Z.W$
    \end{itemize}

\item \textbf{Second Canonical Forms \aRL{ الأشكال القانونية الثانية}}
    \begin{itemize}
        \item $Fx0(X, Y, Z, W) =  (X+Y+Z+W) . (X+Y+\overline{Z}+W) . (X+\overline{Y}+Z+W) . (X+\overline{Y}+Z+\overline{W}) . (X+\overline{Y}+\overline{Z}+W) . (X+\overline{Y}+\overline{Z}+\overline{W}) . (\overline{X}+Y+Z+W) . (\overline{X}+Y+Z+\overline{W}) . (\overline{X}+Y+\overline{Z}+W) . (\overline{X}+Y+\overline{Z}+\overline{W}) . (\overline{X}+\overline{Y}+Z+\overline{W})$
        \item $F0(X, Y, Z, W) =  (X+Y+Z+W) . (X+Y+Z+\overline{W}) . (X+Y+\overline{Z}+W) . (X+\overline{Y}+Z+W) . (\overline{X}+Y+Z+W) . (\overline{X}+\overline{Y}+\overline{Z}+\overline{W})$
        \item $F1(X, Y, Z, W) =  (X+Y+Z+W) . (X+Y+\overline{Z}+\overline{W}) . (X+\overline{Y}+Z+\overline{W}) . (X+\overline{Y}+\overline{Z}+W) . (\overline{X}+Y+Z+\overline{W}) . (\overline{X}+Y+\overline{Z}+W) . (\overline{X}+\overline{Y}+Z+W) . (\overline{X}+\overline{Y}+\overline{Z}+\overline{W})$
        \item $F2(X, Y, Z, W) =  (\overline{X}+Y+Z+W) . (\overline{X}+Y+Z+\overline{W}) . (\overline{X}+Y+\overline{Z}+W) . (\overline{X}+\overline{Y}+Z+\overline{W}) . (\overline{X}+\overline{Y}+\overline{Z}+W) . (\overline{X}+\overline{Y}+\overline{Z}+\overline{W})$
        \item $F3(X, Y, Z, W) =  (\overline{X}+Y+Z+W) . (\overline{X}+Y+Z+\overline{W}) . (\overline{X}+Y+\overline{Z}+W) . (\overline{X}+Y+\overline{Z}+\overline{W}) . (\overline{X}+\overline{Y}+Z+W) . (\overline{X}+\overline{Y}+Z+\overline{W}) . (\overline{X}+\overline{Y}+\overline{Z}+W) . (\overline{X}+\overline{Y}+\overline{Z}+\overline{W})$
    \end{itemize}

\item \textbf{First Canonical Forms \aRL{ الأشكال القانونية الأولى}}
    \begin{itemize}
        \item $Fx0(X, Y, Z, W) = \sum(1, 3, 12, 14, 15)$
        \item $F0(X, Y, Z, W) = \sum(3, 5, 6, 7, 9, 10, 11, 12, 13, 14)$
        \item $F1(X, Y, Z, W) = \sum(1, 2, 4, 7, 8, 11, 13, 14)$
        \item $F2(X, Y, Z, W) = \sum(0, 1, 2, 3, 4, 5, 6, 7, 11, 12)$
        \item $F3(X, Y, Z, W) = \sum(0, 1, 2, 3, 4, 5, 6, 7)$
    \end{itemize}

\item \textbf{Second Canonical Forms \aRL{ الأشكال القانونية الثانية}}
    \begin{itemize}
        \item $Fx0(X, Y, Z, W) =  \prod(0, 2, 4, 5, 6, 7, 8, 9, 10, 11, 13)$
        \item $F0(X, Y, Z, W) =  \prod(0, 1, 2, 4, 8, 15)$
        \item $F1(X, Y, Z, W) =  \prod(0, 3, 5, 6, 9, 10, 12, 15)$
        \item $F2(X, Y, Z, W) =  \prod(8, 9, 10, 13, 14, 15)$
        \item $F3(X, Y, Z, W) =  \prod(8, 9, 10, 11, 12, 13, 14, 15)$
    \end{itemize}

\end{itemize}




 \textbf{ NAND forms  \aRL{بوابات نفي الوصل}}
 \begin{enumerate}

    \item $Fx0$ = $(x\uparrow y\uparrow z)\big\uparrow (\overline{w}\uparrow x\uparrow y)\big\uparrow (w\uparrow \overline{x}\uparrow \overline{y})$

    Explanation
    \begin{itemize}
        \item $Fx0$ = $ x.y.z + \overline{w}.x.y + w.\overline{x}.\overline{y} $

            \item  $Fx0$ = $\overline{\overline{ x.y.z + \overline{w}.x.y + w.\overline{x}.\overline{y} }}$
            \item  $Fx0$ = $\overline{\overline{ x.y.z }.\overline{ \overline{w}.x.y }.\overline{ w.\overline{x}.\overline{y} }}$
            \item  $Fx0$ = $(x\uparrow y\uparrow z)\big\uparrow (\overline{w}\uparrow x\uparrow y)\big\uparrow (w\uparrow \overline{x}\uparrow \overline{y})$
    \end{itemize}

    \item $F0$ = $(w\uparrow x\uparrow \overline{y})\big\uparrow (w\uparrow y\uparrow \overline{z})\big\uparrow (w\uparrow \overline{x}\uparrow z)\big\uparrow (x\uparrow y\uparrow \overline{z})\big\uparrow (x\uparrow \overline{y}\uparrow z)\big\uparrow (\overline{w}\uparrow y\uparrow z)$

    Explanation
    \begin{itemize}
        \item $F0$ = $ w.x.\overline{y} + w.y.\overline{z} + w.\overline{x}.z + x.y.\overline{z} + x.\overline{y}.z + \overline{w}.y.z $

            \item  $F0$ = $\overline{\overline{ w.x.\overline{y} + w.y.\overline{z} + w.\overline{x}.z + x.y.\overline{z} + x.\overline{y}.z + \overline{w}.y.z }}$
            \item  $F0$ = $\overline{\overline{ w.x.\overline{y} }.\overline{ w.y.\overline{z} }.\overline{ w.\overline{x}.z }.\overline{ x.y.\overline{z} }.\overline{ x.\overline{y}.z }.\overline{ \overline{w}.y.z }}$
            \item  $F0$ = $(w\uparrow x\uparrow \overline{y})\big\uparrow (w\uparrow y\uparrow \overline{z})\big\uparrow (w\uparrow \overline{x}\uparrow z)\big\uparrow (x\uparrow y\uparrow \overline{z})\big\uparrow (x\uparrow \overline{y}\uparrow z)\big\uparrow (\overline{w}\uparrow y\uparrow z)$
    \end{itemize}

    \item $F1$ = $(w\uparrow x\uparrow y\uparrow \overline{z})\big\uparrow (w\uparrow x\uparrow \overline{y}\uparrow z)\big\uparrow (w\uparrow \overline{x}\uparrow y\uparrow z)\big\uparrow (\overline{w}\uparrow x\uparrow y\uparrow z)\big\uparrow (w\uparrow \overline{x}\uparrow \overline{y}\uparrow \overline{z})\big\uparrow (\overline{w}\uparrow x\uparrow \overline{y}\uparrow \overline{z})\big\uparrow (\overline{w}\uparrow \overline{x}\uparrow y\uparrow \overline{z})\big\uparrow (\overline{w}\uparrow \overline{x}\uparrow \overline{y}\uparrow z)$

    Explanation
    \begin{itemize}
        \item $F1$ = $ w.x.y.\overline{z} + w.x.\overline{y}.z + w.\overline{x}.y.z + \overline{w}.x.y.z + w.\overline{x}.\overline{y}.\overline{z} + \overline{w}.x.\overline{y}.\overline{z} + \overline{w}.\overline{x}.y.\overline{z} + \overline{w}.\overline{x}.\overline{y}.z $

            \item  $F1$ = $\overline{\overline{ w.x.y.\overline{z} + w.x.\overline{y}.z + w.\overline{x}.y.z + \overline{w}.x.y.z + w.\overline{x}.\overline{y}.\overline{z} + \overline{w}.x.\overline{y}.\overline{z} + \overline{w}.\overline{x}.y.\overline{z} + \overline{w}.\overline{x}.\overline{y}.z }}$
            \item  $F1$ = $\overline{\overline{ w.x.y.\overline{z} }.\overline{ w.x.\overline{y}.z }.\overline{ w.\overline{x}.y.z }.\overline{ \overline{w}.x.y.z }.\overline{ w.\overline{x}.\overline{y}.\overline{z} }.\overline{ \overline{w}.x.\overline{y}.\overline{z} }.\overline{ \overline{w}.\overline{x}.y.\overline{z} }.\overline{ \overline{w}.\overline{x}.\overline{y}.z }}$
            \item  $F1$ = $(w\uparrow x\uparrow y\uparrow \overline{z})\big\uparrow (w\uparrow x\uparrow \overline{y}\uparrow z)\big\uparrow (w\uparrow \overline{x}\uparrow y\uparrow z)\big\uparrow (\overline{w}\uparrow x\uparrow y\uparrow z)\big\uparrow (w\uparrow \overline{x}\uparrow \overline{y}\uparrow \overline{z})\big\uparrow (\overline{w}\uparrow x\uparrow \overline{y}\uparrow \overline{z})\big\uparrow (\overline{w}\uparrow \overline{x}\uparrow y\uparrow \overline{z})\big\uparrow (\overline{w}\uparrow \overline{x}\uparrow \overline{y}\uparrow z)$
    \end{itemize}

    \item $F2$ = $(x)\big\uparrow (w\uparrow \overline{y}\uparrow z)\big\uparrow (\overline{w}\uparrow y\uparrow \overline{z})$

    Explanation
    \begin{itemize}
        \item $F2$ = $ \overline{x} + w.\overline{y}.z + \overline{w}.y.\overline{z} $

            \item  $F2$ = $\overline{\overline{ \overline{x} + w.\overline{y}.z + \overline{w}.y.\overline{z} }}$
            \item  $F2$ = $\overline{\overline{ \overline{x} }.\overline{ w.\overline{y}.z }.\overline{ \overline{w}.y.\overline{z} }}$
            \item  $F2$ = $(x)\big\uparrow (w\uparrow \overline{y}\uparrow z)\big\uparrow (\overline{w}\uparrow y\uparrow \overline{z})$
    \end{itemize}

    \item $F3$ = $ \overline{x} $

    Explanation
    \begin{itemize}
        \item $F3$ = $ \overline{x} $

            \item  $F3$ = $ \overline{x} $
    \end{itemize}
 \end{enumerate}



\textbf{Karnaugh Table \aRL{جدول كارنوف}}

\begin{karnaugh-map}[4][4][1][ZW][XY]
  \minterms{ 1, 3, 12, 14, 15 }
  \maxterms{ 0, 2, 4, 5, 6, 7, 8, 9, 10, 11, 13 }
       \implicant{15}{14}
\implicantedge{12}{12}{14}{14}
\implicant{1}{3}
 \end{karnaugh-map}

    Simplified Sum of products : $Fx0 =  x.y.z + \overline{w}.x.y + w.\overline{x}.\overline{y} $\\
    Simplified product of sums : $Fx0 = (w+x).(x+\overline{y}).(\overline{x}+y).(\overline{w}+\overline{y}+z)$\\

\begin{karnaugh-map}[4][4][1][ZW][XY]
  \minterms{ 3, 5, 6, 7, 9, 10, 11, 12, 13, 14 }
  \maxterms{ 0, 1, 2, 4, 8, 15 }
       \implicant{12}{13}
\implicant{10}{11}
\implicant{9}{11}
\implicant{6}{14}
\implicant{5}{13}
\implicant{3}{7}
 \end{karnaugh-map}

    Simplified Sum of products : $F0 =  w.x.\overline{y} + w.y.\overline{z} + w.\overline{x}.z + x.y.\overline{z} + x.\overline{y}.z + \overline{w}.y.z $\\
    Simplified product of sums : $F0 = (w+x+y).(w+x+z).(w+y+z).(x+y+z).(\overline{w}+\overline{x}+\overline{y}+\overline{z})$\\

\begin{karnaugh-map}[4][4][1][ZW][XY]
  \minterms{ 1, 2, 4, 7, 8, 11, 13, 14 }
  \maxterms{ 0, 3, 5, 6, 9, 10, 12, 15 }
       \implicant{14}{14}
\implicant{13}{13}
\implicant{11}{11}
\implicant{7}{7}
\implicant{8}{8}
\implicant{4}{4}
\implicant{2}{2}
\implicant{1}{1}
 \end{karnaugh-map}

    Simplified Sum of products : $F1 =  w.x.y.\overline{z} + w.x.\overline{y}.z + w.\overline{x}.y.z + \overline{w}.x.y.z + w.\overline{x}.\overline{y}.\overline{z} + \overline{w}.x.\overline{y}.\overline{z} + \overline{w}.\overline{x}.y.\overline{z} + \overline{w}.\overline{x}.\overline{y}.z $\\
    Simplified product of sums : $F1 = (w+x+y+z).(w+x+\overline{y}+\overline{z}).(w+\overline{x}+y+\overline{z}).(w+\overline{x}+\overline{y}+z).(\overline{w}+x+y+\overline{z}).(\overline{w}+x+\overline{y}+z).(\overline{w}+\overline{x}+y+z).(\overline{w}+\overline{x}+\overline{y}+\overline{z})$\\

\begin{karnaugh-map}[4][4][1][ZW][XY]
  \minterms{ 0, 1, 2, 3, 4, 5, 6, 7, 11, 12 }
  \maxterms{ 8, 9, 10, 13, 14, 15 }
       \implicant{0}{6}
\implicantedge{3}{3}{11}{11}
\implicant{4}{12}
 \end{karnaugh-map}

    Simplified Sum of products : $F2 =  \overline{x} + w.\overline{y}.z + \overline{w}.y.\overline{z} $\\
    Simplified product of sums : $F2 = (\overline{x}+y+z).(w+\overline{x}+\overline{z}).(\overline{w}+\overline{x}+\overline{y})$\\

\begin{karnaugh-map}[4][4][1][ZW][XY]
  \minterms{ 0, 1, 2, 3, 4, 5, 6, 7 }
  \maxterms{ 8, 9, 10, 11, 12, 13, 14, 15 }
       \implicant{0}{6}
 \end{karnaugh-map}

    Simplified Sum of products : $F3 =  \overline{x} $\\
    Simplified product of sums : $F3 = (\overline{x})$\\



\textbf{Ligc diagram \aRL{المخطط المنطقي}}




STRUCTURED LOGIGRAM-FILE-TEMPLATE



\begin{tikzpicture}

%%Paramaters
%% var position, can be modified
\def\varPos{ 27.200000000000003 }


\def\FunctionPos{6}
    \node (x1) at (0.0, \varPos) {$ X $};
    \node[nand gate US, draw, rotate=270, scale=0.7] at ($(x1) + (0.25, -0.6)$) (notx1) {};
        \draw ($(x1)+(0,-1ex)$) -| (notx1.input 1);
        \draw ($(x1)+(0,-1ex)$) -| (notx1.input 2);
    \node (x2) at (0.5, \varPos) {$ Y $};
    \node[nand gate US, draw, rotate=270, scale=0.7] at ($(x2) + (0.25, -0.6)$) (notx2) {};
        \draw ($(x2)+(0,-1ex)$) -| (notx2.input 1);
        \draw ($(x2)+(0,-1ex)$) -| (notx2.input 2);
    \node (x3) at (1.0, \varPos) {$ Z $};
    \node[nand gate US, draw, rotate=270, scale=0.7] at ($(x3) + (0.25, -0.6)$) (notx3) {};
        \draw ($(x3)+(0,-1ex)$) -| (notx3.input 1);
        \draw ($(x3)+(0,-1ex)$) -| (notx3.input 2);
    \node (x4) at (1.5, \varPos) {$ W $};
    \node[nand gate US, draw, rotate=270, scale=0.7] at ($(x4) + (0.25, -0.6)$) (notx4) {};
        \draw ($(x4)+(0,-1ex)$) -| (notx4.input 1);
        \draw ($(x4)+(0,-1ex)$) -| (notx4.input 2);



     %% ***Function Fx0 : Gate for term n° 1 [ {'default': "w↑x'↑y'", 'formatted': "w↑x'↑y'"} ]***
           \node[nand gate US, draw, rotate=0, logic gate inputs=nnnn] at (2.5, 0.0) (xandyFx0g0) {};
           \draw (xandyFx0g0.output) -- node[above]{\scriptsize $ w\uparrow \overline{x}\uparrow \overline{y} $} ($(xandyFx0g0) + (1.8, 0)$);
                \draw [line width=0.25mm,   red] (notx1.output)
                -- ([xshift=0cm]notx1.output) |- (xandyFx0g0.input 1);
                \draw [line width=0.25mm,   red] (notx2.output)
                -- ([xshift=0cm]notx2.output) |- (xandyFx0g0.input 2);
                \draw ($(x4) + (0, -1ex)$)|- (xandyFx0g0.input 3);
     %% ***Function Fx0 : Gate for term n° 2 [ {'default': "w'↑x↑y", 'formatted': "w'↑x↑y"} ]***
           \node[nand gate US, draw, rotate=0, logic gate inputs=nnnn] at (2.5, 1.2) (xandyFx0g1) {};
           \draw (xandyFx0g1.output) -- node[above]{\scriptsize $ \overline{w}\uparrow x\uparrow y $} ($(xandyFx0g1) + (1.8, 0)$);
                \draw ($(x1) + (0, -1ex)$)|- (xandyFx0g1.input 1);
                \draw ($(x2) + (0, -1ex)$)|- (xandyFx0g1.input 2);
                \draw [line width=0.25mm,   red] (notx4.output)
                -- ([xshift=0cm]notx4.output) |- (xandyFx0g1.input 3);
     %% ***Function Fx0 : Gate for term n° 3 [ {'default': 'x↑y↑z', 'formatted': 'x↑y↑z'} ]***
           \node[nand gate US, draw, rotate=0, logic gate inputs=nnnn] at (2.5, 2.4) (xandyFx0g2) {};
           \draw (xandyFx0g2.output) -- node[above]{\scriptsize $ x\uparrow y\uparrow z $} ($(xandyFx0g2) + (1.8, 0)$);
                \draw ($(x1) + (0, -1ex)$)|- (xandyFx0g2.input 1);
                \draw ($(x2) + (0, -1ex)$)|- (xandyFx0g2.input 2);
                \draw ($(x3) + (0, -1ex)$)|- (xandyFx0g2.input 3);


    %% y_pos : the position of OR gate according to their related gates

    %% Function Fx0 Large OR Gate

        \node[nand gate US, draw, rotate=0, logic gate inputs=nnnn] at (\FunctionPos, 1.2) (xoryFx0) {};
        \draw (xoryFx0.output) -- node[above]{\scriptsize $Fx0$} ($(xoryFx0.east) + (+3ex, 0)$);



             \draw (xandyFx0g0.output) -- ++(1.6,0) |- (xoryFx0.input 3);



             \draw (xandyFx0g1.output) -- ++(1.6,0) |- (xoryFx0.input 2);



             \draw (xandyFx0g2.output) -- ++(1.6500000000000001,0) |- (xoryFx0.input 1);


     %% ***Function F0 : Gate for term n° 1 [ {'default': "w'↑y↑z", 'formatted': "w'↑y↑z"} ]***
           \node[nand gate US, draw, rotate=0, logic gate inputs=nnnn] at (2.5, 3.5999999999999996) (xandyF0g0) {};
           \draw (xandyF0g0.output) -- node[above]{\scriptsize $ \overline{w}\uparrow y\uparrow z $} ($(xandyF0g0) + (1.8, 0)$);
                \draw ($(x2) + (0, -1ex)$)|- (xandyF0g0.input 1);
                \draw ($(x3) + (0, -1ex)$)|- (xandyF0g0.input 2);
                \draw [line width=0.25mm,   red] (notx4.output)
                -- ([xshift=0cm]notx4.output) |- (xandyF0g0.input 3);
     %% ***Function F0 : Gate for term n° 2 [ {'default': "x↑y'↑z", 'formatted': "x↑y'↑z"} ]***
           \node[nand gate US, draw, rotate=0, logic gate inputs=nnnn] at (2.5, 4.8) (xandyF0g1) {};
           \draw (xandyF0g1.output) -- node[above]{\scriptsize $ x\uparrow \overline{y}\uparrow z $} ($(xandyF0g1) + (1.8, 0)$);
                \draw ($(x1) + (0, -1ex)$)|- (xandyF0g1.input 1);
                \draw [line width=0.25mm,   red] (notx2.output)
                -- ([xshift=0cm]notx2.output) |- (xandyF0g1.input 2);
                \draw ($(x3) + (0, -1ex)$)|- (xandyF0g1.input 3);
     %% ***Function F0 : Gate for term n° 3 [ {'default': "x↑y↑z'", 'formatted': "x↑y↑z'"} ]***
           \node[nand gate US, draw, rotate=0, logic gate inputs=nnnn] at (2.5, 6.0) (xandyF0g2) {};
           \draw (xandyF0g2.output) -- node[above]{\scriptsize $ x\uparrow y\uparrow \overline{z} $} ($(xandyF0g2) + (1.8, 0)$);
                \draw ($(x1) + (0, -1ex)$)|- (xandyF0g2.input 1);
                \draw ($(x2) + (0, -1ex)$)|- (xandyF0g2.input 2);
                \draw [line width=0.25mm,   red] (notx3.output)
                -- ([xshift=0cm]notx3.output) |- (xandyF0g2.input 3);
     %% ***Function F0 : Gate for term n° 4 [ {'default': "w↑x'↑z", 'formatted': "w↑x'↑z"} ]***
           \node[nand gate US, draw, rotate=0, logic gate inputs=nnnn] at (2.5, 7.199999999999999) (xandyF0g3) {};
           \draw (xandyF0g3.output) -- node[above]{\scriptsize $ w\uparrow \overline{x}\uparrow z $} ($(xandyF0g3) + (1.8, 0)$);
                \draw [line width=0.25mm,   red] (notx1.output)
                -- ([xshift=0cm]notx1.output) |- (xandyF0g3.input 1);
                \draw ($(x3) + (0, -1ex)$)|- (xandyF0g3.input 2);
                \draw ($(x4) + (0, -1ex)$)|- (xandyF0g3.input 3);
     %% ***Function F0 : Gate for term n° 5 [ {'default': "w↑y↑z'", 'formatted': "w↑y↑z'"} ]***
           \node[nand gate US, draw, rotate=0, logic gate inputs=nnnn] at (2.5, 8.4) (xandyF0g4) {};
           \draw (xandyF0g4.output) -- node[above]{\scriptsize $ w\uparrow y\uparrow \overline{z} $} ($(xandyF0g4) + (1.8, 0)$);
                \draw ($(x2) + (0, -1ex)$)|- (xandyF0g4.input 1);
                \draw [line width=0.25mm,   red] (notx3.output)
                -- ([xshift=0cm]notx3.output) |- (xandyF0g4.input 2);
                \draw ($(x4) + (0, -1ex)$)|- (xandyF0g4.input 3);
     %% ***Function F0 : Gate for term n° 6 [ {'default': "w↑x↑y'", 'formatted': "w↑x↑y'"} ]***
           \node[nand gate US, draw, rotate=0, logic gate inputs=nnnn] at (2.5, 9.6) (xandyF0g5) {};
           \draw (xandyF0g5.output) -- node[above]{\scriptsize $ w\uparrow x\uparrow \overline{y} $} ($(xandyF0g5) + (1.8, 0)$);
                \draw ($(x1) + (0, -1ex)$)|- (xandyF0g5.input 1);
                \draw [line width=0.25mm,   red] (notx2.output)
                -- ([xshift=0cm]notx2.output) |- (xandyF0g5.input 2);
                \draw ($(x4) + (0, -1ex)$)|- (xandyF0g5.input 3);


    %% y_pos : the position of OR gate according to their related gates

    %% Function F0 Large OR Gate

        \node[nand gate US, draw, rotate=0, logic gate inputs=nnnnnnn] at (\FunctionPos, 6.6) (xoryF0) {};
        \draw (xoryF0.output) -- node[above]{\scriptsize $F0$} ($(xoryF0.east) + (+3ex, 0)$);



             \draw (xandyF0g0.output) -- ++(1.6,0) |- (xoryF0.input 6);



             \draw (xandyF0g1.output) -- ++(1.5,0) |- (xoryF0.input 5);



             \draw (xandyF0g2.output) -- ++(1.55,0) |- (xoryF0.input 4);



             \draw (xandyF0g3.output) -- ++(1.6,0) |- (xoryF0.input 3);



             \draw (xandyF0g4.output) -- ++(1.6500000000000001,0) |- (xoryF0.input 2);



             \draw (xandyF0g5.output) -- ++(1.7000000000000002,0) |- (xoryF0.input 1);


     %% ***Function F1 : Gate for term n° 1 [ {'default': "w'↑x'↑y'↑z", 'formatted': "w'↑x'↑y'↑z"} ]***
           \node[nand gate US, draw, rotate=0, logic gate inputs=nnnn] at (2.5, 10.799999999999999) (xandyF1g0) {};
           \draw (xandyF1g0.output) -- node[above]{\scriptsize $ \overline{w}\uparrow \overline{x}\uparrow \overline{y}\uparrow z $} ($(xandyF1g0) + (1.8, 0)$);
                \draw [line width=0.25mm,   red] (notx1.output)
                -- ([xshift=0cm]notx1.output) |- (xandyF1g0.input 1);
                \draw [line width=0.25mm,   red] (notx2.output)
                -- ([xshift=0cm]notx2.output) |- (xandyF1g0.input 2);
                \draw ($(x3) + (0, -1ex)$)|- (xandyF1g0.input 3);
                \draw [line width=0.25mm,   red] (notx4.output)
                -- ([xshift=0cm]notx4.output) |- (xandyF1g0.input 4);
     %% ***Function F1 : Gate for term n° 2 [ {'default': "w'↑x'↑y↑z'", 'formatted': "w'↑x'↑y↑z'"} ]***
           \node[nand gate US, draw, rotate=0, logic gate inputs=nnnn] at (2.5, 12.0) (xandyF1g1) {};
           \draw (xandyF1g1.output) -- node[above]{\scriptsize $ \overline{w}\uparrow \overline{x}\uparrow y\uparrow \overline{z} $} ($(xandyF1g1) + (1.8, 0)$);
                \draw [line width=0.25mm,   red] (notx1.output)
                -- ([xshift=0cm]notx1.output) |- (xandyF1g1.input 1);
                \draw ($(x2) + (0, -1ex)$)|- (xandyF1g1.input 2);
                \draw [line width=0.25mm,   red] (notx3.output)
                -- ([xshift=0cm]notx3.output) |- (xandyF1g1.input 3);
                \draw [line width=0.25mm,   red] (notx4.output)
                -- ([xshift=0cm]notx4.output) |- (xandyF1g1.input 4);
     %% ***Function F1 : Gate for term n° 3 [ {'default': "w'↑x↑y'↑z'", 'formatted': "w'↑x↑y'↑z'"} ]***
           \node[nand gate US, draw, rotate=0, logic gate inputs=nnnn] at (2.5, 13.2) (xandyF1g2) {};
           \draw (xandyF1g2.output) -- node[above]{\scriptsize $ \overline{w}\uparrow x\uparrow \overline{y}\uparrow \overline{z} $} ($(xandyF1g2) + (1.8, 0)$);
                \draw ($(x1) + (0, -1ex)$)|- (xandyF1g2.input 1);
                \draw [line width=0.25mm,   red] (notx2.output)
                -- ([xshift=0cm]notx2.output) |- (xandyF1g2.input 2);
                \draw [line width=0.25mm,   red] (notx3.output)
                -- ([xshift=0cm]notx3.output) |- (xandyF1g2.input 3);
                \draw [line width=0.25mm,   red] (notx4.output)
                -- ([xshift=0cm]notx4.output) |- (xandyF1g2.input 4);
     %% ***Function F1 : Gate for term n° 4 [ {'default': "w↑x'↑y'↑z'", 'formatted': "w↑x'↑y'↑z'"} ]***
           \node[nand gate US, draw, rotate=0, logic gate inputs=nnnn] at (2.5, 14.399999999999999) (xandyF1g3) {};
           \draw (xandyF1g3.output) -- node[above]{\scriptsize $ w\uparrow \overline{x}\uparrow \overline{y}\uparrow \overline{z} $} ($(xandyF1g3) + (1.8, 0)$);
                \draw [line width=0.25mm,   red] (notx1.output)
                -- ([xshift=0cm]notx1.output) |- (xandyF1g3.input 1);
                \draw [line width=0.25mm,   red] (notx2.output)
                -- ([xshift=0cm]notx2.output) |- (xandyF1g3.input 2);
                \draw [line width=0.25mm,   red] (notx3.output)
                -- ([xshift=0cm]notx3.output) |- (xandyF1g3.input 3);
                \draw ($(x4) + (0, -1ex)$)|- (xandyF1g3.input 4);
     %% ***Function F1 : Gate for term n° 5 [ {'default': "w'↑x↑y↑z", 'formatted': "w'↑x↑y↑z"} ]***
           \node[nand gate US, draw, rotate=0, logic gate inputs=nnnn] at (2.5, 15.6) (xandyF1g4) {};
           \draw (xandyF1g4.output) -- node[above]{\scriptsize $ \overline{w}\uparrow x\uparrow y\uparrow z $} ($(xandyF1g4) + (1.8, 0)$);
                \draw ($(x1) + (0, -1ex)$)|- (xandyF1g4.input 1);
                \draw ($(x2) + (0, -1ex)$)|- (xandyF1g4.input 2);
                \draw ($(x3) + (0, -1ex)$)|- (xandyF1g4.input 3);
                \draw [line width=0.25mm,   red] (notx4.output)
                -- ([xshift=0cm]notx4.output) |- (xandyF1g4.input 4);
     %% ***Function F1 : Gate for term n° 6 [ {'default': "w↑x'↑y↑z", 'formatted': "w↑x'↑y↑z"} ]***
           \node[nand gate US, draw, rotate=0, logic gate inputs=nnnn] at (2.5, 16.8) (xandyF1g5) {};
           \draw (xandyF1g5.output) -- node[above]{\scriptsize $ w\uparrow \overline{x}\uparrow y\uparrow z $} ($(xandyF1g5) + (1.8, 0)$);
                \draw [line width=0.25mm,   red] (notx1.output)
                -- ([xshift=0cm]notx1.output) |- (xandyF1g5.input 1);
                \draw ($(x2) + (0, -1ex)$)|- (xandyF1g5.input 2);
                \draw ($(x3) + (0, -1ex)$)|- (xandyF1g5.input 3);
                \draw ($(x4) + (0, -1ex)$)|- (xandyF1g5.input 4);
     %% ***Function F1 : Gate for term n° 7 [ {'default': "w↑x↑y'↑z", 'formatted': "w↑x↑y'↑z"} ]***
           \node[nand gate US, draw, rotate=0, logic gate inputs=nnnn] at (2.5, 18.0) (xandyF1g6) {};
           \draw (xandyF1g6.output) -- node[above]{\scriptsize $ w\uparrow x\uparrow \overline{y}\uparrow z $} ($(xandyF1g6) + (1.8, 0)$);
                \draw ($(x1) + (0, -1ex)$)|- (xandyF1g6.input 1);
                \draw [line width=0.25mm,   red] (notx2.output)
                -- ([xshift=0cm]notx2.output) |- (xandyF1g6.input 2);
                \draw ($(x3) + (0, -1ex)$)|- (xandyF1g6.input 3);
                \draw ($(x4) + (0, -1ex)$)|- (xandyF1g6.input 4);
     %% ***Function F1 : Gate for term n° 8 [ {'default': "w↑x↑y↑z'", 'formatted': "w↑x↑y↑z'"} ]***
           \node[nand gate US, draw, rotate=0, logic gate inputs=nnnn] at (2.5, 19.2) (xandyF1g7) {};
           \draw (xandyF1g7.output) -- node[above]{\scriptsize $ w\uparrow x\uparrow y\uparrow \overline{z} $} ($(xandyF1g7) + (1.8, 0)$);
                \draw ($(x1) + (0, -1ex)$)|- (xandyF1g7.input 1);
                \draw ($(x2) + (0, -1ex)$)|- (xandyF1g7.input 2);
                \draw [line width=0.25mm,   red] (notx3.output)
                -- ([xshift=0cm]notx3.output) |- (xandyF1g7.input 3);
                \draw ($(x4) + (0, -1ex)$)|- (xandyF1g7.input 4);


    %% y_pos : the position of OR gate according to their related gates

    %% Function F1 Large OR Gate

        \node[nand gate US, draw, rotate=0, logic gate inputs=nnnnnnnnn] at (\FunctionPos, 15.0) (xoryF1) {};
        \draw (xoryF1.output) -- node[above]{\scriptsize $F1$} ($(xoryF1.east) + (+3ex, 0)$);



             \draw (xandyF1g0.output) -- ++(1.6,0) |- (xoryF1.input 8);



             \draw (xandyF1g1.output) -- ++(1.4500000000000002,0) |- (xoryF1.input 7);



             \draw (xandyF1g2.output) -- ++(1.5,0) |- (xoryF1.input 6);



             \draw (xandyF1g3.output) -- ++(1.55,0) |- (xoryF1.input 5);



             \draw (xandyF1g4.output) -- ++(1.6,0) |- (xoryF1.input 4);



             \draw (xandyF1g5.output) -- ++(1.6500000000000001,0) |- (xoryF1.input 3);



             \draw (xandyF1g6.output) -- ++(1.7000000000000002,0) |- (xoryF1.input 2);



             \draw (xandyF1g7.output) -- ++(1.75,0) |- (xoryF1.input 1);


     %% ***Function F2 : Gate for term n° 1 [ {'default': "w'↑y↑z'", 'formatted': "w'↑y↑z'"} ]***
           \node[nand gate US, draw, rotate=0, logic gate inputs=nnnn] at (2.5, 20.4) (xandyF2g0) {};
           \draw (xandyF2g0.output) -- node[above]{\scriptsize $ \overline{w}\uparrow y\uparrow \overline{z} $} ($(xandyF2g0) + (1.8, 0)$);
                \draw ($(x2) + (0, -1ex)$)|- (xandyF2g0.input 1);
                \draw [line width=0.25mm,   red] (notx3.output)
                -- ([xshift=0cm]notx3.output) |- (xandyF2g0.input 2);
                \draw [line width=0.25mm,   red] (notx4.output)
                -- ([xshift=0cm]notx4.output) |- (xandyF2g0.input 3);
     %% ***Function F2 : Gate for term n° 2 [ {'default': "w↑y'↑z", 'formatted': "w↑y'↑z"} ]***
           \node[nand gate US, draw, rotate=0, logic gate inputs=nnnn] at (2.5, 21.599999999999998) (xandyF2g1) {};
           \draw (xandyF2g1.output) -- node[above]{\scriptsize $ w\uparrow \overline{y}\uparrow z $} ($(xandyF2g1) + (1.8, 0)$);
                \draw [line width=0.25mm,   red] (notx2.output)
                -- ([xshift=0cm]notx2.output) |- (xandyF2g1.input 1);
                \draw ($(x3) + (0, -1ex)$)|- (xandyF2g1.input 2);
                \draw ($(x4) + (0, -1ex)$)|- (xandyF2g1.input 3);
     %% ***Function F2 : Gate for term n° 3 [ {'default': 'x', 'formatted': 'x'} ]***
           \node at (2.5, 22.8) (xandyF2g2) {};
           \draw (xandyF2g2) -- node[above]{\scriptsize $ x $} ($(xandyF2g2) + (1.2, 0)$);

                \draw ($(x1) + (0, -1ex)$)|- (xandyF2g2.east);



    %% y_pos : the position of OR gate according to their related gates

    %% Function F2 Large OR Gate

        \node[nand gate US, draw, rotate=0, logic gate inputs=nnnn] at (\FunctionPos, 21.599999999999998) (xoryF2) {};
        \draw (xoryF2.output) -- node[above]{\scriptsize $F2$} ($(xoryF2.east) + (+3ex, 0)$);



             \draw (xandyF2g0.output) -- ++(1.6,0) |- (xoryF2.input 3);



             \draw (xandyF2g1.output) -- ++(1.6,0) |- (xoryF2.input 2);



             \draw (xandyF2g2) -- ++(1.6500000000000001,0) |- (xoryF2.input 1);


     %% ***Function F3 : Gate for term n° 1 [ {'default': "x'", 'formatted': "x'"} ]***
           \node at (2.5, 24.0) (xandyF3g0) {};
           \draw (xandyF3g0) -- node[above]{\scriptsize $ x' $} ($(xandyF3g0) + (1.2, 0)$);

                \draw [line width=0.25mm,   red] (notx1.output)
                -- ([xshift=0cm]notx1.output) |- (xandyF3g0.east);



    %% y_pos : the position of OR gate according to their related gates

    %% Function F3 Large OR Gate

        \node at (\FunctionPos, 24.0) (xoryF3) {};
        \draw (xoryF3)  node[above]{\scriptsize $F3$} ($(xoryF3.east) + (+3ex, 0)$);



             \draw (xandyF3g0.east) -- (xoryF3);


 \end{tikzpicture}



\pagebreak

\paragraph{Q1}




\subsection*{Encode the following text into Ascii}
\subsection*{Encode the following text into Ascii}
    \textbf{Text:} ''\aRL{ Train Your Mind }''

\subsection*{Decode the following Ascii codes into text}
\textbf{Codes:} ['0x54', '0x72', '0x61', '0x69', '0x6e', '0x20', '0x59', '0x6f', '0x75', '0x72', '0x20', '0x4d', '0x69', '0x6e', '0x64']



\hrule width 1\linewidth
\pagebreak

\subsection{Correction}


\paragraph{Q1}




\subsection*{Encode the following text into Ascii}
      \textbf{Encode the following text into Ascii}\\
  \textbf{Text:} \aRL{"Train Your Mind"}\\[6pt]

      \begin{tabular}{|c|c|c|c|c|c|c|c|c|c|c|}
        \hline
        Character
          &                 T
          &                 r
          &                 a
          &                 i
          &                 n
          &                 space
          &                 Y
          &                 o
          &                 u
          &                 r
 \\
        \hline
        Ascii Code Point
          & 0x54
          & 0x72
          & 0x61
          & 0x69
          & 0x6e
          & 0x20
          & 0x59
          & 0x6f
          & 0x75
          & 0x72
 \\
        \hline
      \end{tabular}


    \vspace{1em} % spacing between tables

      \begin{tabular}{|c|c|c|c|c|c|}
        \hline
        Character
          &                 space
          &                 M
          &                 i
          &                 n
          &                 d
 \\
        \hline
        Ascii Code Point
          & 0x20
          & 0x4d
          & 0x69
          & 0x6e
          & 0x64
 \\
        \hline
      \end{tabular}


    \vspace{1em} % spacing between tables


          \textbf{Decode the following Ascii codes into text}\\
      \textbf{Codes:} ['0x54', '0x72', '0x61', '0x69', '0x6e', '0x20', '0x59', '0x6f', '0x75', '0x72', '0x20', '0x4d', '0x69', '0x6e', '0x64']\\[6pt]



      \begin{tabular}{|c|c|c|c|c|c|c|c|c|c|c|}
        \hline
        Ascii Code Point
          & 0x54
          & 0x72
          & 0x61
          & 0x69
          & 0x6e
          & 0x20
          & 0x59
          & 0x6f
          & 0x75
          & 0x72
 \\
        \hline
        Character
          &                 T
          &                 r
          &                 a
          &                 i
          &                 n
          &                 space
          &                 Y
          &                 o
          &                 u
          &                 r
 \\
        \hline
      \end{tabular}

    \vspace{1em} % spacing between tables



      \begin{tabular}{|c|c|c|c|c|c|}
        \hline
        Ascii Code Point
          & 0x20
          & 0x4d
          & 0x69
          & 0x6e
          & 0x64
 \\
        \hline
        Character
          &                 space
          &                 M
          &                 i
          &                 n
          &                 d
 \\
        \hline
      \end{tabular}

    \vspace{1em} % spacing between tables


\pagebreak

\paragraph{Q1}





    \subsection*{Encode the following numbers into BCD, then illustrate the addition in BCD:}
    \textbf{A = } 93952
    \textbf{B = } 73439
    \subsection*{Encode the following numbers into Excess3, then illustrate the addition in Excess3:}
    \textbf{A = } 93952
    \textbf{B = } 73439

  % endif answer










\hrule width 1\linewidth
\pagebreak

\subsection{Correction}


\paragraph{Q1}






$(93952)_{10} = (1001 0011 1001 0101 0010)_{BCD}$ \\[6pt]

$(93952)_{10} = (1100 0110 1100 1000 0101)_{Excess\text{-}3}$ \\[6pt]

\textbf{Explanation:} \\[6pt]

      \textit{Base 10 to BCD:} \\[6pt]
    
    \textbf{Decimal to BCD} \\

    \begin{tabular}{|c|c|c|c|c|}
    \hline
 9  &  3  &  9  &  5  &  2  \\
    \hline
 1001  &  0011  &  1001  &  0101  &  0010  \\
    \hline
    \end{tabular}



      \textit{Base 10 to Excess-3:} \\[6pt]
    
    \textbf{Decimal to X3} \\

    \begin{tabular}{|c|c|c|c|c|}
    \hline
 9  &  3  &  9  &  5  &  2  \\
    \hline
 1100  &  0110  &  1100  &  1000  &  0101  \\
    \hline
    \end{tabular}




      \textit{BCD to Base 10:} \\[6pt]
    
    \textbf{ BCD to Decimal} \\

    \begin{tabular}{|c|c|c|c|c|}
    \hline
 1001  &  0011  &  1001  &  0101  &  0010  \\
    \hline
 9  &  3  &  9  &  5  &  2  \\
    \hline
    \end{tabular}



      \textit{Excess-3 to Base 10:} \\[6pt]
        
    \textbf{ X3 to Decimal} \\

    \begin{tabular}{|c|c|c|c|c|}
    \hline
 1100  &  0110  &  1100  &  1000  &  0101  \\
    \hline
 9  &  3  &  9  &  5  &  2  \\
    \hline
    \end{tabular}




  
\section*{ BCD Addition Explanation }

Decimal calculation: $93952 + 73439 = 167391$

\vspace{0.3cm}

\begin{tabular}{l|rrrrrr}
\textbf{Carry In} &  {\scriptsize 1}  &  {\scriptsize }  &  {\scriptsize 1}  &  {\scriptsize }  &  {\scriptsize 1}  &  {\scriptsize }  \\
\textbf{A (dec)} &  0  &  9  &  3  &  9  &  5  &  2  \\
\textbf{B (dec)} &  0  &  7  &  3  &  4  &  3  &  9  \\
\hline
\textbf{Final Digit (dec)} &  1  &  6  &  7  &  3  &  9  &  1  \\
\end{tabular}

\vspace{0.3cm}

\textbf{Addition in BCD}
\vspace{0.3cm}

\begin{tabular}{l|rrrrrr}
\textbf{Carry In} &  {\scriptsize 1}  &  {\scriptsize }  &  {\scriptsize }  &  {\scriptsize }  &  {\scriptsize }  &  {\scriptsize }  \\
\textbf{A (bin)} &  0000  &  1001  &  0011  &  1001  &  0101  &  0010  \\
\textbf{B (bin)} &  0000  &  0111  &  0011  &  0100  &  0011  &  1001 \\\hline

\textbf{Carry Out} &  {\scriptsize }   &  {\scriptsize }   &  {\scriptsize 1}   &  {\scriptsize }   &  {\scriptsize 1}   &  {\scriptsize }   \\
\textbf{Raw Sum (bin)} &  0001  &  0000  &  0110  &  1101  &  1000  &  1011  \\
\textbf{Correction} &   &   +110  &   &   +110  &   &   +110  \\
\hline

\textbf{Final Digit (bin)} &  0001  &  0110  &  0111  &  0011  &  1001  &  0001  \\
\textbf{Final Digit (dec)} &  1  &  6  &  7  &  3  &  9  &  1  \\
\end{tabular}




  
\section*{ EXCESS3 Addition Explanation }

Decimal calculation: $93952 + 73439 = 167391$

\vspace{0.3cm}

\begin{tabular}{l|rrrrrr}
\textbf{Carry In} &  {\scriptsize 1}  &  {\scriptsize }  &  {\scriptsize 1}  &  {\scriptsize }  &  {\scriptsize 1}  &  {\scriptsize }  \\
\textbf{A (dec)} &  0  &  9  &  3  &  9  &  5  &  2  \\
\textbf{B (dec)} &  0  &  7  &  3  &  4  &  3  &  9  \\
\hline
\textbf{Final Digit (dec)} &  1  &  6  &  7  &  3  &  9  &  1  \\
\end{tabular}

\vspace{0.3cm}

\textbf{Addition in EXCESS3}
\vspace{0.3cm}

\begin{tabular}{l|rrrrrr}
\textbf{Carry In} &  {\scriptsize 1}  &  {\scriptsize }  &  {\scriptsize 1}  &  {\scriptsize }  &  {\scriptsize 1}  &  {\scriptsize }  \\
\textbf{A (bin)} &  0011  &  1100  &  0110  &  1100  &  1000  &  0101  \\
\textbf{B (bin)} &  0011  &  1010  &  0110  &  0111  &  0110  &  1100 \\\hline

\textbf{Raw Sum (bin)} &  0111  &  0110  &  1101  &  0011  &  1111  &  0001  \\
\textbf{Correction} &  -11  &  +11  &  -11  &  +11  &  -11  &  +11  \\
\hline

\textbf{Final Digit (bin)} &  0100  &  1001  &  1010  &  0110  &  1100  &  0100  \\
\textbf{Final Digit (dec)} &  1  &  6  &  7  &  3  &  9  &  1  \\
\end{tabular}




  % endif answer








\pagebreak

\paragraph{Q1}

Unknown command category for 'gray'


\hrule width 1\linewidth
\pagebreak

\subsection{Correction}


\paragraph{Q1}

Answer
\pagebreak

\paragraph{Q1}

Complete the following timing diagram : \hfill\aRL{أكمل المخطط الزمني الآتي}


\paragraph{ Timing diagram  }\aRL{المخطط الزمني}

\begin{tikztimingtable}
%[timing/slope=.0, timing/draw grid]
[timing/slope=.0,  timing/wscale=1, scale=2.5, line width=1.25pt]
% horloge signal
$H$ & [timing/c/rising arrows] C24{1C}
\\
$J$ &1L 1L 1L 1H 1H 1L 1L 1H 1L 1L 1L 1L 1L 1L 1L 1L 1L 1L 1L 1H 1H 1H 1L 1L \\
$K$ &1H 1H 1L 1L 1L 1H 1L 1L 1L 1H 1L 1L 1L 1H 1L 1H 1L 1L 1L 1L 1L 1L 1L 1L \\
$D$ &1L 1H 1L 1L 1H 1L 1L 1L 1H 1L 1H 1H 1L 1L 1L 1L 1L 1L 1L 1L 1L 1H 1H 1L \\
$Q$ &1L \\
$Q0$ &1L \\
$Q0'$ &1H \\
$Q1$ &1L 
\\
\extracode
\begin{scope}[gray,semitransparent,densely dotted,thin]
\horlines{}
\vertlines{0}
\end{scope}
\begin{scope}[red,semitransparent,thin]
\vertlines{1,3,...,\twidth}
\end{scope}

\end{tikztimingtable}






 % end answer


\hrule width 1\linewidth
\pagebreak

\subsection{Correction}


\paragraph{Q1}

Complete the following timing diagram : \hfill\aRL{أكمل المخطط الزمني الآتي}


\paragraph{ Timing diagram  }\aRL{المخطط الزمني}

\begin{tikztimingtable}
%[timing/slope=.0, timing/draw grid]
[timing/slope=.0,  timing/wscale=1, scale=2.5, line width=1.25pt]
% horloge signal
$H$ & [timing/c/rising arrows] C24{1C}
\\
$J$ &1L 1L 1L 1H 1H 1L 1L 1H 1L 1L 1L 1L 1L 1L 1L 1L 1L 1L 1L 1H 1H 1H 1L 1L \\
$K$ &1H 1H 1L 1L 1L 1H 1L 1L 1L 1H 1L 1L 1L 1H 1L 1H 1L 1L 1L 1L 1L 1L 1L 1L \\
$D$ &1L 1H 1L 1L 1H 1L 1L 1L 1H 1L 1H 1H 1L 1L 1L 1L 1L 1L 1L 1L 1L 1H 1H 1L \\
$Q$ &1L 2L 1H 1L 1L 1H 1L 1L 1L 1H 1L 1H 1H 1L 1L 1L 1L 1L 1L 1L 1L 1L 1H 1H 1L \\
$Q0$ &1L \\
$Q0'$ &1H \\
$Q1$ &1L 
\\
\extracode
\begin{scope}[gray,semitransparent,densely dotted,thin]
\horlines{}
\vertlines{0}
\end{scope}
\begin{scope}[red,semitransparent,thin]
\vertlines{1,3,...,\twidth}
\end{scope}

\end{tikztimingtable}






 %  end if debug
 % end answer
\pagebreak

\paragraph{Q1}

Unknown command category for 'flip'


\hrule width 1\linewidth
\pagebreak

\subsection{Correction}


\paragraph{Q1}

Answer
\pagebreak

\paragraph{Q1}

Unknown command category for 'counter'


\hrule width 1\linewidth
\pagebreak

\subsection{Correction}


\paragraph{Q1}

Answer
\pagebreak

\paragraph{Q1}

Unknown command category for 'register'


\hrule width 1\linewidth
\pagebreak

\subsection{Correction}


\paragraph{Q1}

Answer
\pagebreak